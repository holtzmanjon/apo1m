\documentclass{article}

\usepackage{html}

\begin{document}

\section{Modifications for NMSU 1m tertiary rotation:}

\subsection{Problem}

The top of the stovepipe (from the primary mirror cell) has a flange
that extends inward and restricts the bore. Hence the as-built instrument, 
which has a larger outer diameter, cannot be inserted from the top end as
planned.

\subsection{Telescope model modifications}

 At the top of the stovepipe, there
is a welded flange that has an inner bore diameter of 9.984+/-0.001
inches. This flange is 0.40" thick and is welded underneath (so
that there is \underline{not} a clean 90 degree inner joint); we \underline{estimate} that
the bead of the weld extends in 3/8 inch at the worst spot (and, probably,
3/8 inch down as well).  The flange has a bolt hole pattern that
consists of 6 bolt holes (currently threaded for 1/4-20 bolts) centered
on an 11-inch diameter circle. There are two 1/4" pins immediately
inside two of the holes (spaced by 180 degrees) that are on a 10.365 inch
diameter circle (although we have now removed those pins).

See a few \htmladdnormallink{photos here}{../docs/photos/peter}.

\subsection{Possible solutions}

\subsubsection{Option A}
  Mount instrument from below. Construct a new reducing ring
that mounts to the outer rim of the tertiary mechanism housing and
extends up into the inner bore of the stovepipe flange. The lip would
provide a centering mechanism, and would be machined to provide a close
fit. Construct a new top annular plate that mounts to the telesopce via the
11-inch bolt pattern, but extends inward enough to provide holes to
mount to the tertiary housing. This annulus would be bolted to the telescope.
The tertiary mechanism would then be inserted in the back end, held
up so that the lip extends into the inner bore (perhaps chamfering
the lip to make it slip into place easier), then a second person
would insert bolts from the front end and screw them in.

To avoid having to do this more than  once (or a few times), make
the outer diameter of the annular plate less than the OD of the
stovepipe, so that the primary could be lifted out without ever
needing to remove the tertiary rotation mechanism; we could construct
(easily-removable) brackets for holding the baffle. A \textit{rough} measurement
of the outer diameter of the stovepipe was about 11 15/16 inches.

Since the whole mechanism will now be lower by the 
thickness of the reducing plate, the
housing needs to be shortened by an equivalent amount to keep the
mirror in the same location. Possible issue: given the cutouts in
the housing, can it be shortened enough?  Matt Bradstreet measured the
housing with an inner diameter of 9.68 inches, outer diameter 10.41", 
thickness .375 inches, total height 3.1875 inches, and height of metal
remaining above the largest cutout 1.1875 inches (and just a tiny bit
thicker about the other cutouts).

\subsubsection{Option B}
 Mount instrument from below. Construct a new reducing
plate that mounts to the outer rim of the tertiary mechanism housing
and provides a lip that extends up into the inner bore
of the stovepipe flange. The lip would provides a centering mechanism.
The reducing plate mounts to the telescope using bolts that pass
through the telescope flange (skip a top plate entirely), and then
thread into the reducing plate. This will require, however, drilling
out the threads on the telescope flange, so that the bolts can pass
through and thread into the reducing flange (or else the use of
smaller bolts, e.g. \#10. The outer diameter of the reducing plate
needs to be less than the full bore of the stovepipe to allow
clearance for it to be inserted from the back end. It needs to be
large enough to accomodate the 11-inch bolt pattern from above.
However, this 11+ inch circle does run into the welding of the
telescope flange, so it will be necessary to construct a "3-step"
reducing plate, where the intermediate step will be flush up against
the back of the flange, but with a small enough outer diameter to
fit within the weld beading (say 10.75 inches).  We measured the
full bore at 11.549 inches (what was your measurement?), and thus
the third step of the reducing plate would want to have an outer
diameter just big enough to accept the 11-inch bolt pattern.  There
are clearance issues with inserting it in the back end, as the hole
in the primary mirror cell is out of round; the outer diameter
should be no larger than 11.35 inches. This is not really big enough
to accomodate the threaded 11-inch diameter bolt hole pattern (using
\# 10 bolts instead of 1/4-20 would help a bit). To improve the
situation, we propose to use only 4 of the 6 bolt holts on the
11-inch circle, and then to use the two pin holes (on the 10.365
inch circle) for the two remaining bolts; these two should have
good purchase on the threads.

The reducing plate would mount to the rotation housing; the bolts
for this would likely need to be rotated relative to the bolts
coming through the telescope flange discussed above to avoid
collision. Since the whole mechanism will now be lower by the
thickness of the telescope flange plus the thickness of the reducing
plate, the housing needs to be shortened by an equivalent amount
to keep the mirror in the same location; I think this would mean
removing even more of the housing than Option A?

The tertiary mechanism would then be inserted in the back end, held
up so that the lip extends into the inner bore (perhaps chamfering
the lip to make it slip into place easier), then a second person
would insert bolts from the front end and screw them in.

\subsubsection{Option C}
 Leave tertiary mechanism as is, and construct a new telescope 
stovepipe, e.g. to remove the flange or make it have a larger bore.
Major problem here is that this requires pulling the primary mirror cell
(requires crane/lift work), removing the primary mirror, removing
the existing stovepipe and installing a new one. This work requires
significant external help, and could not be attempted until September/October
time frame earlies. It also would require construction of a large new
piece.
 
\end{document} 
