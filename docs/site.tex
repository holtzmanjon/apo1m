\documentclass{article}


\textwidth 6.5in
\hoffset -1in
\textheight 9.5in

\textheight 7.5in
\hoffset -0.5in
\voffset -1in
\textwidth 9.5in

\begin{document}

\setcounter{section}{1}
\setcounter{subsection}{0}
\begin{center}
\textbf{\large APO STAFF INFORMATION  :  NMSU 1M TELESCOPE}
\end{center}
\vskip 0.25in

NMSU operates a robotic 1m telescope at Apache Point Observatory.
Since the telescope is run robotically, APO staff are an 
invaluable resource for noticing any problematic behavior
that may affect the health and safety of the telescope and
on-site personnel. 

NMSU pays an annual fee to cover basic services at APO, and
we reimburse various technical staff members on a per-hour
basis when they work on the telescope.

NMSU understands that APO observers/staff have no formal obligation
regarding the NMSU 1m and that all 3.5m or 2.5m functions take
precedence over any 1m function.

Nonetheless, NMSU would strongly appreciate if APO staff would
be alert to any possible issues with the telescope, and wants
to make sure that all APO staff have access to contact information,
as well as some very basic operational items.

\subsection{Things to be alert for}

We would appreciate if observers/staff note any peculiar
behavior or items that might affect the health of the telescope. 
In particular, it would be great if people could be alert for:
\begin{itemize}
\item dome open after 3.5m dome is closed (by more than several minutes)
\item any especially loud noises from the telescope
\end{itemize}

In particular, we want to avoid:

\begin{itemize}
\item sunlight on the mirror, which could potentially be a fire
hazard
\item damage to telescope mirrors
\item damage to telescope drive surfaces
\end{itemize}

\subsection{What to do}

If something out of the ordinary is noted, but there is no obvious
immediate threat to site or telescope, attempt to contact (in prioritized
order):
\begin{enumerate}
\item Jon Holtzman, holtz@nmsu.edu, 575-646-8181 (work), 
575-522-5065 (home), 575-621-7380 (cell)
\item Tom Harrison, tharriso@nmsu.edu, 505-646-3628 (office) , 505-526-2990 (home)
\item Jeff Coughlin, jlcough@nmsu.edu, (404) 617-6819 (cell)
\item Mark Klaene

\end{enumerate}

If something threatens telescope or personnel safety, take action first, but
please contact above after action has been taken.

\subsubsection{Dome is open}

\begin{itemize}
  \item Keep sun off of mirror: close dome shutter(s) or if it doesn't close, 
rotate away from sun (see operational instructions belwo)
  \item Keep rain/weather off of telescope: close dome with motors. If that
fails, close with crank. If that fails, cover telescope with tarp (one can
be found in the bottom drawer of the toolchest in the 1m dome).
\end{itemize}

\subsubsection{Very loud noise from telescope drive}

\begin{itemize}
  \item Kill telescope power using emergency power off switch on pier inside
door.
\end{itemize}

\subsubsection{Lights are on}

Lights controlled by the power switches inside the door (outside light, and
two sets of wall mounted lights) should NEVER be on, unless being used by
people on site. There are two toggle switches, and one dimmer knob; please
turn them off completely.

There are two clamp lights that are remotely powered for use with remote
webcam viewing. They are turned on during telescope initialization
and if someone is trying to diagnose a problem. These lights can be
turned off by computer control from newton using the command: power1m
off lights. Ideally, contact Jon Holtzman before doing this, in case
something important is being looked at.

\subsubsection{Louvers are open}

Louvers can be closed by computer control from newton, using the command:
power1m off louvers. If big dome fan is running, this should be turned off
before louvers are closed: power1m off domefan.

\subsection{Basic operation}

\subsubsection{Emergency power off}

There is a red button on the telescope pier immediate in front of the
door. If this is hit, all power to telescope motors will be killed. This
will also trigger a watchdog circuit that will attempt to close the upper
dome shutter (the lower dome shutter has no watchdog). If the emergency
power switch is hit, please notify Jon Holtzman.

\subsubsection{Dome control}

There is a control box for the dome rotation and upper dome shutter
inside on the N wall (to the left when you walk in).  Normally, the
upper switch should be in "Automatic" mode, but manual operation can be
achieved by putting it in "Manual" mode. Once in manual mode, the lower
switches can be used to control dome rotation, and to raise/lower the
upper dome shutter. In general, please return switch to "Automatic" mode.

We have had some problems with fuses blowing in the upper dome circuit. 
In the bottom of the control box, the left fuse is for the dome rotation;
this is a spare next to it. In the right side, the fuse has been replaced
by a circuit breaker; if the white button has popped out of the bottom of the
box, the circuit breaker has been tripped, and pushing it in will reset it.
If manual operation fails, the circuit breaker and fuse can be inspected and 
reset/replaced if necessary.

If the dome cannot be closed by motor control, it can usually be closed with
the hand crank that is located in the dome. The hook goes around the small
eyeloop at the top of the dome, and then the crank can be rotated to close
the shutter. 

The control box for the lower dome shutter is on the S wall (to the right
when you walk again). There is a toggle switch for manual operation. To
raise or lower the dome in manual mode, select raise/lower, then apply
the power switch. The lower shutter opens and closes slowly. Note that
the lower dome shutter cannot close fully if the upper dome shutter is
closed; for complete closing, the lower dome shutter must be closed first.

\end{document}
