\documentclass{letter}

\textwidth 6.5in
\textheight 9in
\hoffset -1in
\voffset -1in
 
\signature{Jon Holtzman\\Associate Professor, NMSU}
 
\begin{document}
\begin{letter}
{
Rayleigh Optical Corporation\\
3720 Commerce Drive\\
Suite 1112\\
Baltimore MD 21227 
}
\opening{Dear David and Vilma,}

  You should now be in possession of much of the optical hardware from our
telescope! I enclose a packing list of everything we have included in
the shipment; please confirm that you have received all of the listed parts.

  Please be advised of the following information:

\noindent{Primary mirror}


\begin{enumerate}
  \item When you open the crate, remove the top layer of styrofoam, and top
   layer of soft foam. In one of the
   corners of the next softer foam layer, there is a 2g shock sensor. Please
   remove it carefully and note whether it has been tripped before proceeding
   with removing the mirror.

  \item We did a fairly careful inspection of the primary before packing it up.
   When you look at the mirror from the back, one side is marked ``TOP''. 
   There is a significant crack in the back plate located at about the
   2 o'clock position (looking from the back of the mirror), near the outer
   edge, starting from one of the small holes in the back plate and 
   extending all the way to the edge of the back plate. We saw no evidence
   for any crack in the side wall at this (or any other) location. It 
   appears that the back plate has separated from the ribs around the location
   of this crack.

   The only other possible problem we noted was at approximately the 6:30
   position in one or two of the innermost cells, where it seemed there might
   be some structural flaws between the back plate and the ribs, although
   we are not even sure that this is a real feature.
  
   Our inspection was mostly limited to the sides and back; because of the
   aluminum, we can't see into the front. From our last aluminization,
   however, no particular problems were noted on the front of the mirror.

  \item The mirror was shipped with a band around it which can be used
   for lifting the mirror. This band has two clasps which hold it
   together and act to hold the mirror. Before using the band for lifting,
   confirm that the clasps are firmly engaged.

  \item The lifting band is used in conjunction with a spreader bar and two
   metal lifting straps, all of which are included in the shipment. The
   spreader bar should be attached to a hoist/crane. The two straps should
   be attached to the spreader bar by passing the bolts through the holes
   in one end of the straps. The bar and straps should then be positioned 
   over the mirror. Lower the bar, taking special care to hold the straps
   away from the mirror, until the ``keyholes'' can be placed over the
   two handles located 180 degrees apart on the lifting band. 

  \item When the mirror is lifted by the band, someone should keep a hand on
   a side to prevent it from inadvertently rotating, or to manual rotate
   it to any desired orientation.

  \item Before placing the mirror in the mirror cell, you MUST insert the
   nine plastic feet into the ends of the aluminum tripods in the mirror
   cell. We added some spacers into the tripod supports, so that if you use the 
   black feet, the mirror surface will lie well above the sides of the
   mirror cell. Without the central post (which we removed), you will
   need to be very careful to lower the mirror into a central location
   within the cell. You will need to orient the mirror such that the
   handles on the band fall at two vertices of the mirror cell, in order
   to have enough clearance to get the mirror into the cell.

  \item Once the mirror is resting in the cell, you will need to release the
   band, which is almost certainly providing enough stresses on the front
   plate to affect the figure. To do so, you will need to rotate the mirror
   in the support by about 30 degrees, so that the clamps are at one of
   the vertices and you can get your hand to them (I hope you do not mind
   rotating the mirror by hand.)  Release the clamps, and carefully lower
   the band down into the cell and out of the way. I don't see any reason
   for you to remove the band from the cell; leaving it there should make it
   easier to put it back on later. If you need to take it out, you'll need
   several pairs of hands to do it carefully to insure it doesn't hit
   the mirror; you may also need to rotate it. 

  \item When it comes to lifting the mirror again, you will need to reattach
   the lifting band. This probably requires two people to hold the band
   around the mirror, confirming that the bottom and upper lips are positioned
   correctly on the mirror, plus a third person to actually attach and
   tighten the clasps; if you have removed the band entirely and need to
   put it back in the cell, it may be a bit trickier. 
   Having done this, rotate the mirror 30 degrees (by hand)
   so that the handles are at the vertices, attach the lifting straps
   as before, and off you go.
\end{enumerate}

\noindent{Secondary mirror:}
\begin{enumerate}
  \item the secondary mirror is in the thinner small crate. It is face-down in
    the crate. After removing the lid and top layer of foam, you will see
    the back of the mirror. 

  \item we have shipped a tripod structure (the larger one), along with bolts,
    that you can attach to the mirror and use to lift it out if you do
    not want to handle the mirror by hand. This tripod may also be useful
    for you when you want to test the mirror face-down. However, for
    polishing the mirror face-up, the tripod does not provide a flat back
    surface, so you will probably need to rest the mirror on its plug or
    construct something else to put it on.
\end{enumerate}

\noindent{Tertiary mirror:}
\begin{enumerate}

  \item we have shipped the tertiary mirror even though no work on it has been 
    included in our initial contract. If all proceeds well with the primary
    and secondary (without very significant cost increases!), we will 
    negotiate a price to test, and if necessary, repolish the tertiary.

  \item the tertiary is also shipped face down in its box. After removing the
    lid and top layer of foam, you will see the plate which is attached
    to the back of the mirror.

  \item we have shipped a tripod structure (the smaller one), along with bolts,
    that you can attach to the mirror and use to lift it out if you do
    not want to handle the mirror by hand. However, for
    polishing the mirror face-up, the tripod does not provide a flat back
    surface, so you will probably need to rest the mirror on its back or
    construct something else to put it on.

\end{enumerate}

\noindent{Telescope design parameters:}
\begin{enumerate}
  \item for the purposes of constructing your null, the original design for 
    the telescope has:

   \begin{center}
   \begin{tabular}{lcccc}
   NAME & RADIUS & THICKNESS & DIAMETER & CONIC \\
   Primary & -200 & -62.648 & 40 & -1.202461 \\
   Secondary & -127.059 & 49.00 & & -8.434993 \\
   Tertiary & Inf & 41.648 & & \\
   Focal plane & & &
   \end{tabular}
   \end{center}

   Note, however, that we measure a diameter closer to 41 inches rather
   than 40 inches.

   IMPORTANT: The as-built telescope does not perfectly match these
   spacings, so our final modified design may not be exactly as specified
   above.  DO NOT POLISH BOTH OPTICS TO THESE SPECIFICATIONS.
   However, to the best of our knowledge, these were the figures
   originally specified, and hence would be our best-guess knowledge of
   as-built figures, and thus are probably the best figures to use for
   constructing test fixtures.

  \item we have made a moderately careful set of measurements of allowed
    spacings between our elements, so we should be in a position to
    give you a prescription for the secondary once we have as-built
    numbers for how you will polish the primary. 
\end{enumerate}

\noindent{Contact information:}
\begin{itemize}
  \item Your primary contact during the refinishing procedure will be
    Jon Holtzman, holtz@nmsu.edu. As you know, I will be in Granada Spain
    while the work is being done, but should be easily accessible. I will
    provide you a phone number as soon as I have one. If possible, please
    notify me in advance of any times when you know that critical decisions
    with be required, so I can make sure I'm not out of town during these
    times. If anything comes up that would require an on-site visit to
    ROC, it might be able to be arranged.

  \item Secondary contact information, for non-critical decisions/information,
    and general information:
\begin{itemize}
    \item John Barrentine, Apache Point Observatory, (505) 437-6822, 
    jcb@apo.nmsu.edu, is on-site and can provide a variety of information
    about the telescope

    \item Kurt Anderson, New Mexico State University, (505) 646-1032, was
    closely involved with the original polishing of the optics and initial
    installation into the telescope, and is generally knowledgeable

    \item Rene Walterbos, New Mexico State University, (505) 646-6522, 646-4438,
    is the chairman of the Astronomy Department, and can be contacted as
    a responsible person in a position of authority.
\end{itemize}
\end{itemize}

\closing{Sincerely}

\newpage

Items shipped to Rayleigh Optical Corp. via McCollisters Transportation
         6/24/02

\begin{enumerate}
\item 48"x48" crate containing NMSU 1 meter primary mirror, packed in foam
  \begin{itemize}
   \item crate also contains lifting band attached to mirror
  \end{itemize}

\item 24"x24"x7" crate containing NMSU secondary mirror, packed in foam

\item 24"x24"x13" crate containing NMSU tertiary mirror, packed in foam

\item 1 yellow primary mirror cell, includes
  \begin{itemize}
     \item 3 attached tripod structures for primary mirror support
     \item 12 attachment bolts
     \item 2 eye bolts
  \end{itemize}

\item 1 box containing:
  \begin{itemize}
     \item 1 tripod for attachment to secondary mirror, including 3 bolts
     \item 1 tripod for attachment to primary mirror, including 3 bolts
     \item 9 white feet that can be used on primary mirror supports
     \item 9 black feet that can be used on priamry mirror supports (thicker than
       white feet if the mirror needs to be raised)
  \end{itemize}

\item 1 spreader bar for use in lifting mirror

\item 2 metal straps used for lifting primary mirror

\end{enumerate}
\end{letter}
\end{document}
