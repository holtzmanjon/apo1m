\documentclass{article}

\usepackage{html}

\begin{document}

\begin{LARGE}
\begin{center}
\textbf{RIVMOS (Rapid Infrared and Visible Multi-Object Spectrograph) at the APO 3.5m}
\end{center}
\end{LARGE}

As part of the campaign to upgrade and/or replace the aging 
instruments on the 3.5-m telescope, one of our near-term highest 
priorities is to acquire a new IR instrument.  Although GRIM2 continues
to function, several things limit its scientific potential.
In particular, it has a 256x256 detector which has very high noise. This
gives a small field of view, and makes it difficult to get observations
of fainter objects in narrow-band imaging or spectroscopic mode. In
addition, the spectroscopic modes either have very low resolution or
have rather small slit widths at a bit higher resolution. As a result, we
have considered alternative options for improving the IR instrumentation.

The cost of a modern IR instrument is beyond the reach of our 
budgets, but some form of a cost-sharing partnership is being 
considered.  Several partnership possibilities have been explored, 
and one of the most promising is RIVMOS (Rapid Infra-red and Visible 
Multi-Object Spectrograph).

A team at GSFC led by Harvey Moseley and including Alexander Kutynev, Bruce
Woodgate, and Jian Ge (Penn State) is interested in building an IR instrument for use
at the 3.5m. They have funding from NGST for a prototype instrument 
to demonstrate the capabilities of a microshutter array, to be
potentially used for fully configurable multi-object spectroscopy. The
proposed prototype instrument (RIVMOS) has both imaging and spectroscopic
capability, using a 1024x1024 InSb detector. They would like to use the
instrument at the 3.5m, and in exchange, make it available to the ARC
community. Scientifically, one of their motivations for the instrument
is to allow rapid followup, e.g., of gamma ray bursts. As a result, they
would eventually like to see this instrument installed on a dedicated port
(with associated rotator and guider) on the 3.5m.

Bruce Woodgate has represented this team to APO. He
has a long association with APO, and has 
allowed us to offer the use of his Fabry-Perot camera at the 3.5-m. 

At no financial cost to ARC, GSFC is building the camera described 
in the attachments to the "Phase I" level next year, and we plan to 
commission the camera and shutter prototype at the 3.5-m.  Some 
possible collaborative early use of the instrument is possible by ARC 
astronomers.

To fully implement the camera on the telescope as a facility-class 
instrument, in "Phase II" we would need to dedicate a port to the 
instrument so that it can be used for fast-attack targets of 
opportunity.  This would involve a guider and 
rotator, and control software to enable full remote operation. 
Various spectroscopic modes are envisaged.  An optical layout of 
RIVMOS is attached to this e-mail, as is a three-page description of 
the instrument.

We solicit expressions of interest from the ARC user community, and 
in particular ask you to let us know if you would find this 
instrument useful for your scientific programs at the 3.5-m.  Also, 
if you have suggestions for technical specifications for the 
instrument (filter bandpass, spectorcopic dispersions, etc.) let us 
know because there is still time to refine some aspects of the design.

\begin{itemize}
\item \htmladdnormallink{Information from Bruce Woodgate on RIVMOS}{../woodgate/index.html}
\item \htmladdnormallink{Sketch of optical configuration}{../OpticImage.jpg}
\end{itemize}

\end{document}
