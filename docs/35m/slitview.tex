\documentclass[10pt]{article}

\usepackage{makeidx}
\usepackage{epsf}
\usepackage{html}

\includeonly{%
chap1,%
chap2,%
chap3,%
chap4,%
chap5,%
chap6,%
chap7,%
chap8,%
chap9,%
refs,%
index,%
}

\makeindex
\renewcommand{\[}{\begin{eqnarray}}
\renewcommand{\]}{\end{eqnarray}}

\renewcommand{\tilde}{\~\space}

\renewcommand{\floatpagefraction}{0}
\setcounter{topnumber}{1}
\setcounter{totalnumber}{1}

\newenvironment{hanging}{
        \begin{list}{}{
                \labelsep=0pt
                \labelwidth=0pt
                \listparindent=0pt
                \itemindent=-\leftmargini
                \leftmargin=\leftmargini
        }
}{
        \end{list}
}
\newenvironment{references}{
        \begin{hanging}\raggedright
}{
        \end{hanging}
}
\newenvironment{url}{
        \begin{list}{}{
                \labelsep=0pt
                \labelwidth=0pt
                \listparindent=0pt
                \leftmargin=\leftmargini
        }\item\tt
}{
        \end{list}
}
\newenvironment{code}{
        \begin{list}{}{
                \labelsep=0pt
                \labelwidth=0pt
                \listparindent=0pt
                \leftmargin=\leftmargini
        }\item\tt
}{
        \end{list}
}
\newenvironment{example}{
        \begin{list}{}{
                \labelsep=0.25in
                \labelwidth=2.0in
                \leftmargin=2.75in
                \itemindent=0in
                \itemsep=0in
                \parsep=0in
        }
}{
        \end{list}
}

\newcommand{\note}[1]{\marginpar{\raggedright\scriptsize{#1}}}

\makeindex

\title{\Huge\bf DIS Slitviewer User's Manual}
\author{\Large
Jon Holtzman
}
\date{\Large November 2000}

\begin{document}

\setcounter{page}{1}
\pagenumbering{roman}

%\maketitle

\begin{center}
\textbf{DIS Slitviewer User's Manual}
\end{center}
\begin{center}
Jon Holtzman, NMSU, 11/2000
\end{center}

\begin{htmlonly}
Complete document in a single file 
(\htmladdnormallink{PostScript}{../slitview.ps.gz})
(\htmladdnormallink{PDF}{../slitview.pdf})
\end{htmlonly}


%\tableofcontents
%\listoffigures
%\listoftables

%\clearpage

\setcounter{page}{1}
\pagenumbering{arabic}

\section{Introduction}

This manual describes how to operate the Apogee camera which is used as
the slit viewing camera in DIS. This camera is operated by a separate
program outside the normal Remark interface. For instructions on using
DIS itself, see other documentation.

The Apogee slit viewing camera is made by Apogee Instruments and uses
an 512x512 thinned, back-side illuminated SITe CCD for imaging the
DIS slit. The 24$\mu$ pixels give an angular pixel size of $\sim$ 0.31
arcsec/pixel, with a total square field of view of $\sim$ 150 arcsec on
a side.

The reasons for switching from the old SBIG slit viewer to the new system are:
\begin{enumerate}
  \item The new camera has significanly less overhead for readout time and
        image transfer time
  \item Software for the new system should allow in most cases for remote
        operation, which will hopefully lead to better efficiency
  \item The new camera should have better sensitivity than the old one
\end{enumerate}


\section{Requirements and capabilities}

To run the software and be able to see the images, you must be:
\begin{enumerate}
  \item working on a workstation running an X11 window server,
  \item the X11 server must be running in 8-bit PseudoColor mode.
\end{enumerate}

The software will allow you to take images, with optional automatic dark
subtraction, and will display these on your screen. You will be able to
interact with the images to get pixel coordinates, values, etc., and 
adjust the color map both by adjusting the color map or by redisplaying
with different contrast/brightness. Simple image centroiding is possible
which will suggest instrument offsets needed to place the object in
the slit.  Images will be saved by default in /export/images/dis-slit,
although the auto-saving may be turned off.  If desired, the images
can be automatically copies to your local machine using the secure scp,
if you set up an account which allows remote access without a password
(instructions later).

\section{Running/restarting the program}

To start the program, you must first log onto tycho.apo.nmsu.edu using
the visitor1 account (check with APO staff if you don't know the password).
Once logged into tycho, execute:
$$slitview$$
to start the program. In reality, this will actually be running a program
on another computer (offset.apo.nmsu.edu) to which the CCD is physically
connected, but this is all transparent to you. A separate xterm window
should appear on your screen with a status section in the top half, and
a command section in the lower half. At the current time, you should not
resize this window. Once you take your first exposure, a display window
will appear with the image.

Hopefully this won't ever happen, but if the program appears to be totally 
hung up, you should be able to kill it using a CTRL-$\backslash$. Then 
restart it as above.

If you wish to interrupt an exposure (e.g., you entered an exposure time
which is much too long), you should be able to do so using CTRL-C. This
should stop the current exposure and return you to the command prompt.

\section{Commands}

All commands are \textbf{case insensitive}.
The following commands are available:
\begin{hanging}
\item{EXP [$time$]}

Take an exposure of length $time$ seconds.

\item{REPEAT}

Put program in never-ending loop of taking exposures of the last specified
exposure time. Useful for guiding.

\item{STOP}

Stops the never-ending loop. CTRL-C serves the same purpose. The CTRL-C 
stop is immediate, while the STOP command will wait for the current exposure
to terminate.

\item{WINDOW [\textit{x1 x2 y1 y2}]}

Sets up a window region on the CCD to read out. If no arguments are
specified, you will be prompted to mark the desired corners of the
box on the display window. Move the cursor to the desired corners
and hit \textit{any} key (although hitting reserved keys which also
perform other functions (see below) may produce confusing results).
Otherwise, if parameters are specified on the command line, the 
window will be set to region (x1:x2,y1:y2).  When a
subregion is read out, the overhead time for an exposure is shorter than
for a full image. To restore to full-frame operation, use FULL command,
or WINDOW FULL.

\item{FULL}

Restores full-frame operation if a WINDOW command has previously been
issued.

\item {Offset commands}

You can get the program to print out instrument offsets from the cursor position
or from a centroided position around the cursor by moving the cursor into
the image display and using the C (centroid) or I (integer pixel) keys;
see next section on image display window. Note that the program does NOT
communicate at this time with the telescope, so it outputs instrument offsets
\textit{which you will need to tell the telescope to make}.

\item{NEWCENT xc yc}

Changes the defined slit center to (xc,yc). Suggested offsets will be to
this position. The currently defined slit center in shown in the status
section in the upper part of the screen.

\item{FILENAME [$name$]}

Change the default root file name to $name$. Default file name will be of
form \textit{yyyymmdd} where \textit{yyyy}, \textit{mm}, \textit{dd} will be
the UT date when the software is started. The filename extension will be
.nnn.fits, where nnn will be an automatically incrementing number thats
initial value will be the first number for which no file exists when the
program is started. The filename for the next exposure is shown in the
status section in the upper part of the screen.

\item{NEWEXT [$ext$]}

Change the extension for the next file to be the number $ext$. If a filename
with the new extension already exists, \textit{it will be written over}
by the next exposure.

\item{+DARK/-DARK}

Turn on/off automatic dark subtraction for images. If automatic dark
subtraction is on, then a dark exposure will be taken before the light
exposure whenever the exposure time is changed. If subsequent exposures
are taken with the same exposure time, the original dark frame is 
re-used.

\item{FITS}

Set filetype for future image stores to be FITS. Actually, this is the
only allowed filetype, so effectively, this command just turns on the
automatic storage of images (in /export/images/dis-slit). This is ON
by default.

\item{-DISK}

Turn off autosaving of images. Turn this back on using FITS.

\item{+DISPLAY/-DISPLAY}

Turn on/off autodisplay of images after they are taken. On by default.

\item{+XFER/-XFER}

Turn on/off auto-transfer of images to your home institution after they are 
taken.  Before turning on +XFER, you must set up a host and account to
use with the ???? command. Off by default.

\item{SCALE $low\ high$}

Redisplay the current image with greyscale scaling between $low$ and $high$.
You can also adjust the current colormap by playing with the colorbar in
the image display window - see next section.

\item{NEWSCALE}

Sets mode for autoscaling to be based on mean and variance within image.
This is the default.

\item{SKYSCALE}

Sets mode for autoscaling to have black level somewhat below mean level in
image, and white level above it. 

\item{FULLSCALE}

Sets mode for autoscaling of images to be from minimum pixel (black) to
maximum pixel (white).

\item{SAMESCALE}

Sets scaling for future pixels to be identical to the current scale.

\item{SETTEMP \textit{temp}}

Changes the target temperature for the thermoelectric coolers. One must
be a bit cautious not to set this too low or it will cause undue stress
on the coolers. The coolers can probably reach 50 C below ambient. The
default temperature is -30 C. It takes some time before a new temperature
is reached after changing the set point.

\item{QU}

Exits the program.

\end{hanging}

\section{Image display window}

The first time an image is taken, the program opens an Image Display
window on the console, and loads a color map as well as displaying the
image.  Once an image has been displayed, the Display window will accept
interaction asynchronously of commands in the command window,
provided that a wait for input, or any other I/O is not pending.  To
interact with the image, simply move the mouse onto the display window.
The current pixel location of the cursor will be displayed in a frame at
the base of the image display along with the pixel intensity.  The arrow
keys are used for find control (one pixel at a time) of the cursor
position.

\textit{Important. If the image display window fails to open after the
first image, it may be  because of a problem with the program interacting
with your X11 server. Make sure your X11 server is running in 8-bit 
PseudoColor mode; this is the mode which is required, e.g., for IRAF/XIMTOOL.
If insufficient colors are available in the default colormap, the program
will allocate a private colormap to proceed, in which case you may see
the colors on your display switch as you move the cursor into and out of
the display window. If you are using a private colormap, you will need to
move the cursor into the display window to see the images ``normally''. }

By default, the image window comes up as a moderately small window which
shows a view of the image at 1/2 resolution. With mouse keys (described
below) you can zoom into any region at full resolution. Alternatively,
you can resize the window to a larger size and the program will automatically
display images at full resolution if your resized window is large enough.
This can be done at any time and does not require re-starting the program.

The image will come up with (1,1) in the lower left corner. With this
convention, the image appears in the same orientation as it does on the
sky. If the instrument is at 0 position angle, then N will be up and E to
the left; obviously, if the instrument is rotated, this will no longer be
true.

The following mouse buttons and keyboard keys are active \textit{while the mouse
is located on the image display}:

\begin{center}
{\bf Mouse Buttons}\\
\begin{tabular}{ll}
\hline
Button & Function\\
\hline
LEFT  &ZOOM IN, centered on the cursor\\
MIDDLE&ZOOM OUT, centered on the cursor\\
RIGHT &PAN, move the pixel under the cursor to the center\\
\hline
\end{tabular}
\end{center}

\begin{center}
{\bf Keyboard Commands} (mouse in display window, commands are insensitive
to case)\\
\begin{tabular}{cl}
\hline
Key & Function\\
\hline
 R &RESTORE image to the original zoom/pan\\
 C &CENTROID around cursor position, and print out instrument offset required
    to put object at currently defined slit center\\
 I &print out instrument offset required to put cursor position at currently
    defined slit center\\
 +/= &BLINK Forwards through the last 4 images.\\
 -/\_ &BLINK Backwards through the last 4 images.\\
 P &Find the PEAK pixel near the cursor \& jump the cursor there\\
 V &Find the LOWEST pixel ("Valley") near the cursor \& jump the cursor there\\
 \# &"Power Zoom" zoom at the cursor to the maximum zoom factor\\
 H &Toggle between small and full-screen cross-hairs\\
 ] &Clear boxes and stuff off the image display\\
\hline
\end{tabular}
\end{center}
\noindent{\bf Color Bar Adjustment:}

If you place the mouse on the color bar, these commands are available
to adjust the contrast of the image:
\begin{example}
  \item[LOW CONTRAST]{Hold down the LEFT Mouse button, drag the left
       end of the color bar.}

  \item[HIGH CONTRAST]{Hold down the RIGHT Mouse button, drag the right
       end of the color bar.}

  \item[ROLL COLOR MAP]{Hold the MIDDLE Mouse button, "roll" the
       color bar left or right.}
\end{example}
The position of the mouse cursor displays the range of intensities
represented by that color.

Pressing the R key while the mouse is on the color bar restores the
original color map (undoing any change of the contrast or "roll" changes
made with the mouse buttons).


\section{Details and what to do in case of problems}

When you issue the slitview command from tycho, you are actually just
starting a process on the computer to which the slit viewing camera
is actually connected; this computer is offset.apo.nmsu.edu. As a result,
you cannot kill the program on tycho itself. When slitview is started,
it automatically kills any program which is currently talking to the
camera before a new control program is started up. Only one user can
connect to the camera at a given time. There is a command on tycho called
killview which will attempt to kill any existing slitview processes on the
remote machine.

If you have problems with the program, it may be a good idea to directly
log onto offset.apo.nmsu.edu, user arc (password known by on-site personnel).
Check to see if any processes are running the program slitview.linux. If you
kill these, any running versions of the program should exit. On offset,
there is a script, killview which will automatically do this for you. Then
you can try to start the program again using the slitview command.

\section{To Do items}

\begin{itemize}
\item Implement a Gaussian fit/radial plot routine to allow for focus analysis
   within slit viewer application. (HIGH priority but unfortunately may
   require a fair bit of work!)

\item put default target nighttime and daytime temperature, as well as 
   default slit center position, in a configuration file which is read on
   startup.

\item Implement a "smoothing function" for temperature readout to avoid the
   "noise" in temperature readout.

\item note cooler load so one could adjust temperature accordingly

\item Figure out how to get a blinking cursor in the right place in the command 
   window.

\item Figure out how to allow users to resize the command window without 
   screwing things up too badly.

\item Figure out how to get image automatically redrawn if display window is
   resized.

\item Implement communication with the telescope to put telescope parameters
   in image headers.

\item Implement communication with the telescope to allow offsets to be
   directly implemented by the telescope

\item If telescope communication is implemented, implement autoguiduing 
   algorithm.

\item If telescope communication is implemented, put a compass on the image
   showing N and E given the current rotation.

\end{itemize}

\end{document}

