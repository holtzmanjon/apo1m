\documentclass{article}

\usepackage{epsfig}
\usepackage{html}

\begin{document}

\begin{htmlonly}
Get this entire report in \htmladdnormallink{Postscript}{../mar01.ps.gz}
or \htmladdnormallink{PDF}{../mar01.pdf}.
\end{htmlonly}

\begin{center}
\begin{LARGE}
Throughput measurements on the ARC 3.5m, March 2001
\end{LARGE}
\end{center}

\section{Summary}

Throughput measurements made in March 2001 using SPICAM and DIS show that
the throughput of the 3.5m has degraded over the past several years. DIS
measurements suggest that the degradatation is strongly wavelength 
dependent, with essentially no degradation in the near-IR, but a relative
throughput today of $\sim$ 0.75 at 5000 \AA, $\sim$ 0.67 at 4000 \AA,
and 0.5 at 3800\AA\  compared to values just after the last realuminization.
SPICAM measurements are not as good, but show a clear degradation since
1997, with a relative throughput today of $\sim$ 0.7 in the V band; there
is too much discrepancy in available historic data to determine whether
there is a wavelength dependence as seen in the DIS data.

The gain in the DIS red channel appears to have changed from a previously
measured value.

\section{Introduction}

Observations were made on the night of 6/7 March 2001 to attempt to measure
the absolute throughput of the 3.5m+DIS combination and to look for relative
changes in the system throughput over the past several
years. Observations were made both with SPICAM (photometric
standard fields) and with DIS (spectrophotometric standards,
slitless). Previous measurements have been made with DIS before
and after the last realuminization (\htmladdnormallink{old
report in Postscript}{http://www.astro.princeton.edu/APO/970203.ps})
and some SPICAM photometric zeropoints from 1997 were
posted on a \htmladdnormallink{Web page of Gene Magnier}
{http://www.apo.nmsu.edu/Instruments/APOzeropts.html}; these were the
data used for differential comparison.

The weather appeared to be photometric over the course of the observations
described here. The seeing was also good.

\section{DIS}

Measurements were made of 4 spectrophotometric standards from the list
of Massey et al. Measurements were made both with the 1.5'' slit as well
as slitless; only the latter are described here. For the slitless
measurements, no flat fielding was done.  All stars were placed at
roughly the same location along the slit (had the slit been in place),
so flat fielding corrections would be very minor.  Note that the slitless
measurements have significant background which includes contributions
from all wavelengths.

System throughput was measured by comparing the observations to the
published absolute fluxes. In order to do this comparison, the gain and
the dispersion must be known. Gain was measured from a pair of flat fields,
giving approximate gains of 0.99 and 1.44 e/DN in the blue and red channels,
respectively. \textit{Note that the red gain of 1.44 e/DN represents a change
over previously measured values. It appears that the gain on the red side
may oscillate between two values, roughly 1.44 and 1.98 e/DN; the cause
of such behavior is unknown!} The dispersion was measured from the
wavelength solution performed using HeNeAr observations.

The measured throughput is shown in Figure \ref{fig:t}, and can be compared
with the throughput measured before and after the last realuminization in
Figure \ref{fig:oldt}; both figures are plotted on approximately the same
scale.  Unfortunately, I don't have the older data in tabular form to make
a direct numerical comparison plot. 

It is clear that the throughput is lower than just after the last 
realuminization, but it is not as low as it was before the last aluminization.
Comparing with the curve taken after the last realuminization, it appears that
the throughput degradation becomes more severe as one goes to shorter
wavelengths; it appears that one might get $\sim$ a 50\% improvement in 
throughput at 4000 \AA, 33\% improvement at 5000 \AA,  20\% improvement at 
7000 \AA, and very little change at 9000\AA.

\begin{latexonly}
\newpage
\begin{figure}[h]
\epsfxsize=3.5in
\epsfbox{t.eps}
\label{fig:t}
\end{figure}
\end{latexonly}
\begin{htmlonly}
\begin{center}
\htmladdimg[width=400]{../t.jpg}
\end{center}
\end{htmlonly}

\begin{latexonly}
\begin{figure}[h]
\epsfxsize=3.5in
\epsfbox{oldt.eps}
\label{fig:oldt}
\end{figure}
\end{latexonly}
\begin{htmlonly}
\begin{center}
\htmladdimg[width=400]{../oldt.jpg}
\end{center}

\end{htmlonly}

\section{SPICAM}

Observations were made of four photometric standard fields from the list
of Landolt (1993). Each field contained 4-8 different standard stars of
a range of colors. The four fields were taken at different airmasses to
allow the measurement of atmospheric extinction.

Flat fielding was done using twilight flats. In general, the quality of
the flat fielding was poor, probably because of the known scattered light
problems of the telescope; these were particularly severe given that the
observations were taken with very significant moonlight. Photometry was
done using aperture photometry using an aperture radius of 5 arcsec with
sky measured in an annulus between 10 and 15 arcsec radii. 

Photometric zeropoints were derived, using the expression:
$$ M = -2.5\log (DN/s) + k X + t \textrm{color} + Z $$
where $M$ is the standard magnitude, $k$ the extinction coefficient, 
$X$ the airmass, $t$ the transformation coefficient, and $Z$ the zeropoint;
note that the zeropoints are defined relative to measured DN, not electrons.

In fact, the gain was measured from some ``dome'' flats to be 3.35+/-0.05
e/DN, identical to the 3.37 claimed on the APO SPICAM Web page, so there
is no evidence for any variation in gain, and these numbers can be directly 
compared with previously determined DN zeropoints.

The following zeropoints were derived:

\begin{tabular}{llllll}
Filter&k&t&Z&color&$\sigma$\\
U & -0.649 &  0.187  & 21.816 & (U-B)   &  0.094 \\
B & -0.252 &  0.037  & 23.954 & (B-V)   &  0.032 \\
V & -0.144 &  0.045  & 24.525 & (B-V)   &  0.019 \\
R & -0.060 &  0.007  & 24.551 & (V-R)   &  0.043 \\
I & -0.152 &  -0.221 & 24.441 & (V-I)   &  0.138
\end{tabular}

Note the relatively substantial scatter ($\sigma$), arising most likely
from the poor flat fielding and scattered light problems. As a result, the
accuracy of the derived zeropoints is not spectacular, with possible errors
of 0.1 mag possible in some filters; still the measurements are useful to 
look for large changes in throughput.

These numbers can be compared with some results from 1997 by Ted Wyder
(fall 1997) and Dan Zucker (10/19/97) from Gene Magnier's Web page;
the claimed accuracies are about 3\% (but note the two sets don't always
agree to this level, notably U and I!). Some confusion arises about the
form of the transformation used; at the top of the page, Gene gives a
transformation formula where the airmass correction is referenced to
an airmass of one instead of zero as above (i.e. $M=-2.5\log (DN/s) + k
(X-1) + t \textrm{color} + Z$).  However, inspection of Ted Wyder's plot
suggests that the definition used above, and \textbf{not} the definition
on the Web page, were actually used, and Dan's numbers appear roughly
consistent with Ted's.

\begin{tabular}{llll}
Filter&Z(2001)&Z(1997A)&Z(1997B)\\
U & 21.816 & 22.10 & 22.30 \\
B & 23.954 & 24.43 & 24.39 \\
V & 24.525 & 24.91 & 24.88 \\
R & 24.551 & - & - \\
I & 24.441 & 24.53 & 24.84 \\
\end{tabular}

These numbers clearly show throughput degradation, although the exact amount
is hard to determine given the (apparently) poor flat-fielding/photometry.
The following shows the throughput degradation (in magnitudes) relative to
the two previous data sets.

\begin{tabular}{lll}
Filter&$\Delta$ Z&$\Delta$ Z\\
U&-0.28&-0.48\\
B&-0.48&-0.44\\
V&-0.39&-0.36\\
R&-&-\\
I&-0.09&-0.40\\
\end{tabular}

A degration of 0.4 mag corresponds to 70\% relative throughput in 2001 
relative to 1997.

There is a possibility that the throughput degradation is worse at smaller 
wavelengths, depending on which of the 1997 measurements at U and I are
more accurate.  Better measurements or analysis of both current and past
data would be required to know this for sure.



\end{document}
