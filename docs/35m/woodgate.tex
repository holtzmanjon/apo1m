\documentclass{article}

\usepackage{html}

\begin{document}

\begin{center}
\textbf{Rapid Infrared and Visible Multi-Object Spectrograph (RIVMOS)}
\end{center}

\begin{htmlonly}
Complete document in a single file
(\htmladdnormallink{PostScript}{../rivmos.ps.gz})
(\htmladdnormallink{PDF}{../rivmos.pdf})
\end{htmlonly}

\section{Overview}

The infrared (IR) instrumentation capabilities of the ARC 3.5m are somewhat
limited with the existing GRIM 2 instrument. In particular, this instrument
has a 256x256 detector which has very high noise. This gives a small field
of view, and makes it difficult to get observations of fainter objects in
narrow-band imaging or spectroscopic mode. In addition, the spectroscopic
modes either have very low resolution or have rather small slit widths at
a bit higher resolution. As a result, we have considered alternative options
for improving the IR instrumentation.

Goddard (Moseley, Kutynev, Woodgate, et al.) is interested in building an IR instrument for
use at the 3.5m. They have funding from NGST for a prototype instrument 
to demonstrate the capabilities of a microshutter array, to
be potentially used for fully configurable multi-object spectroscopy. The
proposed instrument has both imaging and spectroscopic capability. They would
like to use the instrument at the 3.5m, and in exchange, make it available
to the ARC community. Scientifically, one of their motiviations for the
instrument is to allow rapid followup, e.g., of gamma ray bursts. As
a result, they would eventually like to see this instrument installed on
a dedicated port (with associated rotator and guider) on the 3.5m. 

Note that Bruce Woodgate has already built one instrument, a Fabry-Perot, which has
been used at the 3.5m by himself and several others.

The following information is taken directly from some of Bruce's presentations
on the instrument, called RIVMOS.

\section{Requirements}

\begin{itemize}
\item Spectral range 0.6-5 microns
\item Use 1024x1024 InSb Aladdin I array detector
\item Field of view: 6x6 arcmin at focal plane of APO 3.5m f10.3 telescope
\item Use 512x512 microshutter array in cooled focal plane (may start with
smaller array)
\item Rapid change from imaging to selected spectroscopic mode ($<$30 sec)
\item Modes: 
  \begin{itemize}
  \item spectroscopic modes, selecting by grism wheel ($\sim$ 6 positions)
  near pupil
   \begin{itemize}
    \item R$\sim$ 50 with prism over 0.6-5 microns
    \item R$\sim$ 2000 echelles with prism cross dispersers, 0.6-2.4 microns,
      2.4-5 microns
    \item R$\sim$ 5000 echelles with prism cross dispersers, 1.2-2.4 microns,
      2.4-5 microns
   \end{itemize}
 \item Imaging mode with $\sim$ 12 filter positions on filter wheel
 \end{itemize}
\end{itemize}

\section{RIVMOS Objectives}

\begin{itemize}
\item Technical goal (for which funding was obtained):
  \begin{itemize}
   \item Demonstrate scientific operations of GSFC microshutter array for
    NGST NIRMOS
  \end{itemize}

\item Science goals
  \begin{itemize}
   \item General use infrared imaging and multi-object spectroscopy, e.g.
   \begin{itemize}
    \item Galaxy cluster redshifts, star formation rates, spectroscopy
    \item Emission line nebulae
    \item Circumstellar spectra
    \item Any science of interest to APO members
   \end{itemize}
   \item Gamma Ray Burst afterglow spectra for redshifts, etc:
    \begin{itemize}
     \item within $\sim$ 5 minutes of alert with position to $<$3arcmin
     \item supporting SWIFT and HETE2 missions
    \end{itemize}
   \item SN Type1A spectra across maximum
    \begin{itemize}
     \item supporting accelerating universe measurements
     \item IR spectra to check for systematic effects in absolute brightness
     \item Preparation for SNAP mission
    \end{itemize}
  \end{itemize} 
\end{itemize}

\section{RIVMOS Schedule}

(From discussion on 2/15/01 between Ed Turner, Bruce Gillespie, Harvey
Moseley, Alex Kutyrev, Bruce Woodgate, and subsequent engineering discussions):

Funding and performance schedule drives 2 phases:

\begin{itemize}
 \item Phase 1:
   \begin{itemize}
    \item Demonstrate microshutter
    \item Begin non-rapid observations, probably in campaign mode, because
      of expense and complications of using liquid helium as a cryogen
    \item 0.6-2.4 microns, 6 arcmin FOV for imaging, smaller for spectroscopy
    \item Nasmyth mount
   \end{itemize}
 \item Phase 2:
   \begin{itemize}
    \item Include rapid spectroscopy
    \item 2.4-5 microns, 6 armin FOV for imaging and spectroscopy
    \item Mirror cell mount, including rotator and guider
    \item Full remote user interface
   \end{itemize}
  
\end{itemize}

\end{document}
