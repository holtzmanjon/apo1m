\documentclass{article}

\usepackage{html}

\begin{document}

\begin{Huge}
\begin{center}
BTC (Big Throughput Camera) at the APO 3.5m
\end{center}
\end{Huge}

\section{Overview}

  We are considering the possibility of making the Big Throughput
Camera (BTC), built by Tony Tyson and previously available at the CTIO
4m (e.g., see \htmladdnormallink{http://www.astro.lsa.umich.edu/btc/btc.html}
{http://www.astro.lsa.umich.edu/btc/btc.html}), operational at the ARC 3.5m.

  The motivation for this would be to increase the field-of-view for
optical imaging at the 3.5m, which is currently limited to $\sim$
290 arcsec on a side with SPICAM. For the long-term, we expect wide field
imaging to be best provided by the large mosaic imager being constructed
by Chris Stubbs et al. at the University of Washington, which will have a
larger field-of-view than the BTC with better corrected images.  However,
the timescale for the delivery of this complex instrument is still not
totally clear. As a result, we consider that making the BTC available on
the 3.5m may be worthwhile \textit{if} it can be done relatively quickly
and a low expense, and \textit{if} there is interest among the ARC user
community for using such an instrument.

The BTC is currently not being used at any other telescope. In exchange
for providing it for use at the 3.5m, it is likely that some telescope
time would be provided to Tyson and/or collaborators.

We solicit expressions of interest from the ARC user community, and
in particular ask you to let us know if you would find this
instrument useful for your scientific programs at the 3.5-m in the relatively
near future. If we proceed to implement this, we would hope to do so by
sometime in fall 2001.

One clear issue with this and any imager on the 3.5m is the problem we
experience with scattered light, which makes accurate quantitative photometric 
measurements difficult. We anticipate that the demand for imaging, including
wide field imaging, might be increased if this problem were solved. At the
current time, we have a contract with a company that is investigating the
problem, and we hope to have a baffling plan to implement during the summer
shutdown; until this study and design are complete, however, we can't
estimate whether any scattered light problem is likely to be present in
the future.

\section{General Instrument Parameters}

The BTC is a 4 CCD mosaic camera, with each CCD being a thinned, backside
illuminated 2048x2048 device with 24 micron pixels. The four CCDs are 
\textit{not} butted against each other; there is a roughly one inch
gap between each of the CCDs, which are roughly 2 inches square.
See \htmladdnormallink{http://www.astro.lsa.umich.edu/btc/tech.html}
{http://www.astro.lsa.umich.edu/btc/tech.html} for details.

At the ARC 3.5m f/10 focus, the plate scale (5.87 arcsec/mm), would
provide a field-of-view of $\sim$ 290 arcsec square on \textit{each}
CCD, with a total field-of-view of $\sim$ 730 arcsec square, but this
\textit{includes} the dead space between the detectors (roughly 150 arcsec),
so multiple images would be required to image such a field contiguously.

\section{Performance Issues}

\subsection{Focus variations across field of view}

The focal plane of the 3.5m is curved so one cannot get perfectly
uniform focus in a flat focal plane. However, in the BTC, each CCD is
individually adjustable in piston and tilt, so each CCD can be put 
in the tangent plane at the center of its field-of-view. Hence, the
focus variations across the field of each detector can be kept 
small, comparable to that existing currently in SPICAM (which has the
same size detector).

The field curvature for a Ritchey-Chretien telescope can be derived
(e.g., Schroeder ...). From this, we find a radius of curvature of
1.4m for the 3.5m. The size of each CCD is 0.05m. Hence, the deviation
of the edge of the CCD from the focal plane is about 0.22mm if the
CCD is adjusted to be tangent to the focal plane at the center of the
field. This corresponds to an image blur of 0.022mm, or 0.13 arcsec,
confirming that this should not be a major contributor to image quality.

%    CCDs can be individually adjusted in piston and tilt.
%    a) Image blur from focal plane curvature:
%       Radius of curvature, R = 1.4m
%       CCD size is 0.05 m
%       ==> deviation of edge of CCD from focal plane = 0.22 mm
%	   and blur is about 0.022 mm, or 0.13 arcsec.

\subsection{Off-axis astigmatism}

In a Ritchey-Chretien, astigmatism increases as one moves off the optical
axis. Again, the amplitude can be derived given the telescope parameters.
We find that at the farthest corner of each CCD, which is roughly 530
arcsec off-axis, we get an angular astigmatism (AAS) of 0.29 arcsec, or
a total image blur of 0.58 arcsec. At the center of each CCD, however,
we get AAS=0.1 arcsec, or a total image blur of 0.2 arcsec.

Hence, some variation in image quality will probably be noticeable
across the field-of-view, but it should not be terrible under normal
conditions.

Of course, one could build an astigmatism corrector to reduce this effect.
Construction of such a corrector would probably be simplified because
the field is not contiguous, so a separate corrector could probably
be placed in front of each CCD, and such correctors might be relatively
simple, correcting only for the mean of the astigmatism across the
chip. However, initially we propose \textit{not} to build any corrector,
especially considering the development of Chris Stubbs' large mosaic
camera, which will have a corrector and supercede the BTC when it
becomes available.

\subsection{Waviness of the CCDs and field flatteners}

Each CCD in the BTC is not perfectly flat, which would result in
additional focus variations across the field in addition to
those discussed above. However, field flatteners are incorporated into
each camera already to correct for this effect, so it is not likely
to cause a significant problem.

At the current time, however, we need to confirm whether these field
flatteners include correction for focal plane curvature of some other
telescope. If so, then the results above would need to be modified.


%    b) Astigmatism: (see Table 6.7 in Schroeder's Astronomical Optics)
%       using m = 10.3/1.8 = 5.722, 
%             beta = 0.4304, (using kappa = 0.21287)
%             furthest corner of CCD from axis = 90 mm = 528 arcsec, 
%             center of each CCD from axis = 54 mm = 317 arcsec,
%       we get
%       ==> AAS = 0.212*(theta^2) = 0.29 arcsec at the furthest corner
%       ==> AAS =                 = 0.10 arcsec at the center of each CCD

%    Questions:
%    - are the corrector lenses in front of CCDs just for correcting
%      wiggle of chips?

\subsection{Filters}

Filters for the BTC are placed inside a slide which can accomodate 4
different filters.  Each filter is 5.8 inches and 10 mm thick. Tony has a set 
including $B_J$,V,R,$I_{tyson}$.  The slide also has a 45 deg mirror for guider 
camera pickoff, for what purpose we're not sure.  The slide is 2 inches thick.

We need to understand the need for the mirror, and also how difficult it
would be to change filters in the slide, to find out whether multiple
slides exist and whether it is possible to rapidly switch between slides,
and also investigate the cost of additional filters.

\section{Mechanical Interface}       

We would need to construct an adaptor to allow the BTC to be mounted
on the 3.5m. Because of the need for a rotator and guider, we would
mount the BTC at the Nasmyth port.

The critical issue here is whether the BTC focal plane can be made 
to coincide with the 3.5m focal plane. For this we still need the
detailed mechanical specifications of the BTC. Qualitative description
of these suggests that there will not be a problem, but this must
be confirmed. 

%2.  Hardware
%    a) Adapter:
%       Need to make detailed specs of adapter between camera and telescope;
%       Questions:
%       - Need to know how much space there will be in front of CCD.  Is
%         it 2 inches, or more? 

\section{Electronic interface}

The camera/electronics combination mounts on the telescope.
An optical fiber and one RS232 line connect the camera/electronics
to the electronics rack which can be as far as 500 ft away.  A Sun
workstation controls the camera functions, and can sit up to 100 ft
from the rack. The camera can be fully controlled remotely by logging
into this computer, so it does not need to be physically accessible
to observers. Hence it should be straighforward to locate the
electronics rack and controlling computer on the mid-level of the 3.5m.

\section{Software interface}

A complete user interface has been developed for the BTC by Gary
Bernstein, which was used by observers at the CTIO 4m. More details
can be seen at \htmladdnormallink{http://www.astro.lsa.umich.edu/btc/btc.html}
{http://www.astro.lsa.umich.edu/btc/btc.html}, but this  should be a
mature interface. No attempt would be made to control the BTC through
Remark; in this respect, the BTC would be operated similarly to SPICAM,
although the BTC interface is graphical.

We would need to develop some additional software to allow the BTC
software to communicate with the telescope to get pointing information
at least, and possibly to allow for offsetting the telescope from within
the BTC interface. At this point, we do not know how much work this
would entail, but suspect that it would not be a major problem.

\section{Additional information}

For additional information on the BTC, see
\htmladdnormallink{http://www.astro.lsa.umich.edu/btc/btc.html}
       {http://www.astro.lsa.umich.edu/btc/btc.html}
\htmladdnormallink{http://www.astro.lsa.umich.edu/btc/tech.html}
       {http://www.astro.lsa.umich.edu/btc/tech.html}

\end{document}
