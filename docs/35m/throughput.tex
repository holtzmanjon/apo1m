\documentclass{article}

\usepackage{html}

\begin{document}

\begin{center}
\textbf{Throughput Monitoring for the ARC 3.5m}
\end{center}
\begin{center}
Jon Holtzman, NMSU, 3/2001
\end{center}

\begin{htmlonly}
Complete document in a single file
(\htmladdnormallink{PostScript}{../throughput.ps.gz})
(\htmladdnormallink{PDF}{../throughput.pdf})
\end{htmlonly}

\section{Introduction}

For all instruments, monitoring throughput requires photometric conditions.
It is not worthwhile to perform throughput monitoring in the obvious
presence of cirrus. If a few clouds are around, one might consider a single
point throughput measurement if the cloud camera suggests it is clear
in the location of the object.

A throughput measurement accurate to several (5-10??) percent can be made
by assuming a typical extinction curve for the site; in this case, only a
single object is required.  The closer the observed star is to the zenith,
the less effect the uncertainty in the adopted extinction will have, so
observing at as low an airmass as possible is desirable. Also, the
use of a mean extinction curve will lead to larger uncertainties at
shorter wavelengths.

A more accurate measurement requires measuring the extinction on the
night, which involves making several observations of objects covering
a range of airmasses. If time permits, the latter should be done, but
even a single object near the zenith will probably prove to be useful.

\section{Monitoring with DIS}

Monitoring throughput with DIS involves slitless spectroscopy of a 
spectrophotometric standard. 

\subsection{Observing procedure}

\begin{itemize}

\item DIS should be mounted with a slit wheel with a standard slit in
one position, and a clear position in the other. The low resolution grating
should be in place. The wavelength settings should be 7700/4400 in red/blue.

\item You should use file names of the form yymmdd with the UT date,
e.g., the night of March 6/7 2001 should use 010307 for the root file name.
Keep a log of the exposures with their associated exposure numbers.

\item Take 5 bias frames to be used to check the readout noise.

\item Take 5 flat fields using the bright quartz lamps, mirror covers
closed, exposure time 12 seconds

\item Take a 60s He exposure, 30s Ne, and 90s Ar exposure

\item Slew to one of the following flux standards using the listed rotation.
I have a  REMARK coordinate file (\htmladdnormallink{throughput.rmk}{../throughput.rmk}) which has these coordinates already entered. \textbf{Please note the
requested rotation!}

\begin{tabular}{llllll}
Name&$\alpha_{1950}$&$\delta_{1950}$&Rotation&Exptime w/slit& Exptime slitless\\
PG0216+032&02:16:43&03:13:08&0&180?&60?\\
G191B2B&05:01:31.5&52:45:52&20&60&30\\
Feige 34&10:36:41.2&43:21:50&0&30&15\\
Feige 66&12:34:54.7&25:20:31&0&15&15\\
Feige 67&12:39:18.9&17:47:24&0&60&30\\
PG1708+602&17:08:35.9&60:13:52&90&180?&60?\\
BD+28 4211&21:48:57.1&28:37:48&0&15&15\\
Feige 110&23:17:23.5&-05:26:22&0&60&30
\end{tabular}

\item Put the object on the slit using the slitviewer. Take a DIS exposure
of the specified time. Confirm exposures aren't saturated.

\item Move the slit wheel to the clear position. Take a DIS exposure of the
specified time. Confirm exposures aren't saturated; note that sky will be
\textit{much} brighter with slitless observations (hence the shorter
exposure times generally).

\item Take additional stars as time permits.

\end{itemize}

\subsection{Reduction procedure}
\begin{itemize}
\item Construct flat field
\item Do wavelength calibration
\item Reduce frames and extract spectra 
\item Correct for extinction
\item Get flux curve, ratio of true flux to observed counts
\item Derive throughput. 
$$Counts(DN/pix) = Flux (ergs/cm^2/s/\AA) A T t {\lambda\over hc} D {1\over G}$$
where $A$ is the telescope collecting area, $T$ is the system throughput,
$t$ is the exposure time, $D$ is the spectrograph dispersion, and $G$ is
the gain. The flux curve, $F$ is $Flux/Counts$ So we have:
$$T = {hc \over \lambda} {G\over F A t D}$$

\end{itemize}

\section{Monitoring with SPICAM}

Monitoring throughput with SPICAM involves observations of some photometric
standards; fields are available which have several stars in a single
pointing.

\subsection{Observing procedure}

\begin{itemize}

\item SPICAM should be mounted with a filter wheel which contains the standard
UBVRI filters. It is important that we always use the same set of filters!

\item You should use file names of the form yymmdd with the UT date,
e.g., the night of March 6/7 2001 should use 010307 for the root file name.
Keep a log of the exposures with their associated exposure numbers.

\item Take 5 biases, to be used to check the readout noise.

\item Take 5 flat fields with the V filter using the bright quartz lamps, 
mirror covers closed, exposure time 2 seconds, to be used to check the gain.
If no twilight flats can be obtained, take mirror flats in all filters.

\item If at all possible, obtain twilight flats, with the shortest
wavelengths being taken closest to sunset/sunrise.

\item Slew to one of the following photometric standards fields.
I have a  REMARK coordinate file (\htmladdnormallink{throughput.rmk}{../throughput.rmk}) which has these coordinates already entered.

\begin{tabular}{lllllllll}
Name&$\alpha_{2000}$&$\delta_{2000}$&Rotation&U Exptime&B Exptime& V exptime & R exptime & I exptime\\
PG0231+051&02:33:41.00&+05:18:40.0&0&120&30&10&5&5\\
RU149     &07:24:16.00&-00:32:17.0&0&120&30&10&5&5\\
PG0942-029&09:45:12.00&-03:08:00.0&0&120&30&10&5&5\\
PG1047+003&10:50:06.00&-00:01:00.0&0&120&30&10&5&5\\
PG1323-086&13:25:44.00&-08:49:50.0&0&120&30&10&5&5\\
PG1525-071&15:28:13.00&-07:15:54.0&0&120&30&10&5&5\\
PG1528+062&15:30:45.00&+06:01:11.0&0&120&30&10&5&5\\
PG1633+099&16:35:34.00&+09:46:22.0&0&120&30&10&5&5\\
PG1657+078&16:59:32.00&+07:42:50.0&0&120&30&10&5&5\\
PG2213-006&22:16:22.00&-00:21:49.0&0&120&30&10&5&5\\
PG2331+055&23:33:49.00&+05:46:49.0&0&120&30&10&5&5\\
PG2336+004&23:38:43.00&+00:42:24.0&0&120&30&10&5&5
\end{tabular}

\item Take observations using exposure times 120, 30, 10, 5, and for UBVRI.

\item Do multiple fields, trying to span a range of airmass between
an airmass of 1 and 2.

\end{itemize}

\subsection{Reduction procedure}

\begin{itemize}
\item Construct flat fields for each filter

\item Do stellar photometry on reduced frames

\item Solve for photometric zeropoints and transformation coefficients,
extinction coefficients if multiple airmasses were observe, else using
standard extinction coefficients.
\end{itemize}

\section{Monitoring with NA2 guider}

Monitoring throughput with the NA2 guider is essentially the same as
using SPICAM.

\end{document}
