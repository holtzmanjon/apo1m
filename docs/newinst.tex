\documentclass{report}[12pt]

\usepackage{epsfig}
\usepackage{html}

\textwidth=8in
\textheight=10in

\hoffset=-1.7in
\voffset=-1in

\epsfxsize 7in 

\begin{document}

\begin{LARGE}
\begin{center}
NMSU 1 meter telescope : instrumentation mounted on Nasmyth rotator port
\end{center}
\end{LARGE}

\vskip 0.5in

\noindent \textit{Rough} sketch of system (dimensions in inches)


\begin{latexonly}
\epsfbox{1m.eps}
\end{latexonly}
\begin{htmlonly}
\begin{rawhtml}
<IMG SRC=1m.gif WIDTH=100%>
\end{rawhtml}
\end{htmlonly}

\noindent Notes on the telescope

\begin{itemize}
\item Telescope is a Ritchey-Chretien design and produces an F/6.06 beam.

\item Entire unit of instrumentation mounted on Nasmyth port rotates as
telescope moves around in the sky. Consequently, weight distribution should 
not be extremely unbalanced, although some degree of imbalance is acceptable.

\end{itemize}

\noindent Notes on the elements

\begin{itemize}
\item CCD/shutter

\begin{itemize}
\item Use existing unit.

\item Location of detector (focal plane) is 1.125 inch behind mounting surface

\item Detector size is about 1 inch square (i.e. 0.7 inch maximum radial extent)

\item Unit mounts into mounting surface with small circular protrusion, of
dimensions ... There is a mounting lip around this protrusion, and mounting
fixtures should be designed to allow relatively easy removal and installation
allowing for repeatable and very stable positioning.

\end{itemize}

\item Filter wheel
\begin{itemize}

\item Design goal is 1 inch thickness for approximately 6 inch diameter 
around optical axis. Entire filter wheel will probably be significantly
larger than this, and outside of this 3 inch radius circle, something 
(e.g., a motor) could protrude back towards the focal plane.

\item Back plate must accept protrusion from CCD/shutter assembly. Allow
for possibility in future of new CCD/shutter and, hence, replacing, the
back plate of the filter wheel.

\item Filter wheel itself should allow 6 3x3 inch square filters. Filters
not to be more than 8mm thick. 

\item Alternate filter wheel should be constructed for 2x2 inch square filters,
holding as many as possible. Obviously, centers of 2x2 and 3x3 inch filters
must be the same.

\item Preferred design for loading filters: filters fit into separate cells
and are held in by small recessed screws pressing against rubber pad mounted
against 2 sides of filter. These filter cells would be roughly 0.5 inches
larger per side than filters themselves. Filter cells then mount into
filter wheel and secured in some fashion.

\item Filter wheel positions must be repeatable to approximately 10 microns
(is this reasonable?).

\item Filter wheel motion must be driven by a motor which is computer
controllable by a PC.

\item Filter wheel position must be encoded in a fashion which is computer
readable.

\item Filter wheel must be light-tight, and there must be no direct path
for light to pass through unit except through the filter. Beware of
glancing edges.

\end{itemize}

\item Guide box
\begin{itemize}
\item Total width of guide box is 10.75 inches including back and front
plates. (This assumes design focal plane location. If this is determined
to be in error, width of guide box might change. Consequently, it's best
to have design of internal guider unit as thin as possible.)

\item Entire assembly to be as stiff as possible to minimize flexure when
rotating on rotator.

\item Mount at rotator is 20 inches diameter. For about 4 inches back from
this mounting point, nothing can extend more than approximately 11 inches
radially from center. Further back, there are no radial space constraints.

\item Guide assembly to be mounted on the rear of the box, as close as
possible to the focal plane. Mounting to rear is preferred to minimize
differential flexure between guider and science image planes.

\item Location of guider pickoff mirror as close to on-axis as possible
without vignetting beam to science instrument. My calculation suggests
center of guider pickoff 1-1.25 inches off axis, depending on exactly
how far pickoff is located from the focal plane. Guider pickoff should
be aligned parallel to rows/columns of science detector to allow mirror
to come as close as possible to optical axis without vignetting.

\item Guider assembly needs to be constructed considering how best to
minimize scattered light off the assembly and into the science camera.

\item Spectrasource camera to be used as guider detector is already
purchased and must be incorporated into design. Dimensions of Spectrasource
camera are ...

\item Relative position of guide camera to CCD camera must be extremely
stable ($<$ 10 microns over 90 degree rotation of entire assembly).

\item Spectrasource camera to be mounted into existing focus mechanism,
which is computer controllable and encoded. (A smaller unit may be available
if current unit dominates current design width.)

\item Possible need to include mounting structure between guider pickoff
and guider CCD for off-axis astigmatism corrector.
\end{itemize}

\item Bearing baffle
\begin{itemize}
\item Purpose of baffle is to block light glancing off inside of bearing
surface. Baffle should have opening with radius 2.25 inches to achieve
this. A hole of this size should comfortably allow the full beam from the
science instrument. It may vignette the guider slightly.

\end{itemize}

\item Nasmyth baffle
\begin{itemize}
\item Purpose of baffle is to block light coming from around the tertiary.

\item Baffle location is constrained so as not to fall into main beam to
primary mirror. 

\item It is not possible to construct a baffle which blocks all light from
around tertiary mirror without vignetting beam to science instrument (slightly)
and guider camera (significantly). Choose compromise size which will 
block all light with appropriately constructed tertiary baffle. Estimated
size of baffle opening is 2.75 inches radius, but final size TBD (also
depends on design and construction of new tertiary baffle).
\end{itemize}

\item Tertiary mirror/baffle
\begin{itemize}
\item Tertiary mirror center is 40.875 inches from focal plane (design
distance).

\item Tertiary mirror baffle should be constructed to obstruct any light
from around tertiary to have direct path to science focal plane.
Specifications ...

\item Tertiary mirror baffle must have obscuration above the mirror to
prevent tertiary from seeing around the secondary mirror. Specifications
...

\end{itemize}


\end{itemize}
\end{document}
