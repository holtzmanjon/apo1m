\documentclass[10pt]{report}

\usepackage{makeidx}
\usepackage{epsf}
\usepackage{html}

\includeonly{%
chap1,%
chap2,%
chap3,%
chap4,%
chap5,%
chap6,%
chap7,%
chap8,%
chap9,%
refs,%
index,%
}

\makeindex

%\renewcommand{\[}{\begin{eqnarray}}
%\renewcommand{\]}{\end{eqnarray}}
%
%\renewcommand{\tilde}{\~\space}
%
%\renewcommand{\floatpagefraction}{0}
%\setcounter{topnumber}{1}
%\setcounter{totalnumber}{1}

\newenvironment{hanging}{
	\begin{list}{}{
		\labelsep=0pt
		\labelwidth=0pt
		\listparindent=0pt
		\itemindent=-\leftmargini
		\leftmargin=\leftmargini
	}
}{
	\end{list}
}
%\newenvironment{references}{
%	\begin{hanging}\raggedright
%}{
%	\end{hanging}
%}
%\newenvironment{url}{
%	\begin{list}{}{
%		\labelsep=0pt
%		\labelwidth=0pt
%		\listparindent=0pt
%		\leftmargin=\leftmargini
%	}\item\tt
%}{
%	\end{list}
%}
%\newenvironment{code}{
%	\begin{list}{}{
%		\labelsep=0pt
%		\labelwidth=0pt
%		\listparindent=0pt
%		\leftmargin=\leftmargini
%	}\item\tt
%}{
%	\end{list}
%}
\newenvironment{example}{
        \begin{list}{}{
                \labelsep=0.25in
                \labelwidth=2.0in
                \leftmargin=2.75in
                \itemindent=0in
                \itemsep=0in
                \parsep=0in
        }
}{
        \end{list}
}

\newcommand{\note}[1]{\marginpar{\raggedright\scriptsize{#1}}}

%\makeindex

\title{\Huge\bf NMSU 1m User's Manual}
\author{\Large
Jon Holtzman
}
\date{\Large November 2008}

\begin{document}

\setcounter{page}{1}
\pagenumbering{roman}

\maketitle

\begin{htmlonly}
Complete document in a
\htmladdnormallink{single PostScript file}{../man.ps.gz}.
\end{htmlonly}


\tableofcontents
%\listoffigures
%\listoftables

\clearpage

\setcounter{page}{1}
\pagenumbering{arabic}


\chapter{Introduction}

The NMSU 1m is an F/6 alt-az Ritchey-Chretien telescope located at the Apache
Point Observatory. It was constructed by the AutoScope corporation, although
the company went out of business before they really completed the project.

The 1m dome is the small dome located closest to the 3.5m dome. We currently
have a rotating tertiary with instruments at the two Nasmyth ports: NA1 has
a 2048x2048 LN2 cooled CCD camera on a instrument derotator, while NA2 has
a single-object multi-color photometer that uses a combination of PMTs and 
APDs as detectors, along with a aperture-viewing camera; the photometer is
not on a rotator.

The 1m is normally run in robotic or remote mode. This manual describes the
interactive command set used for interactive observing, and provides some
notes on scripts used for robotic observing.

\section{Instruments}

As of 2008, there is a rotating tertiary that can be directed on either of the
Nasmyth ports.

\subsection{NA1}

There is an imaging camera on an instrument derotator at the NA1 port. This
camera uses a E2V 2048x2048 CCD that is cooled in a LN2 dewar. The electronics
were supplied by Bob Leach of Astronomical Research Cameras. The pixels are
13.5 microns on a side, giving a pixel scale of approximately 0.46 arcsec/pixel.

The camera is mounted behind a guider box and filter wheel. The filter wheel
holds 10 2x2 inch filters; UBVRI, ugriz, Washington CM, DDO51, and some 
narrow band filters belonging to Rene Walterbos are available.

The guide box has a Finger Lakes Instrumentation CM2-1 camera with an E2V
1024x1024 CCD in it. The guide camera is located on a radial stage and is
pointed at a diagonal pickoff mirror that is itself located on a stage on
top of the radial stage to allow independent focussing of the guider.

\subsection{NA2}

At the NA2 port, a single-object, multi-channel (UBVRI) photometer has been 
built at NMSU.  This uses photomultiplier tubes for UB and avalanche photo-diodes
for VRI. Light from the telescope hits a diagonal mirror with a small hole
in it that allows light into the photometer. The rest of the field is imaged
using an APOGEE 7P camera which has a 512x512 SITe CCD; the pixel scale is
about 1.6 arcsec per pixel. There are no filters in front of this camera.
It is used for object acquisition and subsequent guiding (on a field star).

\section{Computers}

In the 1m control room at APO, there is currently one computer:
command1m, a Dell 3.2 Ghz Pentium D PC, which runs the Linux operating
system.  command1m is the main computer used for operating the 1m.
A variety of software, including IRAF, xvista, TeX/LaTeX, etc. is
available.  On command1m, users should use the tcomm account; for
the current password, check with Jon Holtzman.  The normal mode of
observing is that a VNC server is run on command1m , and all of the
observing programs are run out of this server. See section
\ref{sect:software} for instructions on how to start/kill the VNC
server and how to connect to it.

Another computer, ccd1m, is located in the computer room in the main
building. This computer controls the NA1 camera via a custom card and fiber
optic connection, the guide camera via a USB port (through a fiber extender
to the dome), and 3 remote video Webcams (through USB ports).

\subsection{Dome computers}

In the dome, there is a rack which holds the telescope control
hardware. The left side of rack holds the main power switches for
the telescope/dome, the telescope motor controllers, the guider/tertiary
motor controllers, and the weather station.  The right rack holds
two PCs which control the telescope/guider and the APOGEE photometer
acquisition camera. The two PCs are:

\begin{itemize}
\item tocc1m: PC which handles telescope, dome, and guider control, which 
is called the TOCC computer.  This runs DOS.

\item eyeball: PC for the photometer aperture-viewing CCD. This runs Linux.
When the photometer is not being used, this can be powered down to 
minimize heat production in the rack.

\end{itemize}

There is a single monitor and keyboard in the rack that can be used with
either computer, but the cables need to be plugged into the computer
you want to look at.

\section{Emergency telescope control}

There are two emergency stop buttons in the dome: one on the pier across
from the door, and another small which controls the power strip at the top 
of the left side of the power rack.
\textbf{Please identify the locations of these buttons before using
the telescope.}
Hitting this button will kill power to the telescope and dome motors and
should immediately stop the telescope and dome.  This should
be done if there is ever any question about whether the telescope is
moving safely, or is about to run into something or someone.
If the situation requires that you hit the emergency
stop button, \textit{you must contact Jon before restoring power and
using the telescope}; after hitting the stop button, the telescope will
be lost and not be aware of the fact that it is lost, which can be a
very dangerous situation. Hopefully, you will never need to hit this
button.

Emergency stop is also available through computer control using the
power program which is automatically started when the telescope control
suite (tcomm) is started.

If you kill the motor power with the dome open, the dome will automatically 
close (so long as there is site power). Note that the software will not be 
aware that the dome has closed if this
happens.

%Note the location of two red {\bf panic} buttons in the dome. 
%One is
%located on the telescope structure directly in front of you as you enter
%the dome, and the other is located on the left power rack. Hitting either
%of these buttons will immediately cut power to the telescope. This should
%be done if there is ever any question about whether the telescope is
%moving safely, or is about to run into something or someone.

\section{WARNING about manual telescope motion}

Although it is possible to move the telescope manually, this should
\textbf{not} be done if it can be avoided. Always move the telescope using
computer control, unless instructed otherwise. The reason
for this is that it is possible with an alt-az telescope to have cable
wrap-up problems, which can be avoided only if the software accurately
can keep track of where the telescope is. If the telescope is moved
manually, the software won't know about it, and it may then command
moves which will cause trouble. This is especially true for the rotator,
which has no limit switches at all, and consequently its state of wrap
is monitored by software only.

\chapter{Starting up}

You should get the telescope started and initialized well before it gets
dark, to make sure everything is working well. Everything can be initialized
without opening the dome. 

\section{Power to the telescope}

All power to the telescope components is done using software control
through several network controllable power strips.  These devices
allows network control of 24 outlets in the dome, into which various
components are plugged in. Power switches in the dome should not
be used except in the case of an emergency.

You can control the network power strips using the command line program
\textit{power1m on|off device}; \textit{power1m} by itself gives the
list of device names. You can also control the power using the graphical
power program, just click on the item you want to power toggle, and it
will ask for a confirmation. As mentioned below, don't power the telescope
computer (tocc1m) off without quitting the software first if at all 
possible!

\section{Starting up the software}

\label{sect:software}

Everything for the 1m can be done from inside the control room (or remotely).
The telescope is run from the Linux PC, command1m. 
The normal mode of observing is that a VNC server is run on command1m , and
all of the observing programs are run out of this server.  If you should need
to restart the server, e.g. after a computer reboot, enter \textit{vncserver} (note CentOS 6 requires -depth 32);
if you need to kill a running server, e.g. if the window manager fails,
enter \textit{vncserver -kill :1}.

Users on any machine with a VNC client can connect to the server on command1m, 
and thus control the telescope. On-site, simply enter: \textit{vncserver command1m:1}. If you are off-site, you must channel the
VNC viewer through an SSH channel to be able to access the VNC server:
enter \textit{ssh -L 5909:command1m.apo.nmsu.edu:5901 tcomm@command1m.apo.nmsu.edu} to open a channeled SSH session, then from another window, enter
\textit{vncviewer :9}.

Ideally, the set of telecope programs will already be running in the VNC
window. Under some conditions, however, you may need to know how to restart them.

You can start the suite of telescope control programs by entering
\begin{quote} tcomm \end{quote} in an xterm window; usually, we use
an xterm in the bottom left of the screen.
You can create new xterms using the icon on the control bar
at the bottom of the display.
When you start the telescope control programs, a variety of windows will
be opened.

The tcomm script  will actually start several different programs:

\begin{itemize}
\item command : main command window from which ALL commands should
              be entered.
\item telescope control (port) : lower right, used to communicate with TOCC
              computer. This progam emulates the TOCC computer display and
              you can directly type in this program with results which are
              the same as if you were typing in the dome. Can be started with
              \textit{start\_port} from home directory
\item CCD control (lccd) : talks with the ccd1m computer to control the science CCD. Can be started with \textit{start\_lccd} from home directory
\item FLI control (fccd) : talks with the ccd1m computer to control the guider CCD. Can be started with \textit{start\_fccd} from home directory
\item APOGEE control (accd) : talks with the eyeball computer to control the guider CCD. Can be started with \textit{start\_accd} from home directoy
\item status : upper right, reports telescope and CCD status information to user. Can be started with \textit{start\_status}
\item power : upper panel of screen, shows power status and can be used to
emergency stop the telescope, and toggle individual powers (which should not
be necessary, as these can also be controlled through the command window).
Can be started independently using command \textit{power}
\item rvideo : remote video, for use during telescope initialization to watch
the telescope. Can be started with \textit{start\_rvideo}
\item alive (runs in background): network aliveness program, sends a signal 
every 30 seconds from
local computer to APO; if telescope control program doesn't see this signal
for 5 consecutive minutes, the dome is closed and tracking turned off
\item shutter35m: monitors status of 3.5m shutter, temperature in rack, etc.
\end{itemize}

In the future, the port, ccd, and spec programs may come up in an iconified
state, as they are really only necessary for engineering work - you should
not need to ever type anything in these windows.

When you start tcomm, the command program should automatically check to 
see whether various devices (telescope control program, telescope and dome
motors, science CCD, and guider CCD) are powered up and responding. If
they are, the program will proceed without requesting any input. If not,
the program will ask if you wish to power various devices up. 

Once the powers are checked, you will be prompted to initialize the telescope
and dome if they have not already been initialized - see below for more details on
this. After this, there will be question about which port you are observing
on (NA1 or NA2), which sets the camera to which commands will be directed (this
can be changed later during a command session as well). After the startup
questions, you will get a Command: prompt and the program is
ready to accept normal commands.

For reasons having to do with the automatic filling of the LN2 dewar for
the NA1 camera, the command program should be left running at all times.
For completeness, the command to gracefully quit the program is \textit{QU}.

If there is a commanding problem, i.e. program lost or hung or in infinite
waiting loop, you should be able to kill the programs by hitting a 
CTRL-backslash
in the command window. If any of the other windows fail to disappear with
this, they can be killed individually.

%To quit from the program, you enter the QU command. When this is entered,
%the program will ask you if you wish to quit the telescope control program
%running in the dome. In general, we leave the telescope control program
%running.  It is very important, however, that you do not kill the power to the
%telescope control program in the dome without exiting it gracefully.
%Note that the telescope control program which runs in the dome 
%needs to be exited gracefully
%(i.e., commanded to exit) because the program writes out the current
%rotator position which is essential to record to prevent cable wrapups.
%Because of this, make sure NOT to turn off the computer power via the
%network power switch without gracefully exiting the telescope control
%software first!

\section{Remote video cameras}

There is a single Panasonic network web camera located on the north side 
of the dome, with Internet address video1m.apo.nmsu.edu; however, this 
address can only be seen from within the APO network, so is normally viewed
through the VNC interface. The camera can be commanded to move in pan and
tilt and so can see a relatively large fraction of the dome.

%There are three webcams in the 1m dome: one located along the ``tube'' of 
%telescope, one looking at the NA1 port, and another mounted on the azimuth
%disk looking at the inside of the dome in the direction that the telescope
%is pointing: all three rotate along with the telescope.
%
%All three cameras are controlled by a single program (rvideo, running on 
%ccd1m) that can be started with the \textit{start\_rvideo} command. This
%opens a video window that shows the image from one camera. You can switch
%between cameras by clicking on the buttons in the lower right of the
%video window. Hitting 's' in the video window will take a snapshot, hitting
%'u' will toggle update mode (continuous shots) on and off. 'r' redisplays
%the current image. 
%
You can turn on lights in the dome using the graphical power program (Webcam
lights) or on a command line with \textit{power1m on lights}. Be aware that
turning on the lights with the dome or louvers open can impact observations
at the 3.5m and/or the 2.5m!  The lights are automatically turned off once per 
hour.

\section{Initializing the telescope}

When the computers are powered up, the telescope and CCD programs go
through some initialization commands. For the CCDs, no user input is 
required - you will see a series of commands come up automatically in
the CCD control windows. For the telescope, some interaction is required
which needs user input either from the command window
or from the TOCC computer in the dome. During initialization it is important
to watch the rotator to make sure that cables do not wrap - as a result,
initialization should either be done in the dome, or, if operating remotely,
using a video camera view of the rotator.

After the TOCC computer is booted up, you will see on the screen (both
in the dome and in the telescope control window) the following question:

\begin{quote}
Do you wish to initialize the telescope with a :\\
\ I: normal init (uses stored coords to find home positions - will be very\\
\ \    slow if stored coords are wrong)\\
\ F: full init - finds home positions without any previous knowledge (but note,
telescope must be CCW of azimuth limit, and rotator position must have been
previously stored correctly!\\
\ M: manual init - prompts for approximate current telescope position, then 
finds home position via full init. Use if reported rotator position does
not match observed position\\
\ Q: very quick init (stored coords - assumes telescope hasn't been moved!)\\
\  S: skip initialization\\
Enter your choice: \\
\end{quote}

If the telescope has not been moved by hand, you should be able to do a normal
init (I): this will use the saved positions to move to the home switches, and
then reinitialize on the home switches. If the telescope has been moved
manually, you will need to do a full init (F), which will move the telescope
in azimuth clockwise until it hits a limit switch, and in altitude upward 
until it hits a limits switch, then uses these positions to go back and
find the home switches. NOTE, however, that there is no limit switch for the
rotator, so if the rotator is lost, doing a full init can lead to cable wrap
and DAMAGE, so this must be done with extreme caution, and looking with the
remote video at the rotator.
If it appears that the cables leading to the rotator are wrapping, 
hit the emergency stop button (on upper part of left panel in dome), or,
if running remotely, by clicking the mouse in the red Emergency Stop section 
of the power program at the top of the screen. If this happens, you should
check with Jon before restarting anything.

You will not be able to move the telescope until it is initialized.

Next the program will ask:

\begin{quote}
Do you wish to initialize the dome (Y or N)?
\end{quote}

Enter Y. The dome will spin around until it finds its home sensor. Very
rarely, this fails; if it fails, the countdown on the screen will expire
without the home position being found. If it fails, this is not a disaster 
as you can command a dome initialization from the command prompt later on
with the DI command - you will need to remember to do this!

Next the program will ask:

\begin{quote}
Do you wish to slave the dome (Y or N)?
\end{quote}

Enter Y. When the dome is slaved, it should automatically rotate so that
the slit is oriented in front of wherever the telescope is pointed.

Finally, it will ask you which camera/Nasmyth port you will be using.
This can be changed later with the LEACH/APOGEE and INST commands.

A list of available commands can be viewed at any
time using the HP (Help) command.

\section{Weather and telescope safety}

You are fully responsible for the safety of the telescope in terms of
its exposure to weather. You should \textbf{not} open the dome or the
louvers if there is any chance that weather will damage any of the contents
in the dome.  It is important that you must consider not only clouds 
and the possibility of rain, but also the humidity and the dust conditions.

Our normal mode of operation is to automatically follow the judgement of
the 3.5m telescope. The 3.5m broadcasts its dome status every minute or
so. If the 1m receives a "3.5m is closed" message, or if it does not receive
a "3.5m is open" message, the dome should automatically close and the
telescope should stow. Clearly, it is good to confirm that this actually
happens if possible!

It is possible, but \textbf{highly discouraged} to turn off the slaving
to the 3.5m dome status; this might be done, for example, to allow 
the telescope to be moved during the day without continually stowing
itself. Slaving is turned off using the \textit{-35m} command, and can
be turned back on using \textit{+35m}. If you turn slaving off, you 
will need to enter \textit{clear} to clear the shutdown flags before
you will be able to move, etc.

You can monitor the APO 10-micron all sky image at:
\begin{center}
\htmladdnormallink{http://irsc.apo.nmsu.edu/tonight}{http://irsc.apo.nmsu.edu/tonight}
\end{center}
The main APO weather page is at:
\begin{center}
\htmladdnormallink{http://weather.apo.nmsu.edu}{http://weather.apo.nmsu.edu}
\end{center}

\section{Controlling the CCD temperature}

\subsection{NA2 (Leach) camera}

The NA2 camera (Leach camera) has a E2V 2048x2048 CCD that is cooled using liquid
nitrogen (LN2). Because the dewar has a vacuum to insulate it, it is
important that all efforts be made to keep the camera cold continuously;
thermally cycling the camera can lead to degradation of the vacuum and
contamination on the chip.

The dewar for the science CCD needs to be filled with LN2 to keep
the chip cold. We have a remotely operable filling system which can
be commanded to fill the dewar. This can be done by simply typing
FILL. If the telescope is initialized, The FILL command will move
the telescope to the stow position, open the valve from the filling
dewar, and then suspend things for the default fill time (currently
5 minutes); you can specify any desired fill time using \textit{FILL
t}, where t is the requested time in minutes.  The fill system turns
off by timing only; the power to the autofill system is turned off
after the requested time. There is NO sensor that actually checks
to see if the dewar has actually filled, so if there is some pressure
problem, the possibility exists that the dewar will not have been
filled.  As a result, it is good to monitor the actual temperature.

Note that the valve for the fill system is controlled through the
telescope control hardware, so motor power must be on, and the telescope
computer running, for the FILL command to actually do anything!!

The current hold time for a full dewar is approximately 13 hours. It takes
about 1.5 hours to cool the chip down completely if it starts out at
room temperature. In general, it should be possible to fill the dewar just
before starting observing and have it stay cold for the entire night.

The normal operating temperature is -125 C; this is controlled by a
regulated heater inside the dewar. Sometimes the chip is kept
at -130C during the day, just to conserve a bit of LN2.

Apart from manual fills, the CCD software will attempt to do autofills
of the dewar when it detects that it has started to warm up. Autofills
are triggered when the CCD reaches a temperature warmer than -110
C.  If this occurs, the telescope will move to the stow position
(if it has been initialized), and do a fill. If the motor power is
off when the autofill is triggered (as it should be during the day),
the software will automatically turn it on for the duration of the
fill, then turn it off afterward; if the motor power is on at the
trigger, it will remain on afterwards.

\subsection{NA1 (Apogee) camera}

The NA1 acquisition camera is an APOGEE AP7P with a 512x512 CCD. This
is cooled thermoelectrically, and temperature cycling is not so much
of an issue.

The thermoelectic cooler can cool the chip to about 50 degrees C below the
ambient temperature in the dome. The dark current is a strong function of
temperature, so it is desirable to operate as cold as possible. During the
summer, you may not be able to get the CCD colder than -30 to -35 C, while
during the winter, you might get to -50 or -55 C. If you will be doing
dark current subtraction, you will want to get both science and dark frames
at the same temperature. Unfortunately since the ambient temperature 
changes throughout the night, you may not be able to reach the coldest
temperatures at the beginning of the night, and you will have to decide
whether you want to change the temperature in the middle of your observing.

\chapter{Operating the telescope}

\section{The status window/display}

The upper section of the status window gives information about the 
telescope/dome systems. The left column gives the current RA, DEC, 
hour angle of the telescope, coordinate epoch, and position angle.
The next column gives the azimuth, altitude, rotator, focus positions
for each of the three focus motors, and tertiary port. 
The rightmost column gives the
current UT, local sidereal time, airmass, and the focus position in
a coordinate system which gives mean focus, xtilt, and ytilt.

Other entries give the dome position and status (initialized/unitialized,
slaved/not slaved, open/closed), and mirror cover status (open/closed).

If {\it any} of the items appear in reverse video, there is likely to be
something configured such that you will have significant problems with
your observations (e.g., the dome is closed!). Note that no mirror 
covers are currently installed, so the status of these can be ignored.

The telescope positions are not absolutely encoded, so the status only reports
where the computer {\it thinks} the  item is, which is not necessarily
where it {\it really} is, although if things are working well, the
computer should correctly know where things are.  However, if
something seems peculiar, it is important to check with the remote
video cameras and/or by going out and looking at the telescope/dome!

If you hit a key in the status window, it will toggle into engineering
mode, where some additional information is displayed, and where a
graphical window will come up which shows a running plot of the current 
encoder positions relative to the desired positions. If you accidentally
do this, you can minimize the plot window, and switch back to normal
mode by hitting a key again in the status window.

\section{The command window}

All commands should be entered in the command window. For all commands, 
you will not get a command prompt until the previous command completes.

The software allows you to write simple scripts in an external file which
can be read in by the program to issue a series of commands automatically.
Input script files should have commands which are identical to what you
would need to type in the command window. To execute a script, you use
the  command:

\begin{quote}
INPUT filename
\end{quote}

where filename is the name of the file with the commands to execute. Our 
convention is to use .inp as the extension for script files.
A script can call another script, up to 5 layers deep. 

\subsection{Restarting tcomm}

If for some reason things appear to hang up (i.e., nothing returns
for several minutes after you'd really expect it to, you can restart
the tcomm programs after issuing a CTRL-$\setminus$ in the command
window. This should kill all of the windows, and then retyping tcomm
should start everything up again. However, if this is necessary,
please inform Jon of the circumstances - we would really like to
get things working without ANY hangups.

\section{Dome commands}

\begin{hanging}
\item {OD}

The OD command will open the upper dome slit. It will prompt for a confirmation 
before opening.

\item {CD}

The CD command will close the upper dome slit. 

\item {OLD}
 
OLD will open the lower dome slit. \textbf{The upper dome slit must
be opened first, since it has a lip over the lower dome slit}.  
Normally, we \textit{do not open} the lower dome
slit unless we are doing absolute (all-sky) photometry; the reason
for this is that the lower dome slit is not on a hardware watchdog
circuit, so in case of a failure, it will likely not get closed.

\item {CLD}

OLD will close the lower dome slit. \textbf{The lower dome slit must be closed
before the upper dome slit is closed.}

\item {ODS}

ODS will open both upper and lower dome slits in conjunction.
However, normally we \textit{do not open} the lower dome slit unless we are
doing absolute (all-sky) photometry; the reason for this is that the lower
dome slit is not on a hardware watchdog circuit, so in case of a failure, it
will likely not get closed.
 
\item {CDS}

CDS will open both upper and lower dome slits in conjunction.
 
\item {DS (or DM)}

DS toggles dome slaving on/off, i.e. the tracking of the dome position to
the telescope azimuth position.

\item {DA (or DOMEAZ or XDOME) azpos}

DA will move the dome to the requested azimuth position (0 degrees is N, and
az increases towards the east). DA will turn off dome slaving automatically 
(otherwise the dome would just rotate back to where the telescope is pointed!).

\end{hanging}

\section{Telescope commands}

\subsection{Moving the telescope} 

The following commands can be used to move the telescope:

\begin{hanging}
\item {CO [{\it hh mm ss dd mm ss}]}

The CO command will move the telescope to some user-specified coordinates. 
If you do not specify a position on the command line, the program
will prompt for an RA and DEC, and then ask you to confirm the
move. The coordinates will be interpreted using the current epoch,
which defaults to 1950 (the default epoch can be changed using the
NE command which will prompt you for a new epoch). Format is hh mm ss,
separated by spaces (you can omit the mm or ss if they are zero).

\item{SA}

Move the telescope to an SAO star. The program will ask whether you
want an SAO near near the current telescope position (T), near
some other specified RA/DEC (O), or some specified ALT/AZ (A). If
you choose one of the latter two options, you will be prompted for
coordinates. You will then be prompted for an acceptable magnitude 
range for the SAO star ($<$CR$>$ to allow any brightness). Note that
the on-line SAO catalog only contains stars down to 7th magnitude
and does not have fainter catalog members at this time.
The program will find the nearest SAO star to your coordinates, and
ask if you want to move there; enter Y to move.

\item{SAFOC}

Same as SA, but for a SAO catalog sample that contains only fainter
stars with $V>9$.

\item{ALTAZ}

Move the telescope to specified azimuth and altitude.

\item{RF [{\it id}]}

Move the telescope to the position of star {\it id} in an open user
coordinate file (see section ??). If not specified on the command line,
{\it id} will be prompted for.

\item {PM}

Recalls the last 5 previously commanded moves and allows you to select
one to move back to.

\item{QM [{\it $\Delta\alpha \Delta\delta$}]}

Offset the coordinate by $\Delta\alpha$ arcseconds in right ascension and
$\Delta\delta$ arcseconds in declination. If not specified on the command
line, the offsets will be prompted for.

\item{OFFSET [{\it $\Delta x \Delta y$}]}

Offsets the telescope by $\Delta x$ \textit{pixels} along rows and
$\Delta y$ pixels along columns of the CCD. If not specified on the command
line, the offsets will be prompted for. These allow ``instrument plane''
offsets, i.e., in the coordinate system of the CCD even if the instrument
is rotated with N no longer oriented along columns. Unlike the 3.5m,
the instrument plane offsets are specified in pixels, not arcseconds.
The offset is for the \textit{telescope}, so stars will move by 
($-\Delta x,-\Delta y$).

\item{CENTER}

Move the telescope such that a marked location on the CCD frame will be
moved to the center of the CCD frame. This requires that an image has
already been obtained. If you use this command, you will be prompted to mark the
location of an object in the CCD frame; use C to centroid on the position
of the cursor, or I to just take the integer pixel location of the cursor.

\item{LOCATE}

Move the telescope such that a marked location on the CCD frame will be
moved to any another desired location. Again, this requires that an image
has already been obtained. If you use this command, you will be prompted
to mark the location of an object in the CCD frame, then prompted to
mark a second position to which the object will be moved. Mark these
positions using C to centroid at the location of the cursor, 
or I to use the integer pixel location of the cursor.

\item{PA}

Change the position angle to the specified PA. The position angle is
measured from north to vertical on the chip when the image is displayed
in sky orientation (for the PI CCD, this is (1,1) in the upper left).

\item{INIT}

Go through startup sequence that is run when starting tcomm, i.e.
telescope initialization, dome initialization, and louver query.

%\item{HO}
%
%Reinitialize the telescope with a full initialization.
%
%\item{IN}
%
%Reinitialize the telescope with a normal initialization.

\end{hanging}

\subsection{User catalogs}

Users can create catalogs of object positions which can be referenced by
an ID number to maximize efficiency and minimize input error during the
night. The format of the user catalogs is identical to that used by REMARK
on the 3.5m. Each object should be put on a separate line with the following
format:
\begin{quote}
object name $<$TAB$>$ RA (hh:mm:ss) $<$TAB$>$ DEC (dd:mm:ss) $<$TAB$>$ epoch $<$TAB$>$ (proper motion in RA) $<$TAB$>$(proper motion in DEC)
\end{quote}

The following commands can then be used with these files:

\begin{hanging}
\item{OF}

Open a new input file. The file name will be prompted for.

\item{RF [{\it id}]}

Move to object {\it id}. If no {\it id} is given, it will be prompted for

\end{hanging}

\subsection{Coordinate epoch and update commands}

The default coordinate epoch on startup in 1950. The default epoch for
coordinate entry can be changed using:

\begin{hanging}
\item{NE}

Change the input epoch to a new value. The program will give the current
epoch, and prompt for a new value. Note that the epoch on the display will
not change until you move to an object using the new epoch.

\item{UC}

Update coordinates, i.e. set the current position of the telescope. With
the command, the telescope will display a list of the last 5 commanded
positions; you enter the index of the object that the telescope is 
currently pointed at, and the positions of the azimuth and altitude
axes will be adjusted. The normal mode of operation is to slew to an
SAO star (SA command), take an exposure (EXP or QCK), center star (CENTER),
then update coordinates (UC).


\end{hanging}

\subsection{Focus commands}

The telescope can be focussed by moving the position of the secondary. 
The secondary position is controlled by three separate motors, so the
mirror can be tilted as well as pistoned. Tilting the mirrors changes
the collimation and thus is not a motion which should normally be
done! All focus positions are negative: (0,0,0) is when the secondary
is at the home position, farthest from the primary.

Note that the focus is a function of temperature, since the telescope
structure expands/contracts as a function of temperature. As the
temperature increases, the focus decreases, i.e. goes more negative.
%A \textit{rough} 
%guess of the approximate focus can be found from the expression:
%$$focus = -12580 - 18 \mathrm{auxtemp}$$
%where $\mathrm{auxtemp}$ is the auxiliary temperature measured at the
%side of the mirror cell that can be determined by using the WE command
%and looking at the output in the telescope control window.

The focus commands are:

\begin{hanging}
\item{FOCRUN}

Take a series of exposures at different focus values. You will be prompted
for a starting focus value, a delta focus (amount to change focus between
each exposure), a number of exposures, and an exposure time. After entering
these, the program will automatically take the requested number of exposures.
The program will always approach a given focus value from the same direction,
\textit{assuming you specify a positive delta focus value}. When using
FOCRUN, the CCD will automatically switch into a ``fast'' readout mode
(which still isn't that fast!) which has increased readout noise; 
when done, the CCD will be switched back into the normal readout mode.

\item{FO {$f$}}

Move the focus to position $f$. This command will automatically approach
the desired position from the same direction.

\item{DF {$\Delta f$}}

Moves the focus by a relative amount, $\Delta f$. If not specified on the
command line, the amount to move will be prompted for

\item{XTILT $tilt$ (ENGINEERING)}
\item{YTILT $tilt$ (ENGINEERING)}

Tilt the secondary mirror to the specified angle.

\item{SHO (ENGINEERING)}

Move the secondary to the home position (i.e. farthest from primary).

\end{hanging}

\section{Camera programs}

All camera commands are normally entered in the command window, but these
are passed to the appropriate camera control program, which are usually
run from separate windows on the right part of the desktop.

After images are taken, they are normally written out to disk in FITS
format. The images go into directory \textit{images/yymmdd} under the
tcomm home directory, where \textit{yymmdd} refers to the current UT
date. The default filenames are \textit{yymmdd.index.fits} where \textit{index}
is a running index number: both the root filename and the index number
can be changed, see below.

\subsection{The image display window}

The first time an image is taken, each camera program opens an Image Display
window on the console; each subsequent image will be displayed in this
window.  Once an image has been displayed, some interactive commands
are available in the display window,
provided that a wait for input or any other I/O is not pending.  To
interact with the image, simply move the mouse onto the display window.
The current pixel location of the cursor will be displayed in a frame at
the base of the image display along with the pixel intensity.  The arrow
keys are used for fine control (one pixel at a time) of the cursor
position. Note that, to maintain a small window, the image as originally
displayed will likely only show every other, or even every 4th pixel 
(depending on the detector)

The following mouse buttons and keyboard keys are active while the mouse
is located on the image display:

\begin{center}
{\bf Mouse Buttons}\\
\begin{tabular}{ll}
\hline
Button & Function\\
\hline
LEFT  &ZOOM IN, centered on the cursor\\
MIDDLE&ZOOM OUT, centered on the cursor\\
RIGHT &PAN, move the pixel under the cursor to the center\\
\hline
\end{tabular}
\end{center}

\begin{center}
{\bf Keyboard Commands}\\
\begin{tabular}{cl}
\hline
Key & Function\\
\hline
 R &RESTORE image to the original zoom/pan\\
 + &BLINK Forwards through the last 4 images.\\
 - &BLINK Backwards through the last 4 images.\\
 P &Find the PEAK pixel near the cursor \& jump the cursor there\\
 V &Find the LOWEST pixel ("Valley") near the cursor \& jump the cursor there\\
 \# &"Power Zoom" zoom at the cursor to the maximum zoom factor\\
 H &Toggle between small and full-screen cross-hairs\\
 ] &Clear boxes and stuff off the image display\\
\hline
\end{tabular}
\end{center}

\noindent{\bf Color Bar Adjustment:}

If you place the mouse on the color bar, these commands are available
to adjust the contrast of the image:
\begin{example}
  \item[LOW CONTRAST]{Hold down the LEFT Mouse button, drag the left
       end of the color bar.}

  \item[HIGH CONTRAST]{Hold down the RIGHT Mouse button, drag the right
       end of the color bar.}

  \item[ROLL COLOR MAP]{Hold the MIDDLE Mouse button, "roll" the
       color bar left or right.}
\end{example}
The position of the mouse cursor displays the range of intensities
represented by that color.

Pressing the R key while the mouse is on the color bar restores the
original color map (undoing any change of the contrast or "roll" changes
made with the mouse buttons).

Note that you can also redisplay the current image with a different
stretch using the SCALE command, see below. 

\subsection{Camera Commands}

The following commands are available:
\begin{hanging}
\item{EXP [$time$]}

Take an exposure of length $time$ seconds.

\item{QCK [$time$]}

Same as EXP, but filename index exposure is \textbf{not} incremented, so 
next image will overwrite current one, i.e., exposure will not be 
permanently saved.

\item{SETFILT [$name$]}

Change the filter wheel to filter $name$. You must supply a name which
matches that in the current filter table. For a list of available filters,
use SETFILT without any arguments. SETFILT allows you to operate without
knowing the details of which filter is located in which slot. 
SETFILT will also adjust the telescope focus for the relative focus
difference between the filters (the filter specific offset is shown in
parentheses in the status window).

\item{FILTINIT}

Initializes the filter wheel if it has gotten lost.

\item{FILENAME [$name$]}

Change the default root file name for output images to $name$.

\item{NEWEXT [$ext$]}

Change the extension for the next filename to be the number $ext$.

\item{FITS}

Set filetype for future image stores to be FITS.

\item{-DISK/+DISK}

Turn off/on autosaving of images. 

\item{+DISPLAY/-DISPLAY}

Turn on/off autodisplay of images after they are taken.

%\item{+XFER/-XFER}
%
%Turn on/off auto-transfer of images to NMSU after they are taken. Currently,
%automatic transfer occurs to the subdirectory under /home/1m (on the NMSU
%Linux machines) corresponding to the observation date.

\item{SCALE $low\ high$}

Redisplay the current image with greyscale scaling between $low$ and $high$.

\item{SKYSCALE}

Sets mode for autoscaling to have black level somewhat below mean level in
image, and white level above it. This is the default.

\item{FULLSCALE}

Sets mode for autoscaling of images to be from minimum pixel (black) to
maximum pixel (white).

\item{SAMESCALE}

Sets mode for autoscaling of images to be identical to the current scale.

%\item{CENTER}
%
%Move the telescope so that an object/location marked on the display window
%will move to the center of the CCD field. You will be prompted to mark the
%location of an object in the CCD frame; use C to centroid on the position
%of the mouse, or I to just take the integer pixel location of the mouse.

\item{WINDOW  [xs xe ys ye] or [FULL]}

Sets CCD windowing. If 4 coordinates are given, CCD will be set to window
to requested region. If FULL is specified, windowing will be set back to
full chip mode. If nothing is specified, user will be prompted to mark
two opposite corners on the current image display to define the window
region. In all cases, the final requested window position will be displayed
on the current image.

\item{FLAT \textit{ndesired nmax startexp fudge}}

Takes a series of flat fields, adjusting exposures times to maintain good
S/N, for use in taking twilight flats. A ``good'' flat is defined as one
with a mean level (over bias) of somewhere between 5000 and 20000 DN, with
a minimum exposure time of 1 second. The
routine will start out by taking an exposure of length \textit{startexp}.
From this, it will compute a new exposure time which is required to obtain
a mean of 10000, after multiplying by the fudge factor \textit{fudge},
which accounts for the changing brightness of the twlight sky (\textit{fudge}
should be $\sim 1.5$ for evening twilight and arounge $\sim 0.7$ for
morning twilight. In no case will the program let the exposure time be longer
than 30 seconds, and any 30 second flat will be counted as a ``good'' flat
to prevent a large number of unsuccessful flats being taken when it is too
dark.  The program will continue taking flats until it gets
\textit{ndesired} ``good'' flats or until it has taken ``nmax'' flats.

%\item{CCDFAST}
%
%\textit{Princeton Instruments CCD only}.
%Switches the CCD into ``fast'' readout mode, in which a full exposure takes
%about 30 seconds instead of a minute, at the price of increased readout
%noise.
%
%\item{CCDSLOW}
%
%\textit{Princeton Instruments CCD only}.
%Switches the CCD into ``normal'' readout mode, in which a full exposure takes
%about 60 seconds.

\item{SETTEMP} \textit{temp}
%\textit{Apogee CCD only}.
Set the desired CCD temperature (degrees C). See discussion in previous
section.

\end{hanging}

\subsection{Standard star observations}

Several commands are available to increase your efficiency in observing
standard stars. There is a full catalog of all of the Landolt standards,
plus several others, in the top level observe directory, with filename
standards.apo. To allow use of this catalog in conjuction with your own
catalog of objects, there are separate commands to open and read from the
standards file, so you can have both the standards file and a user file open
simultaneously. The standards file also includes the UBVRI magnitudes of
the stars, and the observing program knows about the throughput of the
1m telescope, so the program is also capable of choosing reasonable exposure
times for you.

The relevant commands are:

\begin{hanging}

\item{OS}

Open the standards file (analagous to OF for user files). 
If you are running in a subdirectory under the
top level observe login directory, you will enter: ../standards.apo

\item{RS [{\it id}]}

Move to standard {\it id}. If no {\it id} is given, it will be prompted for.

\item{STAN [\textit{filtname}]}

Take an exposure of the current standard through filter \textit{filtname}. You
must have already moved to the star using the RS command. The program will move
the filter automatically and will take an exposure based on the known
magnitude of the star and the throughput of the system.

\item{FUDGE [\textit{fudge}]}

Sets a fudge factor for the automatic expossure time calculation of the STAN
commands. All standard star exposure times will be multiplied by the fudge
factor specified. 

%\item{MAG m1 m2 m3 m4 m5 m6}
%
%\textit{Do not use this command unless instructed to - defaults should be
%correct!)}
%Tell the program which magnitude to use for the exposure time calculation
%for each filter. There are 6 filter slots, so 6 integers must be specified.
%The possible choices for the magnitude codes are 1, 2, 3, 4, or 5, corresponding
%to U,B,V,R,or I. So, for example, if UBVRI are loaded in slots 12345, you
%would use the command MAG 1 2 3 4 5 0. The correct defaults are loaded at
%startup.
%
%\item{ZERO z1 z2 z3 z4 z5 z6}
%
%\textit{Do not use this command unless instructed to - defaults should be
%correct!)}
%Tell the program which zeropoints to use for the exposure time calculation
%for each filter. There are 6 filter slots, so 6 zeropoints must be specified.
%The zeropoint gives the magnitude which gives 1 DN/second through the
%desired filters.

\end{hanging}

\section{Guiding commands}

The current guider CCD is a camera purchased from 
\htmladdnormallink{Finger Lakes Instrumentation}{http://www.fli.com}; it
uses a E2V 1024$\times$1024 CCD with 13.5$\mu$ pixels.
%\htmladdnormallink{Spectrasource Inc.}{http://www.spectrasource.com}; it
%uses a Kodak 512$\times$512 CCD with 20$\mu$ pixels.

In general, you can issue the same commands to the guider CCD as you
can to the science CCD. To do so, use the same command names but preceeded
by the letter ``G'', e.g., GEXP \textit{t} will take a guider exposure
of length \textit{t} seconds.

Since our telescope is a three-axis telescope (alt-az-rot) and a single
guide star only contains two positions (x-y), the absolute quality of 
the guiding depends on an accurate pointing model (although less so than
if there were no guider at all!). We use the guider observations to
update altitude and azimuth, and assume that the rotator angle is correct.
Since the guider is located off axis, errors in the rotator angle will
manifest themselves as small arcs of stars in the science camera, centered
around the position of the guide camera (which is off of the chip). 

%To minimize errors from inaccuracies in the pointing model, we recommend that,
%before starting to guide,
%you go to an SAO star near your object, take an exposure, do a CENTER command,
%and then update the telescope coordinates with a UC command before moving
%to your object and starting to guide. 

To start guiding, first take a guider exposure to see if there is an
acceptable guide star and to determine a good exposure time. Ideally, you
will have a star which has good S/N with an exposure time of less than
or approximately equal to 1 second. Once you have a picture taken with
a good exposure time, you can start guiding by issuing the GUIDE command.
This will take a picture with your exposure time, ask you to mark the
guide star in the guide CCD image, and will then start autoguiding on
this star. 

If your original image was taken in full frame mode (WINDOW FULL), then
all guiding images will be done with a small subframe around the
guide star. If your original image was taken in windowed mode, then
all guiding images will be taken with the same windowing.

Guiding will continue until you do a slew with the telescope or unless
you issue a GUIDEOFF command.

Other commands relevant to guiding:

\begin{hanging}

\item GUPDATE \textit{n}

Sets the number of guide exposures to average centroids from before sending
a position update command to the telescope. Default is 5.

\item GSIZE \textit{n}

Sets the size of the box used for computing centroids. Default is 11 pixels.

\item NOSHUTTER

Puts guider in NOSHUTTER mode where exposures are taken without using the
shutter. You must remember to issue the OPEN command to open the shutter
before starting to guide! 
%Currently, this command must be issued from the
%guider CCD control window (not the command window).

\item SHUTTER

Puts guider into normal SHUTTER mode.
%Currently, this command must be issued from the
%guider CCD control window (not the command window).

\item OPEN

Opens the guider shutter.
%Currently, this command must be issued from the
%guider CCD control window (not the command window).

\item CLOSE

Closes the guider shutter.
%Currently, this command must be issued from the
%guider CCD control window (not the command window).

\item WRITE

Toggles mode where each individual guide exposure is saved to disk. Default
is \textit{not} to save each image.
Currently, this command must be issued from the
guider CCD control window (not the command window).

\item GUIDEFOC

\item GUIDEADJ

\item GUIDEMOVE

\item GUIDEINIT

Initializes the guider stages by moving them to their home positions. Users
should \textbf{not} issue this command without being instructed to do so!

\end{hanging}

\chapter{Shutting down}

To close the telescope down, you need to move the telescope to its stow
position, close the dome, and turn off motor power. In general, you should
NOT quit the command program or the telescope control program!

The following commands are used:

\begin{hanging}
%\item{CM}
%
%Closes the mirror covers. The telescope will automatically move to be 
%pointing nearly vertically, since this reduces stress on the mirror cover
%motors. 

\item{ST}

Stows the telescope in the default stow position.

\item{CD}

Closes the dome.

\item{CL}

Closes the louvers. Make sure that dome fans are off first (\textit{power1m
off domefan})!

%\item{QU}
%
%Quits the program. When you enter QU from the command window in the control 
%room,  the program will ask you if you wish to quit the telescope control
%program in the dome as well as the remote programs. Generally, you should
%answer Y to this question; doing so, will require that you restart the
%dome telescope program (by rebooting the telescope control PC) and reinitialize
%the telescope. If you will be restarting tcomm soon and do not want to be
%forced to reinitialize, you can enter N. However,
%\textit{Do not kill the power to the telescope control computer in the dome
%until you have gracefully exited the program running there, or else you
%will not be able to get started again!.}
%If you enter QU on the TOCC PC in the dome, you can quit the program running
%in the dome.
%
%Finally, the program will ask if you wish to kill power to everything in
%the dome. If you have quit the telescope control program in the dome, you
%can answer Y here; if not, answer no.

\end{hanging}

\appendix

\chapter{Command summary}

\input commands

\chapter{Standard stars }

For user convenience, a master coordinate/standards file with a large number 
of standard stars is available which can be used with the OS/RS and STAN
commands. OS/RS allow ease of pointing to a desired standard, and the STAN
command can be used to take an exposure, letting the software calculate a
good exposure time based on the standard magnitude and the system
throughput through a desired filter.

The master standards file is called standards.apo, and resides in the top
level observe directory (/home/loki2/observe/standards.apo). The current
version contains 918 entries. Entries 1-223 come from Landolt 1982,
224-250 are dummy entries, 251-280 are standards in M67 from Montgomery,
281-300 are dummy entries, 301-826 are from Landolt 1992 (some of which
duplicate entries from Landolt 1982), 827-900 are
dummy entries, 901-918 are southern spectrophotometric standards
from Landolt 1992a.

%An abbreviated list of recommended single star standards is presented below.
%These stars were chosen to be at a good brightness for the 1m, sample
%stars around the sky, with care given to include a red and blue star at
%all right ascensions. Stars marked with an asterisk are especially preffered.
%
%The ID numbers are the numbers by which the star is referred to in the
%standards.apo file, i.e., you should do a RS id to move to a given
%star.
%
%\pagebreak
%\begin{tabular}{llllllll}
% &  ID& NAME &   RA    &      DEC  &  Epoch&     V  &     B-V    \\
%$\ast$&  3&BD -15 115&00:37:36.00&-15:04:51.0&1985.0 & 10.881 & -0.190  \\
%$\ast$&324&SA 92 409&00:55:14.00&+00:56: 7.0&2000.0 & 10.627 &  1.138  \\
% &  4&BD -12 134&00:46:19.00&-11:57:32.0&1985.0 & 11.775 & -0.294  \\
%$\ast$&311&SA 92 235&00:53:16.00&+00:36:18.0&2000.0 & 10.595 &  1.638  \\
% &322&SA 92 342&00:55:10.00&+00:43:14.0&2000.0 & 11.613 &  0.436  \\
% &333&SA 92 263&00:55:40.00&+00:36:18.0&2000.0 & 11.782 &  1.048  \\
%\multicolumn{8}{l}{ }\\
%$\ast$& 21&HD 12021&01:57:10.00&-02:10:37.0&1985.0 &  8.874 & -0.082  \\
%$\ast$&356&SA 93 424&01:55:26.00&+00:56:43.0&2000.0 & 11.620 &  1.083  \\
% &352&F 16&01:54: 8.00&-06:42:54.0&2000.0 & 12.406 & -0.012  \\
%$\ast$& 15&SA 93 326&01:54: 4.00&+00:42:40.0&1985.0 &  9.569 &  0.454  \\
%\multicolumn{8}{l}{ }\\
%$\ast$& 29&BD -02 524&02:56:54.00&-02:03:26.0&1985.0 & 10.307 & -0.104 \\
%$\ast$&374&SA 94 251&02:57:46.00&+00:16: 2.0&2000.0 & 11.204 &  1.219 \\
% &361&F 22&02:30:17.00&+05:15:51.0&2000.0 & 12.799 & -0.054 \\
% &375&SA 94 702&02:58:13.00&+01:10:53.0&2000.0 & 11.594 &  1.418 \\
% &373&SA 94 242&02:57:21.00&+00:18:38.0&2000.0 & 11.728 &  0.301 \\
%\multicolumn{8}{l}{ }\\
%$\ast$&381&SA 95 96&03:52:54.00&+00:00:19.0&2000.0 & 10.010 &  0.147\\
% &420&SA 95 236&03:56:13.00&+00:08:43.0&2000.0 & 11.491 &  0.736\\
%$\ast$&415&SA 95 74&03:55:31.00&-00:09:13.0&2000.0 & 11.531 &  1.126\\
%$\ast$& 36&SA 95 52&03:53:28.00&-00:10:58.0&1985.0 &  9.574 &  0.529\\
%\multicolumn{8}{l}{ }\\
%$\ast$&422&SA 96 36&04:51:43.00&-00:10:12.0&2000.0 & 10.591 &  0.247 \\
%$\ast$&426&SA 96 235&04:53:19.00&-00:05: 4.0&2000.0 & 11.140 &  1.074 \\
% &425&SA 96 83&04:52:59.00&-00:14:44.0&2000.0 & 11.719 &  0.179 \\
% &423&SA 96 737&04:52:35.00&+00:22:29.0&2000.0 & 11.716 &  1.334 \\
%$\ast$& 43&SA 96 393&04:51:44.00&+00:00:39.0&1985.0 &  9.652 &  0.598 \\
%\multicolumn{8}{l}{ }\\
%$\ast$&433&SA 97 351&05:57:37.00&+00:13:42.0&2000.0 &  9.781 &  0.202 \\
% & 51&SA 97 346&05:56:41.00&+00:13:17.0&1985.0 &  9.260 &  0.594 \\
% &429&GD 71&05:52:28.00&+15:53:15.0&2000.0 & 13.032 & -0.249 \\
%$\ast$&435&SA 97 284&05:58:25.00&+00:05:12.0&2000.0 & 10.788 &  1.363 \\
% &434&SA 97 75&05:57:55.00&-00:09:29.0&2000.0 & 11.483 &  1.872 \\
%\multicolumn{8}{l}{ }\\
%$\ast$&462&SA 98 653&06:52: 5.00&-00:18:19.0&2000.0 &  9.539 & -0.004 \\
%$\ast$&461&SA 98 193&06:52: 4.00&-00:27:18.0&2000.0 & 10.030 &  1.180 \\
% &443&SA 98 978&06:51:34.00&-00:11:28.0&2000.0 & 10.572 &  0.609 \\
% &459&SA 98 185&06:52: 2.00&-00:27:21.0&2000.0 & 10.536 &  0.202 \\
% &467&SA 98 670&06:52:12.00&-00:19:17.0&2000.0 & 11.930 &  1.356 \\
%\multicolumn{8}{l}{ }\\
%$\ast$&498&SA 99 438&07:55:54.00&-00:16:51.0&2000.0 &  9.398 & -0.155 \\
% & 69&SA 99 367&07:53:26.00&-00:23:13.0&1985.0 & 11.149 &  1.005 \\
% & 73&SA 99 418&07:54:41.00&-00:15: 7.0&1985.0 &  9.474 & -0.041 \\
%$\ast$& 65&BD+05 1668&07:26:36.00&+05:16:16.0&1985.0 &  9.843 &  1.557 \\
% & 72&SA 99 408&07:54:27.00&-00:23: 8.0&1985.0 &  9.807 &  0.407 \\
%\multicolumn{8}{l}{ }
%\end{tabular}
%%\pagebreak
%
%\begin{tabular}{llllllll}
% &  ID&    RA    &      DEC  &  Epoch&     V  &     B-V    \\
%$\ast$&500&SA 100 241&08:52:35.00&-00:39:48.0&2000.0 & 10.139 &  0.157 \\
%$\ast$&505&SA 100 394&08:53:55.00&-00:32:21.0&2000.0 & 11.384 &  1.317 \\
% & 79&SA 100 606&08:52:12.00&-00:06: 4.0&1985.0 &  8.641 &  0.052 \\
% &504&SA 100 280&08:53:36.00&-00:36:41.0&2000.0 & 11.799 &  0.494 \\
% &501&SA 100 162&08:53:15.00&-00:43:29.0&2000.0 &  9.150 &  1.276 \\
%\multicolumn{8}{l}{ }\\
%$\ast$&551&SA 101 363&09:58:19.00&-00:25:35.0&2000.0 &  9.874 &  0.261 \\
%$\ast$&536&SA 101 281&09:57: 5.00&-00:31:38.0&2000.0 & 11.575 &  0.812 \\
%$\ast$&511&BD -12 2918&09:31:18.00&-13:29:20.0&2000.0 & 10.067 &  1.501 \\
% & 85&HD 84971&09:47:59.00&-02:38:29.0&1985.0 &  8.636 & -0.159 \\
%\multicolumn{8}{l}{ }\\
%$\ast$&100&SA 102 58&10:54:32.00&-01:20:38.0&1985.0 &  9.380 &  0.060 \\
%$\ast$&562&SA 102 620&10:55: 6.00&-00:48:19.0&2000.0 & 10.069 &  1.083 \\
% &561&G44 40&10:50:54.00&+06:48:57.0&2000.0 & 11.675 &  1.644 \\
% &564&SA 102 1081&10:57: 4.00&-00:13:10.0&2000.0 &  9.903 &  0.664 \\
%\multicolumn{8}{l}{ }\\
% &566&G163 50&11:08: 0.00&-05:09:26.0&2000.0 & 13.059 &  0.035 \\
%$\ast$&107&HD 100340&11:32: 3.00&+05:21:40.0&1985.0 & 10.117 & -0.242 \\
%$\ast$&112&SA 103 526&11:56: 8.00&-00:25:13.0&1985.0 & 10.903 &  1.089 \\
%$\ast$&106&BD+05 2468&11:14:46.00&+05:02:20.0&1985.0 &  9.348 & -0.116 \\
% &568&G10 50&11:47:44.00&+00:48:55.0&2000.0 & 11.153 &  1.752 \\
% &569&SA 103 302&11:56: 6.00&-00:47:54.0&2000.0 &  9.861 &  0.368 \\
%\multicolumn{8}{l}{ }\\
%$\ast$&595&SA 104 461&12:43: 7.00&-00:32:21.0&2000.0 &  9.705 &  0.476 \\
%$\ast$&605&SA 104 598&12:45:17.00&-00:16:37.0&2000.0 & 11.479 &  1.106 \\
% &114&SA 104 337&12:41:42.00&-00:29:21.0&1985.0 & 11.207 &  0.768 \\
%\multicolumn{8}{l}{ }\\
% &615&BD+2 2711&13:42:21.00&+01:30:17.0&2000.0 & 10.367 & -0.166 \\
%$\ast$&124&SA 105 66&13:38:49.00&-01:08:53.0&1985.0 &  9.426 &  0.977 \\
% &616&HD 121968&13:58:52.00&-02:55:12.0&2000.0 & 10.254 & -0.186 \\
%$\ast$&612&SA 105 505&13:35:25.00&-00:23:38.0&2000.0 & 10.270 &  1.422 \\
% &614&SA 105 815&13:40: 4.00&-00:02:10.0&2000.0 & 11.453 &  0.385 \\
% &606&G14 55&13:28:22.00&-02:21:28.0&2000.0 & 11.336 &  1.491 \\
%\multicolumn{8}{l}{ }\\
%$\ast$&133&SA 106 485&14:43:28.00&-00:33:20.0&1985.0 &  9.484 &  0.380 \\
% &128&SA 106 834&14:38:43.00&-00:11: 0.0&1985.0 &  9.088 &  0.701  \\
%$\ast$&619&SA 106 700&14:40:52.00&-00:23:36.0&2000.0 &  9.785 &  1.362  \\
%\multicolumn{8}{l}{ }\\
%$\ast$&135&SA 107 544&15:36: 3.00&-00:12:14.0&1985.0 &  9.037 &  0.401  \\
%$\ast$&657&SA 107 484&15:40:17.00&-00:21:13.0&2000.0 & 11.311 &  1.237  \\
% &139&SA 1006 544&15:37:48.00&+00:17:16.0&1985.0 & 11.715 &  0.764  \\
%\multicolumn{8}{l}{ }
%\end{tabular}
%\pagebreak
%
%\begin{tabular}{llllllll}
% &  ID&    RA    &      DEC  &  Epoch&     V  &     B-V    \\
% &145&HD 149382&16:33:36.00&-03:58:59.0&1985.0 &  8.944 & -0.281  \\
% &146&SA 108 1332&16:34:35.00&-00:02:18.0&1985.0 &  9.199 &  0.384  \\
% &149&SA 108 1491&16:36:28.00&-00:00:54.0&1985.0 &  9.059 &  0.965  \\
%$\ast$&673&SA 108 551&16:37:47.00&-00:33: 6.0&2000.0 & 10.703 &  0.179  \\
%$\ast$&671&SA 108 475&16:37: 0.00&-00:34:40.0&2000.0 & 11.309 &  1.380  \\
% &674&SA 108 1918&16:37:50.00&-00:00:37.0&2000.0 & 11.384 &  1.432  \\
%\multicolumn{8}{l}{ }\\
%$\ast$&154&HD 160233&17:37:56.00&+04:21:19.0&1985.0 &  9.095 & -0.054  \\
%$\ast$&690&SA 109 537&17:45:42.00&-00:21:34.0&2000.0 & 10.353 &  0.609  \\
%$\ast$&682&BD -4 4226&17:05:15.00&-05:05: 5.0&2000.0 & 10.071 &  1.415  \\
% &688&SA 109 199&17:45: 2.00&-00:29:28.0&2000.0 & 10.990 &  1.739  \\
% &683&SA 109 71&17:44: 6.00&-00:24:59.0&2000.0 & 11.493 &  0.323  \\
%\multicolumn{8}{l}{ }\\
%$\ast$&170&HD 173637&18:45:50.00&-07:56:56.0&1985.0 &  9.375 &  0.236  \\
%$\ast$&727&SA 110 450&18:43:52.00&+00:22:58.0&2000.0 & 11.585 &  0.944  \\
%$\ast$&698&SA 110 340&18:41:29.00&+00:15:22.0&2000.0 & 10.025 &  0.303  \\
% &730&SA 110 319&18:43:55.00&+00:02: 1.0&2000.0 & 11.861 &  1.309  \\
% &724&SA 110 441&18:43:34.00&+00:19:40.0&2000.0 & 11.121 &  0.555 \\
% &703&SA 110 355&18:42:19.00&+00:08:24.0&2000.0 & 11.944 &  1.023 \\
%\multicolumn{8}{l}{ }\\
%$\ast$&174&SA 111 2522&19:36:19.00&+00:35:31.0&1985.0 &  9.706 &  0.164 \\
% &178&SA 111 2009&19:37: 9.00&+00:24:24.0&1985.0 & 10.608 &  0.889 \\
%$\ast$&731&SA 111 775&19:37:16.00&+00:12: 5.0&2000.0 & 10.744 &  1.738 \\
% &177&SA 111 1969&19:36:58.00&+00:23:44.0&1985.0 & 10.384 &  1.961 \\
% &733&SA 111 1925&19:37:29.00&+00:25: 1.0&2000.0 & 12.388 &  0.395 \\
%\multicolumn{8}{l}{ }\\
%$\ast$&744&SA 112 805&20:42:46.00&+00:16: 8.0&2000.0 & 12.086 &  0.152 \\
%$\ast$&743&SA 112 275&20:42:36.00&+00:07:21.0&2000.0 &  9.905 &  1.210 \\
% &739&SA 112 595&20:41:18.00&+00:16:28.0&2000.0 & 11.352 &  1.601 \\
% &741&SA 112 223&20:42:14.00&+00:09: 1.0&2000.0 & 11.424 &  0.454 \\
% &745&SA 112 822&20:42:55.00&+00:15: 4.0&2000.0 & 11.549 &  1.031 \\
%\multicolumn{8}{l}{ }\\
%$\ast$&764&SA 113 466&21:41:28.00&+00:40:14.0&2000.0 & 10.004 &  0.454 \\
%$\ast$&767&SA 113 475&21:41:51.00&+00:39:19.0&2000.0 & 10.306 &  1.058 \\
%$\ast$&194&HD 205556&21:35:11.00&+05:25:11.0&1985.0 &  8.301 & -0.054 \\
% &765&SA 113 259&21:41:44.00&+00:17:37.0&2000.0 & 11.742 &  1.194 \\
%\multicolumn{8}{l}{ }\\
% &215&HD 216135&22:49:41.00&-13:23:12.0&1985.0 & 10.112 & -0.115 \\
%$\ast$&806&SA 114 750&22:41:45.00&+01:12:38.0&2000.0 & 11.916 & -0.041 \\
%$\ast$&807&SA 114 670&22:42:10.00&+01:10:17.0&2000.0 & 11.101 &  1.206 \\
% &805&SA 114 548&22:41:37.00&+00:59: 7.0&2000.0 & 11.601 &  1.362 \\
% &209&SA 114 755&22:41:22.00&+01:12: 6.0&1985.0 & 10.908 &  0.571 \\
%\multicolumn{8}{l}{ }\\
%$\ast$&823&SA 115 420&23:42:36.00&+01:05:58.0&2000.0 & 11.161 &  0.468 \\
% &824&SA 115 271&23:42:41.00&+00:45:10.0&2000.0 &  9.695 &  0.615 \\
%$\ast$&825&SA 115 516&23:44:15.00&+01:14:13.0&2000.0 & 10.434 &  1.028 \\
%\multicolumn{8}{l}{ }
%\end{tabular}

\chapter{Transferring files to NMSU}

There is a simple script available to transfer image files from
a given night down to a disk on ganymede at NMSU. Using this script, you
can specify which images to transfer, and also, at what time
the transfer will take place. 

%It is important to try to minimize the
%network traffic from APO during the night when possible, because the 3.5m
%is often operated remotely and can suffer from poor network performance.
%Consequently, we recommend doing large image transfers early in the morning,
%after the 3.5m has closed, but before the network down at NMSU gets busy;
%7 am is a generally good time. Of course, if you are observering remotely
%and have been using +XFER, then your images will have already been
%transfered as you took them.

To run the transfer script, simply use:
\begin{quote}
dosftp \textit{date} \textit{time}
\end{quote}
This will transfer all the images from the specified date at the
specified time; if \textit{now} is specified, the image transfer will
start immediately.

For example, to transfer images from 981108 at 7am, use
\begin{quote}
dosftp 981108 7am
\end{quote}

The images will be placed in /home/1m/\textit{date}, viewable from all of
the Linux machines at NMSU.

%\chapter{Transferring files to the telescope control program}
%
%You may wish to access two types of files from the telescope control
%program: script files with lists of commands, and user coordinate files
%with your object names and positions. To do so, you must first transfer
%these files to the appropriate location on the telescope control computer
%(loki).
%
%You can do this using the command
%\begin{quote}
%copyscript \textit{filename}
%\end{quote}
%
%This will put the specified file (either script or coordinate file) in
%a scripts directory on loki under a subdirectory given by the username
%of the account from which you are running the command. For example,
%if you are logged in as holtz, the files will go in a holtz/ subdirectory.
%This allows different users to have script/coordinate files of the same
%name without conflicting with each other.
%
%To access files copied with copyscript using the INPUT or OF commands,
%specify the file as username/scriptname, e.g.:
%\begin{quote}
%INPUT username/scriptname
%\end{quote}

\chapter{Robotic operation notes}

\section{Input file format}

\input robotic.tex

\chapter{Telescope recovery notes}

%\begin{itemize}
%
%\item Positive z motion sends rotator counterclockwise
%
%\item Positive x motion sends azimuth counterclockwise from above
%
%\item Recovery mode. 
%\begin{enumerate}
%  \item Start code (telpos.sav) and position rotator (PC) to be clockwise 
%from home position.
%  \item Stop tocc.
%  \item Set last position of rotator to 0 in toccscf.new, and saved\_pos to
%be -9999.
%  \item Restart tocc, and initialize
%\end{enumerate}
%
%\end{itemize}

\include{refs}
\printindex

\end{document}
