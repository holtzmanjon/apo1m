\documentclass{article}[12pt]

\usepackage{epsfig}
\usepackage{html}

\textwidth=7in
\textheight=9.5in

\hoffset=-1.0in
\voffset=-0.5in

\title{Overview of 1-meter Systems}

\epsfxsize 7in

\begin{document}

\maketitle

\vskip 0.5in

Complete document in a
\htmladdnormallink{single PostScript file}{../systems.ps.gz}.


\section{Introduction}

This manual provides an overview of the hardware and software 
systems at the 1-meter.
It is intended to be a starting place.  In some cases, it will point
you to places where you can find more information about a subject.
In other cases, complete documentation will be here in the manual.

Other possibly useful documentation include an overview of the
\htmladdnormallink{telescope/instrument design and performance}
{http://control1m.apo.nmsu.edu/tcomm/1m} , and the
\htmladdnormallink{telescope operation manual}
{http://control1m.apo.nmsu.edu/tcomm/man}.

The manual is organized into five primary sections covering the following:

Telescope Mechanical Systems: Includes information about encoders, motors, etc.

Telescope Optical Systems: Includes information about mirror cleaning, mirror covers, and mirror removal for service.

Instruments: Documents information about the Princeton Camera, filter wheel and guider.

Observatory Dome: Includes documentation of the upper and lower slit as well as dome rotation.

Auxiliary Systems: Includes information about Liquid Nitrogen Autofill system  

There are a few notebooks and manuals that one should be familiar with.
The primary notebooks are the yellow binders entitled, "Autoscope
Model TCS-200  Telescope Control System" and "Autoscope Model TCS-200
Observatory Control System."  These are referred to as "The yellow
telescope binder" and "The yellow observatory binder" respectively in
this document.  Most of the electrical drawings for the telescope are
documented there.  Beware... not all drawings supplied by Autoscope are
for as-built systems.  Perhaps more important than the drawings, the
notebooks contain vendor contact information.  There are copies of these
notebooks in the Astronomy Department and at Apache Point Observatory.

There is a set of blueprints for the telescope in the 1m control room at
APO. As with the documentation, there is no guarantee that this accurately
represents the as-built system.

There is another notebook that was maintained by Dave Summers when he was
working on the 1-meter.  It is a large blue notebook with a label reading
"Dave Summers' 1-meter notes."  In this manual it will be referred to as
"The Blue Notebook" There is only one copy of this notebook and it is
located at NMSU.

There is a blue photo album located in the 1m control room at APO that
includes photo documentation of some procedures.  I'll refer those photos
at points in this manual. There are also digital photos at \htmladdnormallink{http://control1m.apo.nmsu.edu/1m/docs/photos/}{http://control1m.apo.nmsu.edu/1m/docs/photos/}

Finally, there is a manual entitled "Filter-Guider Module \&
Interchangeable Filter Wheel for New Mexico State University."  This is
documentation provided by Dr. Peter Mack of Astronomical Consultants
and Equipment in Tucson, Arizona that I've edited to reflect the as-
built state of things.  It will be referred to as the "Guider book."
This has been included in the main notebooks.

\section{Telescope Mechanical Systems}

The telescope uses an alt-az mount. There are single azimuth, altitude,
and rotation motors.  There are three actuators attached to the secondary
mirror to allow motorized focus and tilt (but not translation) adjustment.
In September 2008, a new tertiary mount was installed that has a motor
for tertiary rotation.

\subsection{Major structural}

\subsubsection{Azimuth}

The entire telescope rests on the main azimuth bearing at the base of the
cone. This bearing cannot be visually inspected; no serious investigation
of it was made until May 2004. At that time, it was determined that the
bearing is likely NOT the bearing called out in the drawings, it appears
to be a size smaller. Jon Davis believes that it is an SKF bearing. 

During May 2004, the telescope developed difficultly moving in azimuth, 
prompting the investigation just mentioned. At this is time, it was
determined there was a felt seal at the top of the bearing that was
partially out of place, allowing some access to the bearing (for dirt,
etc.) Insertion of some cable ties suggested there was little or no
lubrication on the bearings. A visual probe was attempted and it was
determined that the inner surface of the bearing does appear to be covered,
with just a small access from the inner diameter that appears to also have
a felt seal around it. At this point, it was decided that grease lubrication
could only help the situation, so about 1/3 tube of grease was applied (Mobil
fairly lightweight, type ???; this
was done through a small grease port that did not have a grease fitting; we
used such a fitting but the thread of the hole apparently didn't match the
fitting well, so we could not get the fitting to stay in. After lubrication,
a new felt seal was installed using masking tape.

After the lubrication, the telescope appeared to move much smoother, but we 
are still waiting to see if this is a long lasting improvement. The smoothness
was quantified by attaching a force meter to the telescope structure near the
rotator; it took about 40 lbs of force to get the telescope moving, and then
about 25 lbs to keep it moving smoothly. Prior to the lubrication, it took
50 lbs to get it going and 35-40 lbs, irregularly, to keep it moving; the
improvment was fairly dramatic.

Some additional lubrication was injected in November 2006, as the telescope
seemed to be getting a bit more difficult to move. 

Bill K. suggested that Mobil SHC 1500 grease was used, but Ben H. thinks that
is unlikely and suggests Mobil SHC 460 was used. Nick M. says that all of the
Mobil SHC greases are compatible and recommends SHC 460.

The behavior of the azimuth axis warrants monitoring with time, and presumably
routine lubrication, although one cannot continue to add grease indefinitely
without cleaning out the bearings at some point. Servicing the bearing, however,
will require lifting the entire telescope.

In 2010, the pointing appeared to get noticeably worse. In November 11, Nick 
MacDonald helped to inspect the system. He recommended getting replacement
bearings for the azimuth wheel bearings, and to replace the cam followers. We
also increased the pre-load on the soft springs a bit.

Nick recommended Applied Industrial Technology in Albuquerque as a source for
bearings. He determined that the az drive bearings are Fafnir S10PP2C3 (there are
16 of these), and that the az drive idlers are Migill CF1S (there are 16 of these).

The idlers/cam followers for both az and alt were rebuilt by Ben Harris summer 2013.

Az drive re-greased 2014 May 30 (see log).


\subsubsection{Altitude}

From the drawings, it appears that the altitude and rotator bearings are sealed
type bearings, i.e. without maintenance needs.

In Nov 11, Nick M. looked at the altitude encoder, and saw that lots of metal 
flakes were coming off of the altitutde drive. We removed the encoder and
cleaned it up. When remounted, we left off on bracket that appeared to be
overconstraining the system (put it in the toolchest). We also noticed that
the fit of the encoder housing is very tight over the wheel, and wondered
whether it should be remachined.

Nick recommended Applied Industrial Technology in Albuquerque as a source for
bearings. He determined that the alt drive bearings are Nic 5368 (there are
4 of these), and that the alt drive idlers are Migill CF1/2S (there are 4 
of these).

In Decebmer 2012, we pulled off the altitude encoder and replaced the bearings
and cam followers. The bearings are press-fit into the housing. However, one 
of them was quite loose (the one where the encoder is), so we put it in with 
some LocTite to try to keep it in place. The internal shaft is loose in the
bearings, but we think that with preload, it will engage the bearing (and this
appears to be the case).


\subsection{Motors}

The telescope az, alt, and rotator drive motors were manufactured by
Parker-Compumotor.  Their technical support number is (800) 358-9070.
The motors are Dynaserv series DR motors.  There are several copies
of the manual for the motor drivers around.  There are copies in the
Yellow Telescope Manual and the Big Blue Notebook.  The motors and their
drivers are matched sets.  The model number of each motor is noted on
the motor drivers. (Hard to see for sure, but best guess: Az drive DR1070E-115
(resolver, 70N, E means 8 inch, 115V) SN 93020400314, altitude 
DR1015B-115 (15N, B 6-inch) SN 93030900107, rot DR1060B-116 (60N, 6-inche) 
SN93031000075). These are Generation 1 drives (no longer made), and
must be matched set. 

( 2/12 Information on potential replacement drives from Parker/Computmotor: now
any 1070 drive can match with any driver. New ones are digital. Motors
have not changed in output. Footprint is different.  Cabling to motor
will be slightly different. Interface to the drivers will be quite
different. New drives have a 36 pin connector instead of old 50 pin
connector. Consult with Peter Mack, who has done stuff like this. Need
to know what the I/O voltages used for controlare.  ROM quote: 
DRG-1060B-115V motor+driver \$6729. )

The motor drivers are located in the left-hand side of the computer rack
behind the weather station.  If a motor ceases to function, sometimes,
you can gain clues to the problem by looking at the digital readout on
the face of the driver.  The digital readout will read "0" for normal
operation.  Other codes that might be displayed are in the Compumotor
manual.

The motors can be tuned as described in the Compumotor manual, and this
has been done several times, and should be periodically checked. To
tune the motors, one uses an oscilloscope attached to the POSN and AGRND
leads on each motor driver. You put the motors/controllers in a test
mode via a recessed switch on the motor driver. This is supposed to
generate a square wave which you can see on the oscilloscope. There are
three adjustments that can be made  via small pots: the FC (characteristic
frequency) switch, the lim integral switch, and the DC gain. The manual
describes these in a bit more detail. Generally when we have done this,
we have never been able to get a great square wave out of the controllers,
we see a steep rise, a gradual decline at the top of the wave, and then
a steep decline.  Adjusting the switches can change things, more from
the FC switch than from lim integral; often one can see a "peak" at the
beginning of the top of the wave, or a nonlinear top of the wave; generally
we have tried to adjust these features so they are removed. Some of
the adjustments can cause the motors to make loud, unhealthy noises, so
one should make the ajustments gradually and be prepared to back them
off reasonably quickly.

The altitude motor controller was adjusted July 2004: the FC switch was changed 
from 9 to B, limit integral was left at F, and DC Gain was left at 12:00.
The azimuth motor was found at FC=9, Lim integral=5, and DC Gain=2:30.

The motors were adjusted 2/4/2006. At this time, the rotator controller
exhibited some peculiarities, reported an error code occasionally referring
to encoder connector problems, but no specific problem was identified, and
eventually it started to function OK (I believe that this was (much) 
later identified as being due to bad cabling, which was replaced). 
The altitude motor was adjusted to
better contact the altitude drive, causing notable improvement in
slewing behaviour. Motor adjustments were as follows: rotator FC from D
to 9, limit integral F to F, gain to 45 degrees (highest before noise),
altitude B to F, limit integral F to F, gain to highest before noise,
azimuth FC 9 to 6, limit integral 5 to F, gain up.

In October 2006, the altitude and rotator motors were adjusted to attempt
to better mate them to the drive surfaces. The rotator was chattering
significantly. I was unable to get a very good mating of the rotator, but
it was much quieter after adjustment.

The azimuth motor started to develop some serious vibrations in fall 2006.
We had a several hour loud vibration in early November. After this, the
mating was inspected with the boroscope and didn't seem too bad. After 
consideration of the motor mounting, we decided to back way off on the
spring at the bottom of the motor, as it seemed to be overconstraining
the motor in the wrong direction. The soft spring was adjusted out until
moving the azimuth wouldn't move the motor, then back in until the motor
moved both directions when the telescope was rotated, then tightened 1/4
turn further in.

We looked again at motor tuning with Ed in late Jan 2013. Made some adjustments
based both on test waveforms and also by tuning while looking at encoder
feedback at various positions on sky. On azimuth, set to FC=7, LIM=E, ACGAIN=MIN
DCGAIN=tick 4 (all no change). On altitude, FC=C->E, LIM=E->1, AC=MIN, DCGAIN=2.25->3.0.
On rotator, FC=C, LIM=F, AC=MIN, DC=3.75 (all no change). Saw some poorer
behavior (according to encoders) in az at high az rate, tried to increase gain,
but that led to higher frequency vibration.

\subsection{Encoder Mountings / Adjustments}

Encoders are currently implemented on azimuth and altitude axes. At one
point, an encoder on the rotator was considered, but it is not currently
implemented.

The original telescope encoders are all manufactured by Computer Optical Products
of Chatsworth California.  Their phone number is (800) 340-0404.  Their
website is http://www.opticalencoder.com.  The encoder part numbers
are written on the sides.  All are model CP-850-HCE-128K (alt CP-850-HCE-128-F?) encoders.
The differences in the three axes all have to do with their mounting
configuration.  The azimuth encoder has a square flange mount and cable
coming from the back of the encoder.  The elevation has the cable coming
from the back of the encoder, but is manufactured without the flange.
The rotator encoder has the flange but a side-cable.  These are important
distinctions if replacements are ordered.

In spring 2016, the az encoder started spuriously drifting; we identified
the encoder as the cause by attaching az cable to alt encoder, and that looked
fine. Az encoder was
disabled (operation continued with motor positions)
and a new encoder was ordered, but we could not locate one with
exact same specs. Very similar encoder to ours available through Allied
Motion, we ordered CPH50HHD-120000-3/8-L-S-A-R-ET with a different number
of steps/rev (480000 instead of 512000); we ordered three in anticipation
of replacing altitude and having a spare. Due to adqequate performance
with motor tracking and lack of time, one was installed on az but not activated
until spring 2017 -- at which point it was found that it didn't behave,
giving different number of steps CW and CCW. Ed tracked this down to an
impedance issue and constructed a short pigtail to match impedancees, and
after that, behavior seemed fine. During the process, we switched back to
old encoder, and it actually seemed to behave fine, with no spurious drifting!
So we may have a lot of encoders on hand, in office in box on shelf.


\subsection{Limit Switches}

Limit switches are implemented on the azimuth and altitude axes. No
limit switches are implemented on the rotator which means that extra care
needs to be taken to avoid rotator cable wrap issues.

The limit switches throughout the 1-meter system are commonly available
microswitches.  Part numbers are printed on the switch housings and
generally the switches can be ordered through such vendors as DigiKey
and Mouser Electronics.  The smaller roller-rocker switches used in the
mirror covers are available at Radio Shack.

Most of the limits in the system are easy to understand with inspection.
The limits that are more complicated are the azimuth limits.  The azimuth
limits make use of two microswitches and a toggle switch to give a total
range of 540 degrees.  The azimuth limits work as follows: When the toggle
switch is down, the south microswitch is active; when the toggle switch
is up, the north microswitch is active.  When the telescope is at the
south limit switch, it is pointed almost east, i.e. at azimuth approximately
450. As the telescope is rotated counterclockwise (from above) from this
limit, it will trip the toggle switch at azimuth approximately 265, which
is just \textit{after} the telescope has passed the north limit switch.
Then the telescope gets another 360 degrees of rotation before it hits
the north limit switch at azimuth approximately -90.

The toggle switch is activated by a Delryn trigger mounted to an L-bracket
from the azimuth disk.  If the toggle switch fails, the azimuth limits
fail.  The toggle switch is a simple DPDT switch commonly available
at Radio Shack.  Also, in an emergency, Apache Point keeps a stock of
suitable toggle switches on hand.

There are also software limits implemented. In azimuth, these allow motion
315 degrees CCW and 140 degrees CW from the home sensor. Since the
home sensor is at about 305 degrees (N=0), this corresponds to azimuth
positions of -10 to 445 degrees.

\subsection{Motor control}

Motor control is implemented through the use of some Oregon MicroSystems motor
control cards in the telescope computer.

There are three control cards: 
\begin{enumerate}
\item controls telescope axes (az, alt, rotator) and three secondary actuators:
6 axes, two (alt/az) with encoders; 
\item controls guider stage (radial motion and focus), filter wheel rotation 
and encoders, and tertiary rotation: 4 axes, no encoders; 
\item controls dome rotation and shutter. 
\end{enumerate}

For the telescope and guider/tertiary control, we have used a OMS PC38-6E (with
encoder option) for the telescope and  OMS PC38-6 (without encoder option) for
the guider/tertiary control. For the dome control, we only use simple I/O
bits to turn motors on and off, and are using an OMS PC34 card. These all
have adjustable addresses based on jumpers on the cards (see the manuals 
for specifics); the telescope card is at 0x310, the dome card at 0x314,
and the guider card at 0x318.

These three cards are all ISA bus cards that run under DOS. Oregon Micro
Systems still exists (under Pro-Dex), but they have discontinued
construction of the PC3x series. At the end of 2011, we ordered a
ISA OMS PC48-6E (roughly \$1900, S/N 3454), which is supposed to be a drop-in
replacement. We installed it in February 2012 in place of the OMS
PC38-6E. Oregon Micro Systems says they will discontinue production of
these (PC4x) by the end of 2012. Documentation on the PC48 is available
via a \htmladdnormallink{PDF manual}{http://control1m.apo.nmsu.edu/1m/docs/Pc48mand.pdf}.

The PC48E did not immediately work completely: it worked for the XYZ
axes (azimuth, altitude, rotator), but not for the TUV axes (secondary
actuators).  The card had some voltage leakage that OMS attributed to the
encoder option (for UV axes), so OMS issued a RMA and repaired this issue
(from OMS: per EW 138 U55 was removed and pin 6,7,9,10 were clipped off
of U65). However, this did not resolve the issue. Turns out the motor
controllers for the secondary actuators (Anaheim Automation model MBL-500)
have 10 ohm pull-down resistors for step, direction, and enable, while
the OMS cards require pull-up resistors.  How it was working with the
PC38 is a bit of a mystery: perhaps those cards were specially modified,
or perhaps they had ``beneficial'' leakage.

With the PC38-6E we were getting intermittent "hangs" of the system
which we think may have been attributable to the card (but not
sure!). Nonetheless, it is probably a viable spare at some level,
especially for the guider PC38-6. It has been left in the computer,
configured at base address 0x300.

Because of the end-of-life issues, we also ordered another spare PC48-6E
(S/N 3489); this is an NMSU tagged item 430530. In June 2017, this spare
was put into the computer because we had pointing problems, but behavior
was identical, so no evidence for any issue with either card. The "spare"
was left in the computer, with the other PC48-6E in the drawer in the
office.

\subsection{Maintenance}

The primary maintenance required to keep the telescope in good operating
condition is routine inspection and cleaning of all drive surfaces on
the telescope.  This is critical to keep motor and encoder rollers from
slipping.  Use Kimwipes and acetone to clean the azimuth, elevation and
rotator drive surfaces at minimum once a month.  During moth season,
this may need to be done once every few days.

Periodic rotation of the telescope over its full range of motion is
advisable to insure smooth motion.

Periodic lubrication of the azimuth bearing may be advisable.

\section{Telescope Optical Systems}


\subsection{Mirror Mounting / Removal}

%To aluminize the mirrors, it is necessary to remove them from the
%telescope.  See section on secondary for information about removing it.
%I refer the reader to the 1-meter telescope
%blueprints that are stored in the bottom drawer of map cabinet in room
%110 of the Astronomy Building.

Removing any mirror requires at least two people and in some cases three.
For help please consult Mark Klaene at Apache Point Observatory.

\subsubsection{Tertiary mirror}

The tertiary is mounted on a rotating table to allow access to both Nasmyth 
ports. The rotation mechanism is mounted from behind the telescope. Extreme
care must be taken to remove the bolts that mount the tertiary to the 
table, and not to remove the bolts that mount the rotating mechanism to
the telescope, or else the latter will fall out of the back end of the
telescope!! The rotating mechanism mounts to a plate that is mounted to
the telescope stovepipe from the primary mirror cell.  See later section
on the tertiary.

When we removed the tertiary May 2013, we removed the bolts on the inner
circle, and lifted the tertiary mount off of the table; it has a pin, and
needs to be lifted straight up. We then left the entire rotary table
mechanism in the cell. We did this at a vertical orientation, but this
did then require the we lift the tertiary out over the primary; doing it
at an angle might be easier.  Note, of course, we took the baffle off
first.

OBSOLETE: removing the OLD tertiary:
\begin{itemize}
     \item This is easiest done when the telescope is horizontal.
     \item Remove the cylindrical baffle from the tertiary mirror.  This is accomplished by using
     the Allen-head bolts that are located around the circumference of the baffle on the side
     closest to the primary.  Once the Allen bolts are removed, the baffling is only held in
     place by a pressure fit.
     \item There are six bolts that hold the tertiary mount to the telescope. They are located in
     the base of the mount.  The bolts hold the tertiary mount to a tube that runs through
     the center of the primary mirror cell.
     \item Before unbolting the tertiary mount, make sure that someone is helping to hold the
     mirror in position so that it doesn't slip.  
     \item Remove the bolts.
     \item The tertiary is held captive by two alignment pins.  Gently, but firmly pull the tertiary
     mount toward the secondary mirror.  
     \item When free, carefully take it away from the telescope.  At this point, the mirror may be
     taken for washing or aluminizing.
\end{itemize}

Old reinstallation:

\begin{enumerate}

  \item Attach tertiary tripod to tertiary

  \item Attach tertiary support to tertiary tripod

  \item For old system, feed threaded rods through tripod and adjust

  \item However, SUGGEST SYSTEM REDESIGN.
\end{enumerate}

\noindent{OLD Tertiary adjustment (OBSOLETE):}

To adjust the tertiary, remove the cylindrical baffle around the mirror.
The tertiary is adjusted using  three nuts on threaded rod behind the
mirror itself.  The three threaded rods are mounted symmetrically around
a ball-pivot.  Adjusting the nuts causes the mirror to pivot around it's
geometric center. It is possible to piston the mirror by making sure 
the central bolt is loose, then adjusting the nuts on all three of
the threaded rods, iteratively; it make take a little while to move
in piston, and the mirror will need to be readjusted in tilt, but it
is possible to do in a moderate amount of time.


\subsubsection{Primary mirror}

We removed the primary in February 2000 for realuminization. We removed
the mirrors for repolishing in 2002. We removed the primary again
in May 2013 for realuminzation/overcoating; this was done at ATOC in
Albuquerque where they put down aluminum plus 0.5wave (680nm) SiO2.

See photo documentation at
\htmladdnormallink{http://control1m.apo.nmsu.edu/1m/docs/photos/}{http://control1m.apo.nmsu.edu/1m/docs/photos/}.
There is some older photo-documentation of this process in the Photo Album 
in the control room.  On the day of removal, make sure that arrangements
have been made for a suitable crane to hoist the primary mirror cell.
Make sure there are sufficient personnel to operate the crane, and
handle the mirror   this takes a minimum of three people.  You will
need straps to attach the mirror cell to the crane.  Mark Klaene at
Apache Point Observatory is experienced with handling this mirror and
should be consulted regarding questions.  Also, Mark has the straps
for handling the mirror cell.  Aluminization is a process that takes
between 3-5 days.  The typical schedule should be as follows:  On day 1,
remove the mirror cell and transport to Sunspot.  Strip the old coating
that day if possible and prepare for aluminizing.  Day 2, aluminize.
Day 3, check the coating and return to the telescope.  Extra days will
be required if the first coating turns out poor.

\begin{itemize}
     \item First, remove the tertiary as described above.
     \item Remove the elevation drive surface alignment pin.  This is a
     bolt that goes through the semi-circular elevation drive surface and
     pushes against the mirror cell.  The bolt is in place to make sure
     the elevation drive surface is even and not cocked.  Measure the
     separation before removing, so you can reproduce. As of 6/02, we
 measured 1.4 inches between the outer edge of the aluminum altitude
 drive surface (not the yellow part) and the inner raised yellow surface
 (where the eye bolts are screwed in).

     \item Detach the mirror covers, their mounting fixtures and limit
     switches.  The mirror cover panels are easily removed.  First,
     loosen the set screws that attach them to the motor.  Unbolt the
     idler shafts from the mirror cell.  Once done, the panels are free.
     It is critically important to note where each motor came from and
     where each panel belongs before removal.  The mirror covers must be
     reassembled exactly as they were or they will not function properly.
     Before proceeding with motors and limit switches, take careful
     note of the limit switch wiring.  Only a couple of wires need to be
     clipped on the inside of the mirror cell to free the internal limit
     switch wiring harness from the mirror cell.  Unbolt and remove the
     internal limit switch harness.  The motor and idler mounts must
     be unbolted.  The external limit switch mounts must be unbolted.
     The motors and external limit switch mounts may be strapped to
     the telescope frame to get them out of the way of the mirror cell.
     This is shown in the photo album.

     \item Protect the Elevation drive surface.  We used some cardboard
     taped around the wheel to keep it from getting banged or scuffed.

     \item Tie the telescope down.  It will be severely unbalanced when
     the mirror cell is removed.  The top end   with the secondary   will
     want to drop toward the floor.  Therefore, the strap should go near
     the mirror cell.  This is shown in the photo album.
  
     \item It may be a good idea to remove the altitude limit switch/hard
     stop as it may be damaged by putting the full unbalanced weight of
     the telescope on it. However, if this is done, the weight will rest
     on the altitude motor, which is also a bad idea! Need to figure out
     a good solution to holding the unbalanced telescope in place.

     \item Screw four eyehooks into the mirror cell   two on each side
     in the tapped holes provided.  Facing the primary, these tapped
     holes are on the left and right sides of the cell in the reinforced
     sections of the mirror cell.  One of these eyehooks is permanently
     in place now and taped to the side of the mirror cell; it cannot
     be removed once the mirror cell is bolted in place. \textbf{Note
     that you cannot use a shackle on the side of the mirror where the
     altitude drive is, as there is not enough clearance; 
     remember this especially when reinstalling mirror!} 

     \item Attach the lifting straps to the eyehooks as directed by Mark
     Klaene or other crane- experienced personnel.

     \item At this point, the person directing crane operations takes
     over for the most part.  The crane will be lowered through the dome
     slit and attached to the handling straps.  Slack will be taken up
     on the straps.

     \item Remove the bolts that hold the mirror cell to the telescope
     frame.  There are twelve bolts total.  At this point the mirror
     will only be held on by alignment pins.

     \item With help and watching to make sure that you don't bump
     anything, carefully push the mirror cell away from the telescope
     structure.

     \item The mirror cell should be free at this point.  It may then be
     lowered to the floor for service, if necessary, or lifted through
     the dome slit, then lowered to the ground for transport to the
     aluminizing chamber at Sunspot.  The last time the mirror was
     aluminized, it was lifted through the slit.

     \item The original mirror covers, located under the brown table on
     the south side of the 1- meter enclosure, can be bolted to the top
     of the mirror cell to protect the mirror during transport.

     \item The mirror is removed from the cell as follows:  Remove the
     mirror cover central collar.  Remove the teal-colored retaining
     ring directly above the primary mirror.  At this point, the mirror
     itself may be lifted out of the cell. Before doing this, the mirror
     band should be attached. We have found that it is much better to   
     lift the mirror out of the cell by hand rather than trying to use
     a lift attached to the band.

     \item Returning the mirror to the telescope is the reverse of the
     above procedure.  The keys are to have sufficient personnel, take
     your time and don't rush the operation.

\end{itemize}

Notes on reinstalling optics from June 2002/October 2003:

\begin{itemize}
\item Primary:
\begin{enumerate}
  \item Return primary support to previous state: remove extra spacers from the
  feet, the six washers, and the longer bolts, and replace with shorter bolts
  and a single washer (these are in an envelope marked as such).

  \item Install central cone into cell.
  
  \item Primary mirror into cell, install blue retaining ring, 3 black foam 
  retaining blocks, mirror cover lips
  
  \item Primary then ready to go onto telescope. Do not use shackles on the
  side of the mirror that will be against the altitude drive; there is not
  enough clearance! Use two longer bolts to help to guide cell onto telescope

  \item Remove eye bolts from cell, except for one that is inaccesible: tape 
  that one down.

  \item Reinstall spacer bolt for altitude drive.

  \item Reinstall altitude limit switch.

\end{enumerate}

\item Final touches:
\begin{enumerate}

  \item Install elevation drive spacer bolt.

  \item Remove cardboard from elevation drive.

  \item CAREFULLY check balance 

  \item Install mirror covers.
\end{enumerate}
\end{itemize}

\subsubsection{Secondary mirror mounting and actuators}

The top end of the telescope is mounted to the telescope using an eight
strut system, with turnbuckles on each of the eight struts.

The secondary mirror is mounted to a tripod structure. A shaft attaches
to the tripod and goes through a linear bearing on the telescope top end.
The three legs of the tripod are attached to three secondary actuators 
using flex couplers; the actuators allow for focus and tilt adjustment of the
secondary mirror.

In Feb 2009, something happened which led the flex couplers to all fail,
causing the mirror to tilt and become unfunctional. At that time, Tom
Harrison, Charlie Park (Engineering), and Yuping (MTEC) investigated the
situation and found several issues: a destroyed linear bearing, some
rusting on the shaft, some cracks in the turnbuckles, and a poor design
of the struts used to attach the secondary cage to the top end. At that
time, they remade a new set of struts (putting in a 90 degree rotation
between the holes at each end), got new turnbuckles, got a new shaft,
and got a new linear bearing. They also improved the safety system,
putting a better stop at the end of the main shaft, and installing some
straps on the tripod that will catch the mirror if the couplers fail.

They found that the ideal amount to insert the actuators into the couplers
was not to insert them as much as they had been before (this allows more
tilt). As a result of the new mounting, however, the previous focal position
placed the acuator shafts very close to the limit switches. As a result,
the limits switches needed to be moved back a bit. 

The bearing that holds the secondary shaft is fragile and may be destroyed
if the shaft is removed without putting something in its place. Charlie
designed a tool to allow the shaft to be safely removed. \textbf{This
should always be used if the secondary is removed.}

Periodically, the secondary actuators should be checked to make sure
they are not binding.  The primary cause of binding is if one actuator
locks up for some reason and stops moving, causing the shaft to exceed
its tilt limit.  The shafts should be periodically cleaned of dirt and
dust that builds up in the grooves and seems to cause the binding. The
actuators should be mildly lubricated.

The actuators are connected to the secondary with three flex couplers.
\textbf{Before decoupling a actuator or coupler from the secondary, the
secondary should be tied to prevent it from ``falling''; this is especially
essential for the top actuator}.
One should take care when decoupling the actuators from the secondary:
if all three couplers are removed, the secondary will lose all support except for
the central attachment, and so care must be taken to let it down gently.
If all three are removed, it becomes much more difficult to attach them
again, becuase the secondary must be held up while the couplers are
connected. It is recommended to just remove one at a time if possible.
Care must be taken to make sure the couplers are securely attached to
the secondary!

There should be spare secondary actuators; we do occasionally have failures.
Additional replacements may be ordered from \htmladdnormallink{
Eastern Air Devices}{http://www.eadmotors.com}.
The part number is: LA23GCKJ-A213. 
%The flex couplers are made by Huco; the part is a 
%htmladdnormallink{Uni-Lat}{http://www.huco.com/html/h_unilat.htm}.

The secondary actuators are controlled by the PC38 in the TOCC computer,
as the T, U, and V axes of that controller. When looking at the primary
mirror, the U actuator is the one on top, the T actuator is at the lower
left, and the V actuator is at the lower right. The T actuator is hooked
up to cable P113B, the U to P114B, and the V to P115B. There is a separate
cable that connects to the limits switches that must also be connected.

Measuring motion of the actuators gives .625 inches per 10000 steps, or
.001587 mm/step. The actuators are located at a radius of 6.385 inches
(162.19 mm) from the central shaft .
%    .625 inches = 10000 steps
%    5.631 shaft to shaft + 1.754/2 - .245/2

\subsubsection{Secondary mirror reinstallation}

\begin{itemize}
\item Secondary:
\begin{enumerate}
  \item Attach secondary tripod to secondary. 

  \item Attach secondary post to secondary tripod. 

  \item Insert secondary post and install cotter pin (or better replacement)

  \item Attach couplers to secondary actuators.

  \item Note actuator position before June 2002 repolishing was around -2000.
\end{enumerate}
\end{itemize}

\subsection{Aluminization }

Aluminizing the mirror must be done by a qualified person.
Currently, Mark Klaene and Jon Davis are both qualified to help with
this procedure.  There is a custom-made band to hold the mirror in
the Sunspot aluminizing chamber.  This band is located in the 1m crate,
which is usually kept in the base of the 3.5-meter enclosure.

Aluminization at Sunspot Feb 2000.

Aluminization after repolishing, June 2002?

Secondary and tertiary aluminized and overcoated by ATOC (fomerly Optical 
Surface Technology) in Albuqueque, 2007? Primary?

All mirrors removed May 2013. All coated and overcoated at ATOC: secondary
and tertiary with Al + SiO2 (half wave at 500nm), primary Al + SiO2 (half wave
at 680nm); the different overcoat thicknesses were to try to provide overall
near-constant throughput as a f(wavelength) and to maintain decent UV
throughput.

\subsection{Alignment}

\subsubsection{Secondary Adjusment}

Put spot on secondary using print overlay of mirror, kept in secondary mirror
crate.
Collimating telescope in tertiary mount. Center and tilt secondary.

\subsubsection{Tertiary Adjusment}

Collimating telescope in Nasmyth mount. Adjust tertiary.

There is fixture for mounting the APO collimating telescope at the
Nasmyth mount on the back of the filter wheel. Using this, one wants
to adjust the tertiary such that the collimating telescope can be
focussed on the center of the secondary, and also that it can be
focussed back on itself. Both things are changed by either tilt or
piston of the tertiary, so it is an iterative process to get both
things to focus; at a given piston, adjust the tilt such that the
retroreflection is aligned, then see if the center of the secondary
is centered; if not, repiston the tertiary and try again. Hopefully,
you will see if the change in piston made the agreement better or
worse, and you can converge on a proper alignment of the tertiary.

Another method we have used for the tertiary is to map out the image
quality in the focal plane (using the guider on its stage), and
determine the location of the best images, and use this as the optimal
center of the field. To get this location onto the science camera, you
tilt the tertiary. The method for doing this has been to use a laser
pointer mounted at the top of the telescope. Adjust the laser so that
it is pointed at the guider pickoff at the location of the best images.
Then adjust the tertiary until the laser spot is pointed at the
science camera.

Three components are needed to set up the laser for adjustment.  The first
is a flexible flashlight-holder.  The holder has a clip on one end that
may be attached to the secondary struts.  The second component is the
laser pointer itself.  Finally, you need two small clamps.  One small
clamp may be used both to hold the laser pointer in the holder and keep
the laser pointer powered on.  The second clamp may be used to help hold
the flashlight holder to the secondary mounts.  CAUTION:  When the laser
pointer is powered on, take great care not to look into the beam.

\subsubsection{Primary Adjusment}

On sky, adjust primary to minimize coma.

\subsubsection{Tertiary Adjusment}

On sky, minimize astigmatism with tertiary rotation?

\subsubsection{Puntino Shack Hartman - final adjustment}

In 2011, we purchased a Puntino Shack-Hartmann sensor with associated
software from Spot Optics. To reduce the cost, we provided an existing
SBIG camera (ST8) for use with the system. The SBIG ST8 is controlled
by a special version of the software that Spot Optics provided us with.

The Puntino needs a reference frame to compare the star frames with.
This is obtained by screwing in the small LED that came with Puntino
(the instrument must be off the telescope to do this, and the 
aluminum mounting plate taken off -- mark the orientation!). One then takes
a reference frame with the software.

The Puntino mounts at NA1 to an aluminum plate that is clamped to the
filter wheel. To try to match the back focal distance of the CCD camera
(and as a result allow the guider to be in focus for acquisition and guiding),
some extra spacers (not currently known how much) should be inserted.
If mounted in a repeatable orientation (TOP towards CCD controller) then
one can use repeatable commanding for offsetting the telescope and tilting
the secondary to produce repeatable results on the SH.

Before using, the filter wheel should be moved to an OPEN positions!

See Appendix \ref{sect:puntino} for details of different Puntino runs.

\subsection{Tertiary and tertiary control}

In July 2008, we obtained a new tertiary mount from ACE. The new tertiary
is on a rotating table to allow use of both Nasmyth ports. In addition,
the new tertiary mount was designed to allow easier adjustment of the
mirror alignment, although this is still manual, not motorized adjustment.

The rotation mechanism must be inserted from the back end of the telescope.
It mounts to a plate that is bolted on from the front side of the telescope
to the flange on the stovepipe that extends up through the primary. Installing
or removing the rotation mechanism requires two people, because the screws
thread in from the front, but the mechanism is inserted from the back. 
The tertiary itself is bolted to the rotation table from the front.
Extra care must be taken when removing the tertiary to remove the correct
bolts: if the bolts to the rotation mechanism were removed, the mechanism
would fall out of the back end of the telescope!  (I think it must not
be possible, or at least very undesireable, to pull the entire thing out from 
the top.)

The tertiary rotation is controlled by a stepper motor which drives a 
worm gear that is coupled to the rotation table. To obtain stable 
positioning of the tertiary, there is also a brake that clamps the
table in place. The brake is controlled by a 24V DC motor. The brake
MUST be released before the table is rotated, and should be activated
while motor power is still being applied after a move. There are limit
switches that are triggered when the brake is in the open or closed
position.

There are two home switches near the position of the two ports. We
have obtained reasonably repeatable positioning by moving to a home
switch, then moving past the home switch in the same direction. Then
reverse direction until home is hit, and reset the position counter. Then
the tertiary can be moved, ideally in a single commanded move, to the
desired position, as determined by the optical alignment procedure.

Cable connection from tertiary: (but note, we have two cables coming out!
maybe the motor power one was originally in the big connector, and we had
to break it out for some reason???)

\begin{itemize}
  \item Motor driver: AMP 4-pin breakout cable to ACE motor controller box
    \begin{itemize}
       \item A+ phase: Pin 1 on tertiary to pin 1 on ACE control box
       \item A- phase: Pin 2 on tertiary to pin 2 on ACE control box
       \item B+ phase: Pin 3 on tertiary to pin 3 on ACE control box
       \item B- phase: Pin 4 on tertiary to pin 4 on ACE control box
       \item Inside ACE box, these go to the APM 3540M motor controller
    \end{itemize}
  \item Encoder: DB9 breakout cable direct to IO38 breakout box
    \begin{itemize} 
      \item Ground: Pin 12 on tertiary to DB9 pin 1 on IO38 breakout (green+white)
      \item Phase A+: Pin 6 on tertiary to DB9 pin 3 on IO38 breakout (orange)
      \item Phase A-: Pin 7 on tertiary to DB9 pin 4 on IO38 breakout (orange+white)
      \item Phase B+: Pin 8 on tertiary to DB9 pin 7 on IO38 breakout (brown)
      \item Phase B-: Pin 9 on tertiary to DB9 pin 9 on IO38 breakout (brown+white)
      \item Index+: Pin 10 on tertiary to DB9 pin 2 on IO38 breakout (blue)
      \item Index-: Pin 11 on tertiary to DB9 pin 8 on IO38 breakout (blue+white)
      \item +5V: Pin 5 on tertiary to DB9 pin 5 on IO38 breakout (green)
    \end{itemize}
  \item Brake and home switches: AMP 9-pin breakout cable to ACE motor controller box
    \begin{itemize}
      \item  Brake motor +: Pin 13 on tertiary to pin 2 on ACE 9-pin
      \item  Brake motor -: Pin 14 on tertiary to pin 3 on ACE 9-pin
      \item  Brake engaged: Pin 15 on tertiary to pin 4 on ACE 9-pin
      \item  Brake disengaged: Pin 16 on tertiary to pin 5 on ACE 9-pin
      \item  Brake common: Pin 17 on tertiary to pin 6 on ACE 9-pin
      \item  Port 1 home: Pin 18 on tertiary to pin 7 on ACE 9-pin
      \item  Port 2 home: Pin 19 on tertiary to pin 8 on ACE 9-pin
      \item  Port common: Pin 20 and 21 on tertiary to pin 9 on ACE 9-pin
    \end{itemize}
\end{itemize}

Cable connections inside of ACE box
\begin{itemize}
   \item Motor drive: connects motor driver output to motor
   \item Motor control: connect motor driver input to DB15
     \begin{itemize}
       \item +5v: to pin 1 on DB15
       \item step: to pin 9 on DB15
       \item direction: to pin 10 on DB15
       \item enable: to pin 12 on DB15
     \end{itemize}
   \item Home switches
     \begin{itemize}
       \item Connect NA1 home (9-pin 7 from tertiary) to pin 2 on DB15
       \item Connect NA2 home (9-pin 8 from tertiary) to pin 2 on DB15 for guider axis?
     \end{itemize}
   \item Brake logic
     \begin{itemize}
       \item Brake on (9-pin 4 from tertiary) to pin 15 on DB15
       \item Brake off (9-pin 5 from tertiary) to pin 14 on DB15
       \item Brake common (9-pin 6 from tertiary) to pin ?? on DB15
       \item Brake engage:  OPTO 22 output to pin 13 and pin 14 on 9-pin from tertiary. DB15 pin 8 to
             OPTO 22 input.
       \item Brake disengage: OPTO 22 output to pin 13 and pin 14 on 9-pin from tertiary. DB15 pin 7 to
             OPTO 22 input.
     \end{itemize}
   \item Other
     \begin{itemize}
       \item Connect ground (from tertiary) to pin 6 on DB15
     \end{itemize}
    
\end{itemize}
Cable connection from ACE box to IO38 box
\begin{itemize}
  \item DB15 to T axis DB9
    \begin{itemize}
      \item  ground: DB9 pin 1 to DB15 pin 6 
      \item  step: DB9 pin 2 to DB15 pin 9 
      \item  direction: DB9 pin 3 to DB15 pin 10
      \item  enable: DB9 pin 4 to pin 12 
      \item  +5V: DB9 pin 5 to DB15 pin 1
      \item  home: DB9 pin 9 to DB15 pin 2
    \end{itemize}
  \item DB15 to IO DB25
    \begin{itemize}
      \item Brake on: pin 15 on DB15 to pin 6 on DB25 (I/O bit 6)
      \item Brake off: pin 14 on DB15 to pin 19 on DB25 (I/O bit 7)
      \item Engage brake: pin 8 on DB15 to pin 10 on DB25 (I/O bit 10)
      \item Disengage brake: pin 7 on DB15 to pin 22 on DB25 (I/O bit 11)
    \end{itemize}
\end{itemize}

\subsection{Baffles}

Alt-az telescope can present challenges for proper baffling. The 1m
has two baffles located in the Nasmyth tube, a baffle around the
tertiary which has components both around the tertiary and up towards
the secondary, and a baffle around the secondary, which has both
a vertical and horizontal component.

As of July 2004, the Nasmyth baffles are located 2 inches and 7 3/4 inches
from the rotator mounting surface (in the direction of the tertiary). 
These have a 4.5 inch diameter central hole.  

\subsection{Mirror Cleaning}

The original notes are in The Blue Notebook under the "Optics" section.
For questions, contact Mark Klaene at Apache Point Observatory.
To obtain CO2 cylinders, talk to Mark or Norman Blythe.  CO2 cylinders
are obtained from Valley Welders in Alamogordo.  One cylinder typically
lasts for about five cleanings.  A cleaning session for the primary,
secondary and tertiary typically takes about 30 minutes.

To clean the mirror with CO2:
\begin{itemize}
     \item Connect the CO2 mirror cleaning tube   typically located
     under the brown table against the South wall in the 1-meter dome
     to the CO2 bottle.  Make sure there is a washer in the connector
     of the tube.  There is a washer for the cleaning tube in the small
     drawers between the two tool chests in the 1-meter enclosure.
     Fittings, etc. may be found on the observing floor of the 3.5-meter.

     \item Open the CO2 cylinder for a few seconds to blow the tube
     clear of particulates.  It sometimes is good to blow the CO2 at
     some plain glass to see that you are getting a good clean stream.

     \item Humidity needs to be below 80\% and should be below 60%.
     Humidity needs to be above 15\%.  In conditions this dry, there is a
     danger of arcing.  Ground the bottle using one of the straps from
     the observing floor of the 3.5-meter or some grounding tape   Jon
     Davis has some in his office.

     \item Hold the tube at a 30-40 degree angle, 6-7 inches from the
     mirror.  Move in sweeping motions   slow enough to get the dust off;
     but fast enough that you don't get condensation.

     \item Should only keep the CO2 bottle open 3-4 minutes at a time
     to prevent freeze-up.

     \item Wear eye-shielding.  Glasses are not sufficient protection.
     Wear gloves.  Eye shields are stored in the cryogenic equipment
     cabinet on the observing floor of the 3.5- meter enclosure.
     There is a pair of gloves in the larger of the two red tool boxes
     in the 1-meter enclosure.

\end{itemize}

To wash the mirrors (old procedure?):
\begin{itemize}
     \item Place a plastic tarp over the azimuth wheel to keep water
     out of the motors and cables that come through the azimuth cone.
     There is a plastic folded plastic tarp located in the big cabinet
     in the 1-meter control room at Apache Point.

     \item Use medical grade cotton or lint-free wipes.  Use only alcohol
     and distilled water on the mirror.

     \item To start, apply alcohol to a cotton ball or wad, make one
     short, gentle, sweep on the mirror, then throw the cotton away.
     Do not reuse the cotton once it's contacted the mirror.  Dust and
     grit picked up from the first swipe can etch the mirror surface.

     \item Continue the procedure with the cotton until the entire mirror
     surface has been cleaned.

     \item Rinse generously with distilled water.  Mark Klaene has water
     jugs with spray wands that can be used for this purpose.

     \item Let the mirror air dry.
\end{itemize}


\section{Telescope auxiliary systems}

\subsection{Guider}

\subsection{Filter Wheel}

The Filter Wheel and guider were both manufactured by Astronomical
Consultants and Equipment of Tucson, Arizona.  Their phone number is:
(520) 579-0698.  There is a manual written by the Filter Wheel and
Guider's designer, Dr. Peter Mack.

One issue with the guider is that, since it is on the rotator, there
are cable wrap issues. Since we do not have an ideal solution, the
cables may occasionally go bad at the connectors. All of the cables
use AMP connectors, which use crimp pins. These can be ordered from
a variety of electronics sources, e.g. Digi-key corp: the AMP part
numbers for the pins are 1-66103-7 and 1-66105-8. There is a crimp
tool in the 1m tool box that allows one to attach wires securely, and
another tool for removing pins from their sockets.

The Filter Wheel Housing consists of a removable/interchangeable wheel,
four small switches that encode position information, a detent lock that
holds the filter in a given position   this detent lock is driven by
a motor and has three separate switches to sense its position   and a
rotation motor.  All of the switches are simple microswitches that may
be ordered from an electronics catalog.  Details about the operation of
most of these components is in the Guider manual.

There are replacements for both of the motors of the Filter Wheel on
the top shelf in the 1- meter control room at Apache Point.

The most complicated part of the Filter Wheel has to do with the Detent
switches.  Two of the switches surround the Detent motor cam shaft.
Of these, the lower most senses when the Detent is completely disengaged.
The upper most senses when the Detent is up, but not necessarily engaged
in the filter wheel.  The final switch is on the opposite side of the
filter wheel, near the Detent spring.  This switch is activated when
the Detent fully engages in the filter wheel.  When this final switch
is engaged, the filter wheel is said to be "home."

The filter wheel operates as follows:  The detent is disengaged.
The filter wheel is commanded to run most of the way to the final
position.  The detent is run in.  At this point, the detent is not
necessarily engaged.  A "run-until-home" command is executed.  When the
detent engages, the filter is home.  The four "encoder" switches are
then read to determine whether the filter moved to the correct position.
Note, unless the filter wheel is "home" the encoder switches will not
necessarily read the correct position.  One can get a combination of
switches when the detent is not engaged that will give a false reading.
How the filter encoder switches work is described in detail in the
Guider Manual.

To troubleshoot the filter wheel box, it may be run while detached from
the telescope.  To do this, remove the top of the filter wheel and take
the filter wheel out as described in the "Filters" section of this manual.
The main body of the filter wheel box may be removed from its cover by
removing the bolts around the perimeter of the filter wheel.  The cover
stays mounted to the guider module on the telescope.  The filter wheel
may be reinstalled without the cover plate.  It slides in as described in
the "Filters" section.  Place the top on the filter wheel box.  Once you
plug the filter wheel back into the control cables, you will be able
to run the filter wheel as though it were on the telescope.  However,
you will be able to see the operation of all the switches and motors.

There are two filter wheels that may be used.  One is a ten-position wheel
for 2"X2" filters.  The other is a six-position wheel for 3"X3" filters.

We had problems with the triggering of the "detent out" switch in the summer
of 2011. This was eventually fixed by adjusting an allen screw in the box
that triggered the switch. In the process of diagnosing this, Ed Leon 
documented how the circuitry worked; his diagram can be found in
\htmladdnormallink{FilterWheelDetent.pdf}{../FilterWheelDetent.pdf}.

\subsubsection{Filters}

Filters for the 1-meter are located on the Filter Shelf in the 1-meter
control room.  This filter shelf is the lower most shelf above the
computer "Loki."  There are three sets of filters available for use on
the 1-meter:  1) A Johnson broadband set, 2) A Gunn broadband set, and 3)
A narrowband set owned by Dr. Ren� Walterbos.  The Johnson and Gunn sets
are NMSU property and may be used freely in the 1-meter.  The Walterbos
narrowband filters may be used in the 1-meter as long as he does not
have a priority project on another telescope.  Dr.  Walterbos' filters
are stored in a filter box.  The combination to this box is 656.

An SDSS set of filters was purchased from Custom Scientific.

The original UBVRI set of filters was damaged spring 2013 and had shown
degradation for some time before that. A new set of UBVRI filters was
purchased at that time from Astrodon.

Great care should always be taken in the handling of filters.  Only hold
them by the edge.  Remember that filters are very fragile. Do not drop
them. Avoid getting fingerprints on the filters.  The filters will need
periodic cleaning even with careful handling.  Use canned air to gently
blow filters free of dust.  Dust blowing is the most commonly required
cleaning action.  In the event that something more difficult to clean off
gets on the filter, there is a filter cleaning kit on the Filter shelf in
the 1-meter control room.  The kit contains solvents for gentle cleaning
of the filters.  Use only lint free wipes or medical grade cotton to
wipe the filters.

There is hardware for installing filters in the filter wheel on the
Filter shelf in the 1-meter control room.

To install/remove filters:
\begin{itemize}
     \item Make sure the telescope is stowed with the filter wheel upright (i.e. the handle on the
     filter wheel box is upright.)
     \item On the spec computer issue a DETENT 0 command to disengage the detent.
     \item Loosen the two Allen screws on the filter lid and remove the lid.
     \item Using the finger holes provided, carefully lift the filter wheel from the filter wheel box
     and place the wheel on a clean work surface.
     \item Three small screws and teflon washers hold the filters in place.  To remove a filter,
     remove the three screws and carefully push the filter out of the slot.  To push the filter
     out of the slot, either use clean cotton gloves (the 3.5-meter has a supply of these in
     their filter cleaning supplies) or a lint free wipe.  Do not touch the filter directly.
     \item Insert a new filter into the slot.  Replace the three screws and teflon washers.  In the
     case of thick filters (such as the Walterbos narrow band set) spacers may be used to
     keep the screws from being screwed down too tightly.
     \item Note:  Do not screw the screws down too tightly on any filter.  "Finger tight" is
     usually sufficient to hold a filter in place.  Too much pressure and you can crack a
     filter.
     \item Using the finger holes on the filter wheel, replace the wheel into the box.  Make sure
     the groove on the filter wheel slips into the wheels in the box.
     The filter wheel goes in with the detent notches towards the
     tertiary; i.e., the side with the dots that mark the filter 
     positions towards the tertiary.
     \item Carefully replace the lid.
     \item Before re-engaging the detent, it's a good idea to check and make sure that the filter
     wheel was reseated properly.  With telescope motor powers off, carefully turn the
     handle on the filter wheel drive motor (the motor with a handle) and make sure that
     the wheel does not bind or give you any resistance.  If you listen carefully, you should
     hear the filter wheel activating the detent switches as it turns, also.
     \item Restore motor power and issue a DETENT 1 on the Spec computer to move the
     Detent back in.
     \item Issue a FILTINIT to re-init the filter wheel.
     \item After installing new filters, you must update the file /control1m/tcomm/filters.dat, to give the name of the filter and the focus offset relative to the
BVRI filter set. The file /control1m/tcomm/filters.master is supposed to have this
information for all possible filters, so you can just extract the relevant
line.
\end{itemize}

\subsection{Guider}

As noted above, the Guider was built by Astronomical Consultants and
Equipment of Tucson, Arizona.  They can help with obtaining spare motors
or other parts that might have failed.  The cabling is documented in
the Filter Wheel and Guider manual as noted above.   Note this box uses
motor controllers Applied Motion Products 3540M

%The guide camera
%is a Spectrasource Teleris 2 camera built by Spectrasource Instruments.
%Their phone number is: (818) 707-2655.

The guider module is a self-contained unit that may be entirely removed
from the telescope.  This is useful when diagnosing problems.  There are
two stages   the first is a radial (X) stage for positioning the camera
in the telescope field of view.  The second is a focus (Y) stage that
moves the pick-off mirror further from and closer to the guide camera.
Note the focus mirror does have the effect of acting like a radial stage.

One issue with the guider is that it is wired such that the limit
switches are recorder tripped if the limit circuit is \textit{open}. This
means that in the case of connection/wire failure, the limits will read
open and the stage will not move. So if the stage is not moving, it's
a good idea to check the limit circuitry/wiring.

Another issue is that occasionally on the focus stage, the gears become
disengaged. 

Some notes about the guider:
\begin{itemize}
     \item For 1,000,000 Y-steps you need to move the X-stage -40,000 steps to stay in the same
     place.
     \item +3,000,000 Y-steps corresponds to -500 units of secondary travel.
     \item 7500 X-steps is approximately 10 arcsec on the sky.
     \item To get the guider to the center of the PI field of view, need to set X=-270,000 and
     Y=-2,547,500
     \item There are a handful of other guider notes in the Big Blue notebook.
\end{itemize}

\subsubsection{Filter/guider wiring redesign, July 08}

Six cables from telescope:
\begin{itemize}
\item Radial stage motor control 4-pin to ACE box

\item Focus stage motor control 4-pin to ACE box

\item Filter rotation 9-pin to 4-pin in ACE box

\item Limits 9-pin: 3 pins (1,2,5) to radial stage DB15, 3 pins (3,4,5) to focus stage DB15

\item Filter encoder 9-pin: pin 1 to focus DB15, pins 2,3,4,5,6 to PC38 IO DB25.

\item Filter detent motor: 9-pin to ACE box
\end{itemize}


\subsection{Mirror Covers}

MIRROR COVERS ARE CURRENTLY NOT IMPLEMENTED. The old system, described below,
was too unreliable.

Mechanical drawings of the mirror covers along with electrical drawings
are in The Blue Notebook.   There are six mirror cover motors.  They are
Pittman motors, part number GM9413-5 and are available from Automation
Solutions in Englewood, Colorado; phone number (303) 792-5518.

When operating the mirror covers, the telescope must always be near
vertical.  To open the mirror covers manually:  There are three switches
on the TCS box on the left-hand side of the rack in the dome.  Flip the
left-most to manual control.  Flip the middle switch to "Set 1" Hold the
"Open" switch up until the covers stop moving.

To close the mirror covers manually:  As above, make sure the left-most
switch is in the manual control position.  Flip the middle switch to
"Set 2."  Hold the "Close" switch down until the covers stop moving.

The "Set1"/"Set 2" switches refer to which set of limit switches (i.e. the
open or close set) are active and determine the order in which power is
applied to the mirror cover motors.  "Set 1" is for "open" and "Set 2"
is for "close."

To move the mirror covers via computer, make sure to leave the left-most
switch in "Computer Control."

There are two possible weak points in the mirror cover design.  1)  The
motors themselves may be right at the torque limit.  New motors should be
investigated and identified to replace those that are currently in place.
2)  The mirror covers have a rigid coupling between the motor and the
covers themselves.  Perhaps this should be replaced with either a flex
coupling or a slip coupling of some sort.

Occasionally, the bottom two mirror covers don't seem to close completely.
This is easily remedied by loosening the set screws in the motor
shaft-couplings and gently pushing the covers all the way closed.
Re-tighten the set screws.  This should be checked about every two weeks.

The most common failure mode with the mirror covers is damage to the
gear head due either to the torque problem mentioned above or mirror
cover collision.  If the mirror covers fail, loosen the set screw
that holds the mirror cover to the motor on each mirror cover panel.
At this point, the mirror cover panels will be free to swing open.
Open the mirror covers in the correct order   A panels first, B second
and C third.  Re-tighten the set screws that hold the mirror cover panels
to the motor shaft.  If a given mirror cover panel will not remain open,
strap it open using duct tape or cable ties.  Replace the damaged motor.

Some possibly useful information if one wanted to redesign the
mirror covers: the mirror cover central rest is about 17 inches diameter.
The tertiary baffle is about 16 inches diameter and extends upward 
starting about 5 inches above the central rest.

\section{Detectors/cameras}

\subsection{Leach camera with E2V/Marconi CCD}

We obtained a E2V/Marconi 2048x2048 CCD 
(\htmladdnormallink{CCD42-40}{http://control1m.apo.nmsu.edu/1m/docs/42-40-BI-NIMO-Cpak.pdf} backside illuminated)
with 13.5 micron pixels through an internal grant at Los Alamos National
Labs written by Tom Vestrand. The CCD was obtained and put in an Infrared
Labs dewar by Bob Leach at UCSD, who supplied the controller. The dewar
is a model WD5, serial number 3538, job number ARMF28H.

Some drawings and photos for dewar/electronics can be found in
\htmladdnormallink{http://control1m.apo.nmsu.edu/1m/leach}
{http://control1m.apo.nmsu.edu/1m/leach}. Leach says 
"I think the attached drawings are the applicable ones. We built a similar
system for Wyoming, so don't get misled by that label. The heater is
terribly simple, just a 25 ohm resistor connected to the utility board."

Library software was provided by Leach, and LANL provided a simple command
line interface to this. The library stuff is inserted as a kernel module.
LANL supplied a computer with the camera; we have subsequently reinstalled
OS/software to make it compatible with other computer systems; however,
the RedHat 9 LANL installation+software was left on an unmounted partition
on the computer ccd1m; see this manual LaTeX source for account information.
% account: observer password: *Aggies
% account: root     password: jbRR9a

Leach provides application interface routines (astropci), as well 
as a Linux JAVA control program called VOODOO. We have used the interface
routines to implement our own software. Leach's software can be obtained
from astro-cam.com; access control information can be found in the LaTeX
source for this manual. Various Leach documentation from this web site
can be found at \htmladdnormallink{http://control1m.apo.nmsu.edu/1m/docs/leach}
{http://control1m.apo.nmsu.edu/1m/docs/leach}.
% account: jwren
% password: *mira

The Leach astropci software is installed under the 1m/tcomm source directory.
There are different versions for different kernels, see the documentation
for more information.

The dewar developed some leaks in 2007, and upon opening, several chips
were discovered in the dewar window. Inquire to Infrared Labs revealed
that we had a non-standard sized window, so a new top plate was purchased
with a standard size window and installed.

We had some issues with the shutter failing to open and opening incompletely
in December 2008. At that time, we ordered and received a replacement
shutter assembly from Leach; the shutter is a Pronto Magentic E64 (141176 2570)
shutter. The shutter comes with a small electronics board (to transition from
opening voltage to hold voltage???), but Leach told us to remove the board
and connect the solenoid directly to the HIROSE connector from the 
pins connected to a power supply through the timing card. We also discovered
that the shutter needs to be physically modified to fit in our mounting
scheme: the solenoid is farther from the shutter than in the Prontor default,
and several bolt holes need to be redrilled to allow for flush bolts to
be put in. Instead of making this modification, we simply replaced the old
solenoid with the new one, and redid the wiring (made the cable significantly
shorter. In fact, the old solenoid is probably still OK,
problems may have just been electrical with perhaps some small mechanical
alignment of the solenoid issues. We have retained the spare shutter and
the old solenoid for potential future use.

\subsection{FLI guider camera}

The guider was built around the format of an old Spectrasource CCD camera 
(Spectrasource is no longer in business). This constrains the physical
dimensions of a guide camera to be cylindrical in shape within a diameter
of a few (?) inches.

The current guider camera is a \htmladdnormallink{Finger Lakes Instrumentation}{www.fli-cam.com} 
MaxCam CM2-1, which has a 1024x1024 thinned backside illuminated sensor, with
13 micron pixels. Mechanical interface drawings can be found at
\htmladdnormallink{http://control1m.apo.nmsu.edu/1m/docs/fli}
{http://control1m.apo.nmsu.edu/1m/docs/fli}. This camera was purchased/installed
in summer/fall 2005.

The guider camera is controlled via a USB interface. To minimize the number
of computers in the dome, we purchased a fiber/USB converter kit made 
by Icron, so that the control computer, ccd1m, is located in the main 
computer room. A pair of fibers run from the 1m dome patch panel through
the cone to the receiver box, into which the FLI camera is connected.

Power is supplied by a small DC supply that is located up on the arm by the
camera. Communication to the camera is by a custom 10-pin cable (looks like
and Ethernet connector, but isn't) which goes to a small box with a control
card that is attached to computer via USB.

We sometimes have some trouble with the shutter not closing fully, especially
during cold weather. As a result, normal operation is to command the shutter
open at the beginning of the night and just leave it open; the small trails 
created during readout don't significantly affect the centroids. The camera was
opened in November 2006, and the shutter blades were cleaned with isopropyl
alcohol and reassembled. 

The guide camera stopped functioning in September 2008. It was returned to FLI 
and they repaired and cleaned it.  Specifically, they replaced components on
the A/D and internal power supply board, baked the camera, replaced o-rings
and dessicant packs, cleaned the sensor, and purged the camera.

FLI no longer makes this model, and their newer models do not appear to have
the same form factor.

\subsection{Apogee AP7P CCD Camera}

Before obtaining the Leach E2V CCD camera, we used an Apogee camera as
the main science instrument. This is now again being implemented as 
the acquisition/guide camera for the high speed photometer.

On the photometer port (also used for SDSS/APOGEE fiber feed), the camera
looks at the focal plane. This is accomplished using a Nikon macro lens.
To achieve maximum magnification, the lens needs to be focussed for the
closest possible object, which is about 10 inches away. This requires that
the camera be located as far down as possible in order to achieve focus.
Even still, the pixel scale is between 1.5 and 2 arcsec/pixel -- it would
be better to have a camera with smaller pixels.

The Nikon lens has an adaptor for a C-mount that screws into the camera.
The APOGEE camera has a large bolt circle, to which a snout that is 3
inches in diameter is mounted. This is barely large enough to fit around
the Nikon lens. The snout goes into a 3-inch diameter JMI focuser. This
is controlled using a JMI SmartFocus unit.

The \htmladdnormallink{Apogee}{http://www.ccd.com} 
AP7P is a thermoelectrically cooled CCD.  Documentation for this camera is located
on the documentation shelf in the 1-meter control room at Apache Point.
The camera has a 512x512 thinned back-side illuminated CCD.
The camera is plugged into the parallel port of the Linux machine,
eyeball, which must be located in the 1-meter enclosure.  Power to the Apogee is
supplied by its own power supply and special cable.
Follow instructions in the manual for safely connecting and disconnecting
the Apogee.

If the camera window fogs over, it is time to replace the Apogee's
desicant.  Consult the Apogee manual for detailed instructions.
Replacement desicant is located in the upper most of the big file drawers
in the 1-meter control room.

Apogee has kept some documentation for their AP series at
\htmladdnormallink{http://www.ccd.com/tech.html}{http://www.ccd.com/tech.html}.

\subsection{Apogee Alta CCD Camera}

In 2016, and APOGEE Alta F2 camera was purchased because lifetime for 7P seemed
questionable with parallel port interface. Alta camera has USB interface,
and identical mounting surface. The AP7P failed in spring 2017.

The Alta F2 camera has a C mount, which is a smaller diameter thread than
the AP7P. A C-mount to Nikon-mount adapted was purchased for installation.
The chip is a 1536x1024 with 9 micron pixels.

APOGEE completely redid their software API, so it was necessary to rewrite
the camera interface. Examples were provided in C++, so this was adopted
for the lower level camera control.

\subsection{JMI SmartFocus}

The HSP/APOGEE-feed acquisition/guide camera is mounted in a JMI Focuser
which has a JMI SmartFocus control. Originally the serial interface to the
SmartFocus was connected to the HSP Windows computer, and the Windows
software from JMI was used to control the unit.

In summer 2017, a Raspberry Pi was purchased to control the focuser.
The INDI drivers were loaded onto the Pi using the binary libindi 
packaged. Access to the pi (control1m, pi1m) is via default pi
username and password. The INDI server is started using:

nohup indiserver -v -m 100 indi_smartfocus_focus > /dev/null

A remote client is used to control the server; the standard is Kstars/EKOS. 
Unfortunately, this is not available for CentOS. It can be installed on
MacOS. The client needs to connect to port 7624 on the Raspberry Pi;
for remote operation this can be done using an SSH tunnel. 

Power for the Pi and the JMI focus are on a single power strip connected
to remote outlet on the azimuth disk (hsp1m).

\subsection{Princeton Instruments CCD Camera}

The original science CCD camera was was purchased from Princeton Instruments 
of Trenton, New Jersey.  This camera was retired in the late 90s (?) and is
now on long term loan to New Mexico Tech, via Bill Ryan, for their MRO 
observatory (I don't know if they are actually using it).

Princeton Instruments was bought out and is now Roper
Scientific.  Roper Scientific's technical support number is (609)
587-9797.  The people who seem to be most familiar with our system are
Paul Sandyck and Rob Allen.

There are several things to be familiar with when mounting and dismounting
the PI CCD camera.  First off, the shutter assembly on the camera screws
on and off with a flanged collar.  This collar needs to be tight,
otherwise, the camera can turn, changing the camera's orientation on
the sky.  Make sure not to rotate the shutter housing when tightening
the collar -
the mechanical solenoid that holds the shutter open can bind inside,
on the camera housing.
After tightening the shutter housing, run a test exposure to make sure
the shutter opens cleanly.  Next, assuming you are looking at the shutter
and the camera's fill tube is upright, there is a scratch on the shutter
housing at roughly the 10 o'clock position.  This scratch lines up with
a scratch on the filter wheel.  This allows you to mount the camera such
that north is approximately at the top of the chip.

The camera is held to the filter wheel with a set of three cleats that
screw into the filter wheel.  Since the camera is only clamped in place,
it is critical that the cleats be very snug to make sure the camera does
not rotate.

The camera control hardware consists of three components:  the camera
head, which is mounted to the telescope; the camera controller which is
in the left-hand side of the rack; and the computer interface card which
is installed in the Dell 486 Computer that is housed in the left-hand
side of the computer rack.  The camera operates at -110 Degrees C.
There is a temperature control knob on the front of the camera controller.
Next to that is an LED that shows green when the camera is at the correct
temperature and amber when not at the correct temperature.

Between the Dell 486 and the camera controller is a pair of roughly
4-foot long "Parallel" cables.  Care must be taken when plugging these
cables into the back of the camera controller
  so that the pins do not bend.  These cables have a tendency to pull
  free of the controller as
the rack is moved or as the temperature in the dome shifts.  Check
periodically to make sure the cables are firmly seated.

\subsubsection{Fast Readout Mode}

Fast readout mode capability has been added to the Princeton Instruments
camera controller in-house.  Inside the camera controller unit, on the
left hand side of the largest board visible from the top is a set of
jumper pins labeled with speeds (e.g. 200 KHz, 50 KHz, etc.)  This set
of jumper pins determines the readout speed of the Princeton CCD.  Note,
the faster the selected speed, the higher the dark current.  Under the
front plate of the camera controller again on the inside   is a box
with a relay.  This box contains a relay that alternately connects the
common (right-hand) side of the jumpers with either the 200 KHz or the
50 KHz pin.  The relay is triggered when 12-volts are applied to pins 1
and 2 of a 9-pin D-connector run out the back of the Camera Controller.
The default (i.e. power off) state of the relay runs at the 50 KHz rate.

\subsubsection{Vacuum pumping the Princeton Camera}

Vacuum pumping the Princeton Camera may be done on-site at Apache
Point Observatory.  It should only be done with the assistance of
APO personnel who are familiar with the operation of the vacuum pump.
Those people include Mark Klaene, Jon Davis, and Jon Brinkmann.

The following are guidelines to follow when pumping the camera.
\begin{itemize}
     \item Make sure that the camera has warmed to room temperature.  Do not pump the
     camera when it is at Liquid Nitrogen temperatures.
     \item The vacuum pump is a large unit typically stored in the back work area at Apache
     Point.  Before using the pump, make sure that you are instructed in its use by
     qualified APO personnel listed above.  Also, make sure no one else has planned to use
     the pump in the time it takes to pump the PI camera   typically 4-6 hours.
     \item The fittings to connect the PI Camera to the vacuum pump are in the right-most red
     tool box in the APO work area.  The qualified person assisting you can help you
     locate all fittings.
     \item The PI Camera should be pumped until it reaches a vacuum of at least 10-7 Torr.
\end{itemize}

\subsection{Remote video camera}

Three Logitech QuickCam 4000 Pro web cams, with USB interface are
connected to the ccd1m computer, via the USB/fiber interface boxes.
These require the pwc driver.

We also have a Panasonic eggcam with its own PCI interface card, but this
requires a computer in the dome to run (eyeball was used for this purpose).

\section{ Observatory Dome}

The Observatory Dome was built and installed by AshDome.  Richard Olson,
President of AshDome, is very helpful in answering questions and
helping one track down replacement parts.  AshDome's phone number is:
(815) 436-9403.  Their website is:  www.ashdome.com.

\subsection{Upper Dome Slit}

There is a manual control box for the upper dome slit on the north wall of
the 1-meter enclosure.  To operate the slit manually, put the dome into
"Manual" mode with the upper switch then use the raise/lower switch as
appropriate.  There is a limit switch next to the motor at the top of
the dome that automatically shuts the shutter power off when the dome
is all the way open or closed.

Procedure to follow in the event that the upper shutter fails to close:
If the 1-meter dome does not close, immediate inspection of the dome
and its control systems are required.  The following is a point-by-point
list detailing the steps and order that should be followed in the event
that the dome does not close.

\begin{itemize}
     \item Visually inspect the dome shutter.  From outside, make sure that ice, or some other
     culprit has not caused the shutter to leave the tracks.  From inside the enclosure, look
     straight overhead and see that the dome-stop switch is making good contact with the
     switch-trigger (a metal piece that looks a lot like a handle).  If you cannot see the
     front of the dome shutter or the "handle" there's a good chance that the dome had slid
     back too far and is not engaged with the motor gear.  If the dome fails either of these
     tests, the 1-meter contact person AND either Mark Klaene or Jon Davis should be
     contacted immediately so that further plans may be made.  Cover the telescope as
     described in step 7 if bad weather is imminent.
     \item If the dome appears to be on track, proceed with an inspection off the electrical
     system.  Go to the gray box on the north wall of the dome.  Place the top switch into
     the "Manual" position.  Use the "Open/close" switch to try to move the shutter.  If it
     closes, fine. If it hangs up, one could try such things as "rocking" the shutter to see if
     it will clear an obstruction in the tracks. 
     \item Next, if "manual" control of the dome fails to work, check the circuit breakers.  In the
     gray box to the right of the door, are the circuit breakers.  Look for any that are
     triggered; flip them back into position and try the shutter again.
     \item If the circuit breakers are not tripped, check the fuses.  There are four fuses in the
     gray box on the North wall of the dome.  The fuses are located in the bottom of the
     box.   Of the four fuses, two are "active" and two are "spares." The fuses not labeled
     "spare" are the "active" ones.  Check both of the active fuses.  If either one appears to
     be burned out, replace it with a fuse from the spare slot.  Try to move the shutter as
     described in step 2.
     \item At this point, if bad weather is imminent, skip one step and close with the hand crank. 
     Continue diagnostics after the telescope is safe.  (The reason for waiting to do the
     handcrank until now is that checking the fuses and circuit breakers should go pretty fast.)
     \item If the circuit breakers are OK and the shutter fails to move, check the shutter relays. 
     The shutter relays are located in the small gray box next to the lower shutter on the
     dome itself.  It requires working from the ladder.  The box is opened with six Phillips
     head screws located on the sides of the box (Note: two are missing as of this writing).  The entire box comes off the
     wall; you may need to unplug it to inspect the relays adequately. Check the three
     relays.  The relays plug into sockets.  Often, damage takes the form of a splotch of
     silver on the side of the relay where the contact was blown.  Replacement relays are
     located in the bottom drawer of the center tool box on the table on the south side of
     the dome.  Replace any suspicious relays and try the shutter again.
     \item If the relays are OK and the shutter still fails to move, it's time to try closing with the
     hand-crank.  The hand crank is located in the base of the 3.5-meter enclosure on the
     wall to the right of the big double doors (note: this location is subject to change by
     APO policy.  If the manual crank cannot be located, contact Bruce Gillespie or Mark
     Klaene to find where it's been moved to.)  The hand-crank is a long pole with a hook
     on the end.
     \item Return to the 1-meter enclosure with the hand-crank.  Look overhead.  On the bottom
     of the motor is a large eye-hook.  Hook the eye-hook with the pole.  Turn the crank. 
     The dome shutter should move.  Close the dome if possible with this method. 
     Remember to un-hook the manual crank when done.  If for some reason, you meet
     resistance when trying to crank the dome, or it fails to move, there is probably
     something wrong that will require further inspection.
     \item If all else fails and the dome cannot be closed at all.  Cover the telescope with the
     canvas and plastic tarps located under the brown table against the south wall.  If
     possible, move the dome so that it is not over the computer rack that's against the east
     wall of the dome.  Call Jon Holtzman or Mark Klaene immediately. 
\end{itemize}

We had continuing problems with the original relays for the upper dome
slit. These dome relays were Icron MY4 relays, 110/120 VAC (Mfg part
MY4-AC110/120(S).  They can be purchased from a variety of sources,
e.g. www.mouser.com. Twenty were ordered in February 2006 of the MY4N
variety (with indicator light) for about \$8.00 each.  

In late 2007, Mark Klaene and Dave Woods redid the upper (and lower) slit control
to use more robust relays. These appear to work much more reliably, although
we have still had some failures that resulted in blown fuses.

The dome box has two fuses, with spots for two spares; these are 6A slow 
blow fuses.


\subsection{Dome Home Switch}

There is a dome azimuth home switch located just south of east, mounted
to the slip-ring contactor post.  There are spares for this switch in
the  big red tool box on the south side of the 1-meter enclosure.


\subsection{Lower Dome Dropout}

Automated lower dome dropout operation was designed and built in-house by
David Summers.  Electrical drawings are in the Yellow Observatory folders
and there are copies stored in the master control box inside the 1-meter
enclosure itself on the South wall.  To achieve lower dome automation, a
Bodine AC motor was simply attached to the winch that raises and lowers the
dropout. 

The winch is a Fulton K2550 Work winch; this was purchased to replace the
original K1550 winch which has a lower gear ratio. The winch was purchased
from Grainger.  The higher gear ratio winch was obtained so the the dome
would not ``bounce'' on the limit switch, which happened with the lower
gear ratio; the downside of the higher gear ratio is that the lower dome
slit takes about 3 minutes to open or close.  These winches are ratcheted
for pulling the load in; the load is let out using a slip clutch, so
the ratchet \textit{is not} disengaged when the dome is lowered.

The shaft is 2.5" from the mounting surface, and the shaft has a 0.5"
(??) diameter.  There is a relay box for motor control on the dome it's
the hinged small box to the right of the upper dome slit relay box.

To lower the dropout using the master box on the south wall of the
enclosure, flip the lower red switch into the "Lower" position and flip
the upper red switch into the "Power On" position.  It is important
to perform this operation in the correct order.  Once the dropout has
powered itself off at the limit switch, move both of the red switches
to the power off or automatic positions.

To raise the dropout using the master box on the south wall of the
enclosure, flip the lower red switch into the "Raise" position and flip
the upper red switch into the "Power On" position.  Once the dropout has
powered itself off at the limit switch, move both of the red switches
to the power off or automatic positions.

If the motor or power fails, the dropout may be closed by grabbing the
blue clutch between the motor and the winch and turning by hand (or use
a wrench on the nut) until the shutter is closed.

Sometimes the lower dome gets closed too tight (mostly with the old
winch) and the dome clutch does not slip to let it open again. If this
happens, simply take a wrench and turn the nuts on the dome side of the
blue clutch in the opening direction (relieving tension on the cable);
the dome should start to open.

Lower dome closed limit switch: white goes to common, yellow to norm closed.

In late 2007, Mark Klaene and Dave Woods redid the lower dome slit
system to use more robust switches (and relays?).

\subsection{Rotation and Motor Adjustment}

The dome's rotation motor is located on the north side of the dome just
above the dome control box.  The rotation motor gear engages track
that goes around the circumference of the dome.  Good engagement of
the gear and rail are critical for accurate positioning of the dome.
The dome motor is on a hinged base.  Engagement of the gear and rail is
maintained by a single piece of threaded rod that comes up from the motor
mount and through the motor base.  There is a nut on the the threaded rod.
If the nut works loose, the motor can be allowed to "bounce" on the rail
instead of driving the dome.  When this happens, a rather distinctive
bumping noise can be heard.  To fix this, the hinged piece needs to be
tightened. However, the bolt which currently runs through the bottom plate
was poorly designed so there is nothing apart from friction to keep it
from turning when the nut is tightened. As a result, it takes some care
to tighten down the motor. Until a better design can be implemented, the
preferred strategy is to use a C-clamp to tighten down the hinged plate
until the motor engages fully, then the nut can be tightened without
having the bolt rotate; do not use pliers to try to hold the bolt at
the threads will likely become damaged. Tighten the clamp and then the
nut so that there is good engagement between the motor gear and the rail.

The dome motor gear seems to come disengaged first when the dome slit
is either over the motor or opposite the motor (Dome azimuth of 0 or
180 degrees.)  To check that the dome is well-engaged, put the dome into
manual mode and rotate until the slit is over the motor.  Run the motor
in short bursts.  It should start and stop smoothly.  If there is a loud
banging or clanking as the motor starts and stops, it's time to adjust
the motor.  Repeat this test with the dome 180 degrees from the first
test position.

Periodically, the rollers around the circumference of the dome should be
greased so that the dome operates quietly with minimal wear.  There is a
spray can of open gear lube located on the brown table against the south
wall of the enclosure. However, Jon Davis recommends using a dry lube
because of our dust exposure. He likes Dow Corning Silicone Lubricant 557,
but unfortunately this comes in an aerosol can and it is not recommended
to use an aerosol near optics. Instead, John recommends spraying some
in a squirt bottle outside the dome, waiting for it to warm up, then
squirting it from the bottle. There is a can of the lubricant in the
APO machine shop, in a green aerosol can.

However, Norm Blythe recommends using a grease on the sprockets, and
another lubricant on the rollers.

Dome rollers were relubed 2014 May 30 with a 3in1 Garage Door Lube.

\subsection{Dome encoder}

The dome encoder is coupled to the shaft at the motor. The encoder is a
BEI Motion Systems (1-800-350-2727) Optical Encoder E206-1000-3G. It was
originally read by the Tech 80 card in the telescope computer, but in
July 2005 this stopped functioning (not positive which element failed),
and we switched over to using the CP4016 card instead. The latter only
reads 16-bits, so some code modification was necessary to count encoder
wraps (this only functions properly when dome is moved under computer
control!). The cable connections are different for the two cards, so
we made a converter cable to work with CP4016. The CP4016 also only
has inputs for the 3 poles, while the Tech 80 had the 3 poles plus
their inverses (a support person at COP said it was OK to just hook
up the 3 poles).

A new encoder was ordered and installed in November 2006. At this time,
it was still left hooked to the CP4016. The dome was having some serious
positioning problem, unclear if this was due to the encoder, the
coupler with the dome motor, or the dome motor to dome interface. A
new, longer coupler was installed with the new encoder.

\subsection{Dome control}

The dome is controlled through the Autoscope OCS box. This box takes
input from an OMS PC34 card in the telescope computer.  The OCS box
itself contains Opto-22 relays to switch things on and off. This
box controls: rotation power, rotation direction, shutter power, shutter
directory, dome home switch sense, watchdog and watchdog reset, and
LN2 fill system through an auxiliary port.

A description (probably not fully accurate) of the system can be found
in \htmladdnormallink{the Autoscope documentation}{http://control1m.apo.nmsu.edu/1m/docs//OCS.pdf}, along with some
\htmladdnormallink{schematics}{http://control1m.apo.nmsu.edu/1m/docs/OCS_diagrams.pdf}. 

Some information is also available about the \htmladdnormallink{OMS PC34 card}
{http://control1m.apo.nmsu.edu/1m/docs/PC34.pdf}.

\section{Dome auxiliary systems}

\subsection{Cabling and Conduit}

Cabling to the telescope is done through conduit that runs from below
the computer rack into the pier.  There is a hole in the bottom of the
azimuth cone and cables may be run up through that hole.

The easiest way to run cable to the telescope is to tie off one end of
the cable on the telescope structure, then drop the cable through the
hole in the top of the azimuth plate and work it until it goes below
the azimuth cone.  Next, get an extension ladder and place it alongside
the pier on the north side.  Just below the floor of the dome is a small
access door that opens with a flat-bladed screwdriver.  Open the door.
You'll see where the cables turn the corner into the conduit.  Take the
loose cable and direct it into the conduit.  The conduit itself has
access doors.  Unfortunately, the design of the dome is such that the
doors do not open very far.  Do your best to reach in through the access
doors in the conduit and guide the cable to a place where you can grab
it from the exit hole under the rack.

Cables must be checked periodically to make sure there are no wraps
down in the azimuth cone.  There is sufficient slack in the cables (and
new cable runs should be done with sufficient slack) that the azimuth
would need to wrap several times before cable wraps become an issue.
To check on the cables, look down in the azimuth cone and check that
the cables are hanging freely and not twisting around themselves.

Rotator cables should be checked for wraps at each init as directed in
the telescope operations guide.

When working on conduit, be aware that mice and rats often get into the
conduit.  You should take hanta virus precautions.  Wear a dust/nuisance
mask rated for protection against organic vapors.  There are masks in
the 1-meter control room.  Also, Mark Klaene at Apache Point Observatory
has masks that can be used.  Wear rubber gloves.  If rodent droppings are
found, make a mixture of 50% chlorine bleach and 50% water.  Spray this
mixture on the droppings to neutralize any potential hanta virus.
Once work is completed, discard the rubber gloves and mask.

\subsection{Instrument Rack}

The instrument rack sits against the eastern wall of the observatory
enclosure.  Ideally the rack should keep the computers and other sensitive
electronics at a comfortable operating temperature, while not allowing
heat from that equipment to affect dome seeing.

Ron Yarger, on the APO staff, is a qualified air-conditioning service
technician.  He can be consulted regarding the heat exchanger unit on
the roof of the enclosure over the left-hand side. (This unit has been
removed!).

The left side of the rack houses, from bottom to top, the motor controllers,
weather station (in front of motor controllers), telescope control (TCS) box,
observatory control (OCS) box, guider/filter wheel control, and main power
box. All of the motors and controller boxes are plugged into a power strip
at the very top of the rack. This power strip is plugged into the main power
box, which gets its power from the Masterswitch unit in the right side of
the rack. The only utility of running the power through the main power box
is that it enables the power switches on this box, and in particular, the
emergency stop button which has an access point just inside the front door.
Beware that the main power box takes several seconds after power is supplied
before power goes out to the motors, etc.

There is also a temperature control senson on the main power box that will
kill power if the temperature goes out of range. This is an Omega series 
CN9000A/Model CN9111A controller which we have set to have a wide 
allowable range of temperature.  There is a manual in one of the notebooks 
for this and also a PDF file (M1191.pdf) in the docs directory. This unit
started to fail in 2009 (with calibration errors EE8) so it was temporarily 
removed and the temperature circuit bypassed (10/09).

The right side of the rack houses the telescope control computer, the CCD
LN2 autofill control, a monitor, the 16-output APC masterswitch
unit, and a UPS unit near the bottom, below the compuer.  

The UPS is a APC SU1400RM2U, which is a model that has been discontinued.
Replacement batteries (replacement battery cartridge RBC24) were still 
available in 2008, at which time the batteries were replaced.

%Below is a drawing of the rack layout.  Cables come up from below the
%right-hand side of the rack.  The motor controllers, which are labeled,
%are accessed from the rear of the rack.  The power strip on the left-hand
%side of the rack acts as an emergency stop switch.  The power strip on the
%right only controls the computer monitor.  The monitor should be powered
%off for normal (i.e. automated) operation.  There is a temperature
%readout / control in the middle of the Main Power Box.  This unit is
%part of the air conditioning system.  However, if the temperature gets
%too far above or below the set point, the effect is that the monitor
%will not turn on.  To adjust the set point, press and hold the * button
%while adjusting the temperature up or down with the arrows.

Currently the rack is vented into a vent hole in the dome floor that lead
via conduit into the pier. There is a fan at the base of the pier, which
, when activated, will draw air through the rack. As of 2004, this fan
has been put under computer control via the APC power strip. For this to
work, rack needs to be fairly well sealed, allowing air to come in only
at the top of the rack and passing across the electronics before venting
at the bottom of the rack. A filter should be placed over the opening at
the top of the rack to prevent moths, etc from being draw into the rack;
dimensions of the hole are ~16x14 inches.  The power strip at the bottom
powers the fan in the venting and also the fan in the telescope pier;
it is plugged into the Masterswitch unit.

It may be a good idea that a better insulated rack be purchased
for the equipment in the 1-meter enclosure.  A good source of new racks
appears to be ITS Enclosures.  It is possible that their enclosures
could be placed outside the 1-meter enclosure itself.  Their products
may be viewed on the web at: http://www.icestations.com

\subsection{Autofill System}

The auto-fill system was purchased from VBS Industries in California.
The product manager was David Carnahan.  The engineer was Alan Ziegler.
They may be reached at (408) 371- 3303.  There is a file in the upper
drawer of the tan filing cabinet with details regarding the construction
of the auto-fill system.

The auto-fill system has a manual control box located in the rack just
under the monitor, on the shelf with the screen switcher.  The key should
always be in the "on" position.  Fills are triggered via one of the TCS
auxiliary output lines.

The system has an auto-shutoff feature, where
The autofill shuts off when Liquid Nitrogen makes contact with a sensor
mounted in the fill tube that mounts in the CCD dewar.  This sensor has
a gold-colored tip and may is located in the right-angle pipe near the
top of the fill tube assembly.  The wire to the shut-off sensor is in
the metal "cable" that is attached to the fill tube. However, we were
never able to get this to work reliably. As a result, the sensor connection
has been replaced with a plug that closes off the circuit. Dewar fills
are shut off using a timed fill duration instead of the sensor.

One should periodically inspect the Autofill lines.  Make sure there are
no apparent breaks and that nothing has come unscrewed in the plumbing.
The most likely place for a disconnect is at the azimuth cone, just
below the black solenoid that controls the fill.

%According to Princeton Instruments, the maximum fill is achieved when the
%dewar is horizontal.  This was borne out in testing done with the dewar.
%Essentially, the rotator angle must be 90 or 180 degrees.

\subsubsection{Manual Override of the fill}

Fills can be triggered in the dome by powering on the autofill system
from the APC MasterSwitch and pushing the green button on the front of
the VBS manual control box.  If needed, a manual fill may be terminated
by turning the key to the "off" position.  Just be sure to turn it back
"on" before returning to the autofill mode.

\subsection{LN2 Hoisting and Attachment}

A 100-liter Liquid Nitrogen dewar is used to supply the E2V detector
with coolant.  This dewar is connected directly to the VBS Autofill system
documented above via a flow regulator.  VBS-supplied hose runs from the
flow-regulator into the conduit under the enclosure and up through the
center of the azimuth cone to the solenoid.

As of 2006, the big LN2 dewars are left below the enclosure, and a long
hose goes up through the cone, the valve, and eventually, the dewar.
It takes about 7 minutes to get a complete fill of the dewar, without
any pressure regulation on the big dewar.

%To get the big LN2 dewars into the 1-meter enclosure, use the Genie
%Superlift stored at the base of the enclosure.  The Superlift is a manual
%hoist that can lift over 800 pounds 20 feet into the air.  Norman Blythe
%at Apache Point Observatory is familiar with the lift's operation.
%Documentation on the lift is located in the top file drawer in room 110
%of the Astronomy building.  There is also lift documentation in the tube
%mounted to the base of the lift itself.
%
%To hoist LN2 into the 1-meter:
%\begin{itemize}
%     \item Open the hatch in the floor of the 1-meter enclosure.  This requires two screwdrivers;
%     one to turn the latch in the floor and one to pry the hatch far enough open that you
%     can grab it.
%     \item Remove the protective tarp from the Superlift and set aside.
%     \item Place the two red ramps (stored under the table against the south wall of the
%     enclosure) between the legs of the Superlift.
%     \item Remove the used LN2 dewar from the enclosure by raising the lift   using the
%     handcrank   until the platform is level with the enclosure floor.
%     \item Disconnect the used dewar from the Autofill system using a 7/8 wrench.
%     \item Move the dewar onto the platform and secure, using straps, to the handles mounted to
%     the back of the platform.
%     \item Lower the dewar.
%     \item Lift off the platform using the dewar-handling cart at APO.
%     \item Lift the fresh dewar onto the platform using the dewar handling cart.  The red ramps
%     you've put into place are designed to get the axles of the cart up and over the wheels
%     at the front of the Superlift.
%     \item Raise the fresh dewar into the dome using the Superlift.  One person can do this, but
%     it is easier if two people operate the crank.  Bring the platform up until it is level with
%     the enclosure floor.
%     \item Pull the fresh LN2 dewar to its position next to the rack.  Make sure that when
%     unloading the Superlift, someone is stabilizing the lift so it doesn't rock when the
%     dewar is coming off.
%     \item Connect the dewar to the Autofill system.
%     \item Lower the platform.
%     \item Close the floor hatch.
%     \item Cover the Superlift with the canvas tarp and secure.
%\end{itemize}
%
%The big LN2 dewars that are used to fill the PI CCD weigh 515 pounds
%when full and 235 pounds when empty.  One fill takes approximately 7
%lbs of LN2.

\subsection{Fans and Louvers}

The enclosure's louvers open by simply applying and maintaining power to
the enclosure's compressor system.  
%Currently, the system is designed such
%that when the louvers in the enclosure open, four box-fans just inside
%the louvers come on simultaneously.  If this system proves good at venting
%air from the dome, the fans should be replaced with more permanent fans.

The way the louvers work is that power is applied to the appropriate
socket of the APC master switch.  The socket does not supply power
directly to the louvers.  Instead, it simply energizes a relay
that in turn activates a power strip.  The louvers come on
when power is applied to the power strip.  The relay is housed in a
small black box that currently sits under the southeastern louvers.
The power strip is plugged into an outlet mounted to the relay-enclosure.

There is a large fan in front of the SE louver. This is plugged into
one of the controllable APC Masterswitch outlets.

\subsection{Weather Station}

The weather station is well-documented in the Yellow Observatory Binder.
One item not documented is that if the weather station loses power for
less than ten seconds, it will not come back up fully operational.
Someone examining the rack will see lines across all of the places
where numbers would normally be displayed.  The weather station will
not respond to commands from the Tocc.  To remedy, simply unplug the
weather station for approximately fifteen seconds, then plug it back in.

\subsection{Tools \& Spares}

Most required tools are in the red tool boxes on the south side of
the 1-meter telescope enclosure.  Additional tools may be found in the
3.5-meter and Sloan work areas.  As a rule, tool drawers are labeled with
the items that are enclosed.  When borrowing tools from the 3.5-meter
or Sloan, make sure to alert the appropriate personnel (Mark Klaene or
Jon Davis, if no one else is available) and return the tools as soon
as possible.

The upper-most shelf by the window over control1m  in the 1-meter control
room contains spare motors and a breakout box for testing the TCS and
OCS boxes.  There are also spares available in the two large drawers under
the white board in the 1-meter control room.  There's a small supply of
spare relays (including spare dome relays) and motors (including mirror
cover motors) in the lower-most drawer of the big tool box on the south
side of the 1-meter enclosure.

\appendix

\section{Computers}

The computer control system for the 1m consists of the following 
computers:
\begin{itemize}
  \item tocc1m, the telescope, dome, guider stage, and filter wheel control 
        computer, a PC running DOS
  \item ccd1m: computer with Leach controller card that controls the E2V CCD,
        also the FLI guider camera, and remote web camerasi
  \item eyeball: the Apogee CCD control computer and remote video controller,
        PC running linux
  \item command1m: the master control computer that sends commands to all of the
        above computers, a PC running Linux
\end{itemize}

In 2016, command1m failed, and all control functionality was transferred to
ccd1m (telescope commuication, Leach communication, FLI communication, and
master control). In 2017, ccd1m failed, and a replacement was installed. At
this point OS was updated to CentOS6. This required recompilation which is
an issue for camera devices, requiring updates to new camera libraries, and
subsequent rewrite of camera interface software.

The current method of commmunication between the comoputers is via commands 
over sockets. On the DOS machine, this is enabled using 
networking software from
NetManage, their PC/TCP Software Development Kit. This allows for socket
communication, which was implemented circa 2004. There is a set of 5 manuals,
and software comes on a set of 3.5 floppies and some CDROMs. The serial
and key number can be found in the LaTeX source of this manual.
% Part: PC50232
% License: 1 User
% Serial #: 4000-0386-4736
% Key: 1388-1998-0369

The old method of communication between the computers was via disk files that
are seen by each computer on an NFS mounted disk on control1m. In fact, the
software for telescope and guider PCs are also located on the control1m disk.
Most computers used PC-NFS Version 5.1; the installation disks can be found
in the 1m control room; the PI computer was still on version 4.0.
Note that if one wants to change the license of PC-NFS, the only way is
to delete the existing version and reinstall.

The power to the computers (and several other things) is controlled by a 
network device that has 16 outlets that can be switched on and off via
network commands. There is a separate 8 outlet network power strip located
on the azimuth disk.

\subsection{Network}

The APO network uses the following:
\begin{itemize}
  \item Router: sampepys, 192.41.211.1
  \item Netmask: 255.255.255.0
  \item Nameserver: 192.41.211.10, 192.41.211.40, secondary 192.41.211.98
  \item IP addresses for 1m machines:
    \begin{itemize}
      \item tempa1m: 192.41.211.18  (tempager3E 00:20:4A:AD:81:C8)
      \item ccd1m: 192.41.211.19 (Princeton control PC) ( 00:13:20:53:E6:BF)
      \item powera1m/power1m : 192.41.211.20 (power control (APC)) (00:C0:B7:92:01:5A)
      \item eyeball : 192.41.211.21 (remote video, Apogee CCD control ) (00:04:76:EA:41:E3)
      \item tocc1m: 192.41.211.22 (telescope control PC) ( 00:03:1D:03:0D:08)
      \item powerb1m/spare1m : 192.41.211.23 (available for notebook, etc.) ( 00:C0:B7:76:96:79)
      \item command1m : 192.41.211.24 (PC in control room )
      \item tempb1m : 192.41.211.25 (tempager3E 00:20:4A:AD:81:7D)
      \item hsp1m/nmsuhsp : 192.41.211.220 (photometer computer) (00:80:2F:10:55:14)
      \item powerc1m/photpower : 192.41.211.221 (photometer power strip) (00:16:18:65:13:90)
      \item video1m : 192.41.211.130 (network video camera) (00:80:F0:AD:E3:2A)
    \end{itemize}
 
\end{itemize}

As of December 2010, APO switched the 1m computers over to a local network, with addresses obtained via DHCP.

\subsection{The telescope computer}

A new telescope computer was ordered March 2006 from 
\htmladdnormallink{Industrial Computers Inc}{http://www.eindustrialcomputer.com}. This contains a 
backplane with 5 ISA and 7 PCI slots and a single board 
(PICMG slot) computer with integrated display, Ethernet, etc, plus an 80 Gby 
disk, CDROM, and floppy drive. The SBC is an LGA 775 with Pentium 4 processor.
Ethernet is via an Intel 82562EZ 10/100.  It uses DDR2/533 memory, currently 
a 256Mby SIMM. 

Installation of DOS on this computer proceeded as follows: DOS installed
using standard 3 MS-DOS installation disks. After installation, computer
hangs on boot while loading HIMEM.SYS. Reboot from floppy and replace
HIMEM.SYS with Windows version (usr/local/1m/dosfiles). Network software
was switched from PCNFS to PCTCP to allow socket communication. To get
network to work, drivers for 82562EZ (Intel PRO 100) were downloaded
from Intel (PRODOS.EXE); this file didn't execute on DOS and needed
to be unpacked on a Windows machine.  This included both ODI and NDIS
drivers. I could get the ODI driver to load (LSL, E100BODI, IPXODI),
but was unable to get the PCTCP packet driver interface ($\backslash$PCTCP$\backslash$ODIPKT)
to work with it. I then installed the NDIS driver. First, I did a PCNFS
installation, but of course the new device is not in the installation
list. I then modified $\backslash$LANMAN$\backslash$PROTOCOL.INI and put in a section for
the new device driver (E100B.DOS). The driver loaded successfully on
bootup. Commented out PCNFS initialization. Finally, I redid the PCTCP
installation and it saw the NDIS device. I used frame type DIX-ETHERNET
and PCTCP kernel worked!

During 2007-2010 we would get sporadic hangups of the telescope computer,
sometimes going a long time without one, and sometimes having multiple
hangups in rapid succession. In April 2010, the entire single board computer,
including CPU and memory was replaced to see if this makes any difference.
The replacement single board computer was bought from IPC Hammer and was an 
FS-979 (also possible known as PG-7791); it differs slightly from the original
in that it has two different Ethernet controllers. A Pentium 4 CPU was purchased
from MemoryLabs.

Unfortunately, this didn't seem to make a difference for the hangs. I then
modified the software to update the telescope status less frequently (~4 seconds), 
and this appears to have reduced the hangs significantly, although not 
eliminated them. Subsequently, I suspect that the culprit is the motor
control card (PC38-6E); see information in the motor control section about
the replacement of this card.

In July 2014, the computer started to freeze/crash shortly after reboots. I
took the card out, and discovered that the two big fans in the computer case
were not attached to power, so I connected them. I went back to the old SBC
because I thought a fan on the board wasn't working, but it was the same with
the old SBC (not the CPU fan but a smaller one on the board). I left the old
SBC in there, so the FS-979 is back in the in the drawer in the office; not
clear if there could be a problem with it or not! Note that when I connected
the new card, I had to change the weather station to COM2 (in sysscf.new).

Linux (Redhat Enterprise 4-U1) was also installed on this computer for 
possible convenience.

The ISA slots are populated by:
\begin{itemize}
\item \htmladdnormallink{Oregon Micro Systems PC48-6E}{http://control1m.apo.nmsu.edu/1m/docs/Pc48mand.pdf} (SN 3454, with EW138)  motor controller
(short card that controls telescope alt/az/rot and 3 focus axes), 
\item OMS PC38-6 (SN4053) motor controller (controls guider radial and focus stage, filter wheel, tertiary rotation), 
\item Oregon Micro Systems PC38-6E (SN 902)  motor controller
(spare long card that used to control telescope alt/az/rot and 3 focus axes), 
\item \htmladdnormallink{OMS PC34}{http://control1m.apo.nmsu.edu/1m/docs/PC34.pdf} motor controller (controls dome), 
\item Technology 80 5312 encoder card (long card previously used with dome encoder),
\item Computer Optical Products CP4016 motor/encoder controller purchased for
use with rotator encoder. The Tech 80 card was used for the dome encoder,
but in July 2005 we had problems and switched over to using the CP4016
card instead. 
\end{itemize}
A weather station is plugged into a serial port (COM1 or COM2, depending on how it it attached to SBC).

\subsubsection{Old telescope computer}

The old telescope computer was an \htmladdnormallink{Industrial Computer Source}
{http://www.icsadvent.com} (tech support and customer service 1-800-480-0044,
sales 1-888-294-4558, original sales order 144151, 7/9/97) 200 Mhz Pentium
PC (7000-8MB chassis, motherboard 586MBH200, floppy drive, RAM, KB3
with touchpad, 2" CPU fan, power supplies Turbo-Cool 300 Slim with P8/P9
power connectors); three of these computers were ordered, one
for the telescope, one for the guider, and one spare.  The operating
system is DOS. The program it runs is called TOCC.EXE and it lives on
E:\\BIN\\TOCC.EXE.

The telescope computer has a 166 Mhz processor, a 1.2Gbyte disk, and
8 Mbytes RAM. It has eight (??) ISA slots. These are populated by:
video card (SIIG VGA), 3c509 networking card, PC38-6E (SN 902)  motor controller
(long card that controls telescope alt/az/rot and 3 focus axes), PC34
motor controller (controls dome), PC38-6 (SN 4053) card for filter wheel and guider,
Technology 80 5312 encoder card (long card used with dome encoder), and
a Computer Optical Products CP4016 motor/encoder controller purchased for
use with rotator encoder. The Tech 80 card was used for the dome encoder,
but in July 2005 we had problems and switched over to using the CP4016
card instead. A weather station is plugged into a serial port (COM1),
and a voltage meter used for measuring the PI CCD temperature is plugged
into the parallel port.

The networking is done using PC-NFS. The PC-NFS configuration is as follows,
as reported by the NET ALL command:
\begin{itemize}
\item The name of this system is tocc1m, and its IP address is 192.41.211.22.
\item The license number of this copy of PC-NFS is PC-NFS10D3DD92.
\item The subnet mask is 255.255.255.0 [0xffffff00].
\item It is in network information service domain noname,  for which there is no NIS server.
\item DNS configuration file:  resolv.cnf
\item DNS primary server:  galileo.apo.nmsu.edu (192.41.211.40)
\item DNS alternate server:   tycho.apo.nmsu.edu (192.41.211.98)
\item DNS search domains:  apo.nmsu.edu
\item The authentication server is sdsshost.apo.nmsu.edu (192.41.211.171).
\item Non-local routing via gateway sampepys (192.41.211.1).
\item You are logged in as nobody, with UID -2 and GID -2.
\item It is Sun Feb 04 00:56:16 2001, MST
\item The license number of this copy of PC-NFS is PC-NFS10D3DD92.
\end{itemize}

\subsection{The Apogee CCD/remote video computer}

This computer, eyeball, runs Linux (RedHat 9). It was purchased from
Elite PC, and is a  600 Mhz Pentium III with 46Mby (!!) memory and a 80 Gbyte
disk drive. 

\subsection{The control computer}

The control computer, command1m, runs Linux (RedHat Enterprise 4). 
It is a Dell Dimension 5150. See the LaTex source for this manual for 
account information.
% account: tcomm  password: clear$skies
% account: root   password: hbox@217

On RHEL4, sometimes on the virtual desktop, the window manager stops
allowing windows to be resized or moved; if this occurs, kwin --replace
seems to fix it!

\subsection{The ccd computer}

The camera control computer, ccd1m, runs Linux (RedHat Enterprise). It
is located in the APO computer room. It communicates with the dome via
a fiber link using a fiber repeater purchased from Icron. This computer
controls the Finger Lakes guider camera (USB), three USB webcams, and the
Leach camera (custom card). See the LaTex source for this manual for
account information.
% account: tcomm  password: clear$skies
% account: root   password: hbox@217

The Leach controller requires a kernel module to be loaded. This is compiled
for a specific kernel version (2.4), so significant attention must be paid
if the system were to be upgraded. In addition, the 2.4 kernel required
some modifications to allow control of the Quickcam 4000 Pro webcams.

\subsection{Power control}

Power is controlled by two network controllable power switches.

The first commandable power controller is an APC MasterSwitch, part number
AP9210, serial number wa9720838756, Ethernet address 00:c0:b7:92:01:5a;
this has 8 plugs. It is statically configured as power1m.apo.nmsu.edu.
It can be accessed via a serial port, via the WWW,
or via SNMP. The regular usage is via SNMP; four machines can access it
this way, and these machines are control1m, eyeball, tycho, and tocc1m.
To access via serial (only necessary if the WWW password has been
forgotten), one needs a female 9-pin serial connector cable. This
cable should have a null model pin-swap. The Linux program minicom can
be used to communicate over serial port, using /dev/ttyS0.

The second commandable power controller is an APC ..., Ethernet
address 00:c0:b7:76:96:79; this has 16 plugs. It is statically configured
as spare1m.apo.nmsu.edu. This has a serial port
cable included with the device; serial communication as above.

To access via http, one needs a username and password. These are set
the same as the observe account on control1m except only 8 characters are
used for the password.  In desperate situations, there is a backdoor
password that was supplied by APC. See the APC manual for this device or
LaTeX source for this manual for access information.
% 8 plug Masterswitch: power1m
%   user: observe  password: clear$sk
%   backdoor password:  95;;:<^IBA=<   (APC case# 1887240)
% 
% 16 plug Masterswitch: spare1m
%   user: observe  password: clear$sk
% 

The ccd1m computer, located in the APO control room, is also on a
controllable power switch.

% computer room masterswitch
%   apo/foenix
%   manager/208volts

Photpower (powerc1m) is a controllable power switch for the high speed photometer.
% hspphot/hsp-phot

%hsp1m VNC nmsu-hsp admin: nmsuhsp

\subsection{OBSOLETE: guider computer}

WE NOW NO LONGER HAVE A SEPARATE GUIDER COMPUTER. The guider stages are
controlled by the telescope computer, and the camera is controlled by the
ccd1m computer.

The guider computer was an \htmladdnormallink{Industrial Computer Source}
{http://www.icsadvent.com}, identical to the system for the telescope computer
described above. The program it runs is called GCS.EXE
and it lives on E:\\BIN\\GCS.EXE.

The guider computer has a 200 Mhz processor, a 1.2 Gbyte disk, and 8
Mbytes RAM. Its ISA slots have an SIIG video card, a 3c509 network card,
a long card used to communicate with the Spectrasource guide camera, and  a long
PC38 card for the guider stage and filter wheel control.

The networking is done using PC-NFS. The PC-NFS configuration is as follows,
as reported by the NET ALL command:

\begin{itemize}
\item PCNFS 119I : The PC-NFS Version is 5.1a.
\item The name of this system is pic1m, and its IP address is 192.41.211.22.
\item The license number of this copy of PC-NFS is PC-NFS2208E0F6.
\item The subnet mask is 255.255.255.0 [0xffffff00].
\item It is in network information service domain noname,  for which there is no NIS server.
\item DNS configuration file:  resolv.cnf
\item DNS primary server:  galileo.apo.nmsu.edu (192.41.211.40)
\item DNS alternate server:   tycho.apo.nmsu.edu (192.41.211.98)
\item DNS search domains:  apo.nmsu.edu
\item There is no authentication (PCNFSD) server
\item Non-local routing via gateway sampepys (192.41.211.1).
\item You are logged in as nobody, with UID -2 and GID -2.
\item It is Sat Feb 03 01:38:48 2001, MST
\item The license number of this copy of PC-NFS is PC-NFS2208E0F6.
\end{itemize}

\subsection{OBSOLETE: spare computer}

The spare computer is an \htmladdnormallink{Industrial Computer Source}
{http://www.icsadvent.com}, identical to the systems above. It has a
200 Mhz processor, 1.2 Gbyte disk, 64 Mbyte RAM, a CDROM, and a 3c509
network card.

The spare is no longer functional. The old control computer, loki, may
serve as an adequate spare.

\subsection{OBSOLETE: PI control computer}

The PI control computer wass a Dell 486 running Window for Workgroups.

\section{Software control}

\subsection{Instrument blocks}

Imaging program has INIT/GOS command. Point at a bright star before
issuing command. INIT moves star around in 3x3 grid by fixed amount on sky
(50 pixels, but can specify with INIT argument), finds centroid, and gets
linear transformation solution; it loads this in internal varaibles. In
addition to this, instrument block needs location of instrument relative
to rotator center. To do this, determine offset in RA/DEC to move
star from rotator center to center of instrument while at PA=0. Then
INSTBLOCK dra ddec command will send proper SETINST command to tocc to
set instrument block, using the offset and the transformation information.

At a lower level: SETINST inst sx sy dra -ddec rot, where rot is in degrees.

For instrument at NA2 (no rotator), need to add extra arguments: rot0 fixed, 
where rot0 should be the negative of the derived rotation, and fixed should
be 1 to identify a fixed instrument.

The instrument blocks show up at the end of the sysscf file.

\section{Maintainance checklists}

\subsection{Routine (weekly) maintenence}

\begin{itemize}
\item Clean all drive wheel surfaces

\item Inspect dewar connections for leaks. Tighten fill tube nut.

\item Inspect azimuth and rotator drives to confirm no cable wraps. 

\item Inspect cables, etc. on azimuth disk (be careful of the fiber!) Does the
big LN2 hose need to be tightened on azimuth disk?

\item Check video camera light bulb(s)

\item Inspect mirror for possible cleaning
\end{itemize}

\subsection{Nighttime checks}

\begin{itemize}
\item Check and adjust collimation

\item Pointing model

\item Check instrument blocks

\end{itemize}

\subsection{Longer term maintenance}

\begin{itemize}
  \item Run telescope to limits to check proper performance and smooth motion

  \item Run secondary to home and reset position

  \item Run guider to home and reset position. Check that guider home position
        is good default for focus.

\item Pointing model

Before running a pointing model, make sure the time is set well on the
computer (SETTIME), and check the azimuth motor scale (CAZ). 
Check the rotator center and implement it in pm.pro.

For a pointing model, one looks at stars of known position at many locations
across the sky, and determines the pointing correction as a function of
altitude and azimuth. There is an XVISTA procedure, pm.pro, that will take
the pointing model data, based on an input file of desired positions, 
pm.dat.  Given the poor pointing of the telescope, one needs to use 
a preliminary pointing model to even be able to find the stars on the
frames.

If totally lost, use spiral.pro to try to find a bright star. Once found,
use pmwrite.pro to write a pointing model entry for it; \textbf{IMPORTANT,
pmwrite want to write target RA/DEC with observed az/alt, so need to use
GUIDEINST to move telescope, not OFFSET!} Can gradually move around sky,
adding stars to a pointing model file until enough are found to try to
do an initial model. Need to use posrecen to change from output file
form to TPOINT input file format.

Also have option (recenter$<$0) that will use last GUIDEINST on next star,
if things are really bad, this may help to find stars for positions that
are adjacent to previous position (see, e.g., pmstep.dat).

pm.pro takes images, finds bright star and moves it to rotator center, then
takes new image, stored in pm???.fits. So this has observed az/alt to 
get star in rotator center. For pointing model, compare this with 
predicted az/alt. pm.pro creates proper input file for TPOINT using
program posrecen.c. Use output pm.dat for TPOINT INDAT command,
then CALL ALTAZ and USE NRX NRY to 
implement the proper alt-az Nasmyth pointing model terms, then FIT multiple
times to get a solution. To see results graphically, do PLTON xwindows, then
CALL A9 (for 9 different graphs) or GSMAP (to see residuals as a function 
of position in sky). You can remove points using, e.g. MASK r g 50 to mask
points with residual greater than 50 arcsec.  Once a satisfactory fit is
found, OUTMOD file.mod will output the model.

To implement the model, use TPFILE command in command window, which asks
for TPOINT .mod file, then transmits the coefficients via the TOCC TPOINT
command. 

OLD WAY:
To analyze the data, the positions of the stars on the frames are measured.
This can be doing using the XVISTA pos.pro procedure. Then the measured
positions need to be reformatted into a file that TPOINT can read; the
program coord2 does this; it takes the output .dat file of pos.pro as
input and outputs to standard output a .mcd file. The .mcd file needs to
have 4 header lines put into it (someday I should fix all the software to
just do everything properly automatically!). Once you have the .mcd file,
you run TPOINT (currently available on avalon only, after source 
holtz/star/etc/login and source holtz/star/etc/cshrc). Using INDAT file.mcd
to read in star file, INMOD file.mod and UNFIT to undo the pointing corrections
made by the operational pointing model, CALL ALTAZ and USE NRX NRY to 
implement the proper alt-az Nasmyth pointing model terms, then FIT multiple
times to get a solution. To see results graphically, do PLTON xwindows, then
CALL A9 (for 9 different graphs) or GSMAP (to see residuals as a function 
of position in sky). You can remove points using, e.g. MASK r g 50 to mask
points with residual greater than 50 arcsec.  Once a satisfactory fit is
found, OUTPUT file.mod will output the model.

To implement this model, move the .mod file to control1m:/export/tocc and edit
the file /export/tocc/toccscf.new to use the new .mod file.

\item{Check focus motors to confirm proper operation}

\item{Visually inspect all filters}

\item{Check vacuum on dewar, pump.}

\item{Check all telescope limit switches (altitude and azimuth)}

\item{Check dome shutter hard limits, motor adjustment, etc.}

\item{Wash mirrors}

\end{itemize}

\subsection{To-do items}

\begin{itemize}

\item Install guider corrector (done early 2008)

\item Permanently bolt shelf to telescope (done 4/2008)

\item Redo cables through azimuth cone, replace fiber if needed (done 3/2008)

\item Run fiber with more fibers between building and dome

\item Diagnose optics. Action...

\item Implement fast readout for CCD camera

\item Implement separate shutter for guide camera

\item Further investigate/implement encoder on rotator. If rotator is out
of round, tabulate look-up of scale vs. rotator angle, once latter can
be measured reliably and repeatably.

\item Redesign dome rotation motor mount to allow for proper easy adjustment
of motor engagement with track. Install soft-start relay on dome motor
if possible (1/3 hp single phase, split phase, 1750 rpm motor)

\item Repair rack air conditioning unit and/or replace rack. Consider relocation
of equipment to outside of dome.

\item Design/implement some sort of dome sensor to confirm dome closure when
commanded. Implement software for email/phone notification in case of failure.

\item Implement more permanent mounts for fans in front of louvers. Color of
extension cords should be changed (site policy has orange for clean power,
i think); extension cords should be safely/permanently laid.

\item Determine/implement backup policy for all computers. 

\item Determine/implement upgrade policy for all computers.

\item Add fuse to lower dome circuit (rated lower than relay!).

\item Install rain sensor, and implement rain sensor shutdown in software

\item Replace auto/manual switch for upper dome (it's showing signs of age)

\item Replace light in OCS power switch

\item Consider replacing connectors on guider with "L' connectors to reduce
cable strain

\end{itemize}

\section{Maintenance/repair log}

\begin{itemize}

\item June 2017: enabled new az encoder to try to address pointing issues.
Encoder worked only after Ed added impedance
matching. Severe pointing issues seem to be unrelated to encoders, however,
and seems to just be very large pointing model term (AW, also NRX/NRY).

\item May 2017: severe pointing issues noted

\item May 2017: significant computer failure of ccd1m. Replaced with new
computer, updated to CentOS 6. Significant issues with camera drivers (Leach,
FLI) require rewrites.... also need to develop software for new Alta camera

\item spring 2017 New ccd1m computer ordered

\item fall 2016?: failure of command1m, all control transferred to ccd1m.

\item spring 16: new APOGEE Alta camera purchased for long-term support, but
not installed

\item spring 16: az encoder failure, disable x_encoder. Replacement ordered
and installed, but not enabled

\item 10/17/16: Ed replaces 6A fuse after dome shutter fails halfway open/closed

\item 8/18/16: Ben and Ed replace dome rollers after more rotation failures

\item 5/30/14: Today we lubed the dome wheels and and track with a 3in1 Garage Door Lube. 

We installed a permanent grease fitting.  It seems strange to us that there is just one grease port but this must be a small bearing. We greased the azimuth bearing with Mobil 460. Uncertain how much grease to add we rotated the telescope by hand in azimuth as we added 20 pumps with the grease gun. After 360 degrees of rotation, we added another 10 and then parked the telescope. 

\item 7/9/13: reinstalled all three mirrors after coating and overcoating
by ATOC in Albuquerque. All went fairly smoothly. Used print overlay
of secondary to put a spot in physical center of mirror. 
Did alignment of secondary
with collimating telescope, only adjust tilt a little. Did alignment of tertiary
with collimating telescope: adjusted tertiary rotation, and piston a bit.
Did initial primary aligment on sky. Note azimuth scale change because of
new az encoder shaft.

\item 6/13: Ben rebuilt all azimuth idler boxes. Readjusted azimuth motor.
Replaced aluminum encoder cam with steel one.

\item 8/24/11: dewar removed and put on pump, appears to have been very
soft. It has been warm most of the summer because of the filter wheel
problem and the weather.

\item 8/24/11: Ed/Ben repair filter wheel which has been failing since
late June. Ed: "1-m Filter wheel detent problem is fixed.  The original
symptoms were that the 1m would not rotate the filter wheel because
the 'DETENT OUT' signal was not being received by the OMS controller.
The detent motor could be heard turning so attention focused on the
'DETENT OUT' limit switch.  I was able to measure switch action at the
detent connector by using a very long probe to actuate the switches.
Both checked out ok for switch function.  This led me to troubleshoot the
system and interconnections.  Had some problems due to poor documentation
but broke through when Ben built a breakout cable for the OMS (Oregon
Motion Systems) and we were able to stimulate the status bits manually
and confirm that the OMS side was working correctly.  We verified the
cabling so that left the limit switch itself (even though the very
first test did not agree).  Ben and Bill removed the unit and the bench
inspection revealed that the switches were indeed functioning well.
We reverse documented the circuit and saw that the 'DETENT OUT' switch
was not actuating when the detent was indeed out. Ben discovered that
the small allen set-screw was very loose and had screwed itself in.

Ben and I adjusted the switch trip points and verified that the filter
wheel would be able to turn freely.  Ben did apply a small amount of
locktite to the set-screw to prevent the failure mode.  Bench testing
showed proper operation in and out.  Ben cleaned the inside of the unit
with IPA and removed moth and dust debris. We re-installed the unit on
the instrument and tested it using Jon's provided GUI.  Detent functions
are working properly.

I have documented the circuitry (attached) that I saw and it is saved
on my computer under my 1m path."


\item 3/31/11: Bill replaced 656/25 in slot 3 with 657/27

\item 3/28/11:  replaced broken dome encoder coupler and adjusted dome motor, after apparent failure of dome tracking.
Ed:  "Just wanted to add a couple of things for the documentation of
this repair.  Ben also addressed the probable root cause of the flex
coupling failure.  The encoder shaft and motor shaft were misaligned so
Ben modified the encoder mounting bracket to proper alignment.  We also
re-terminated the encoder connector with a high pressure housing and
receptacles from Molex.  Same form factor but the new receptacles grab
the encoder posts from all four sides instead of the previously installed
economy type which only spring tensions along one side."
Ben: "Mark spotted the flexure was cracked in half for the encoder. Ed
found another flexure in your tool box i installed. Also we found the
motor assembly was loose, we tightened it back up. when testing everything
i found that the up limit on the lower hatch was out of adjustment. I
readjusted it and is working correctly."

\item 2/24/11: repair of azimuth home sensor.
Ed: "This email simply documents what Ben and I did to the telescope
this morning.  We disconnected the TCS rack and brought it into the
lab for servicing.  Visual inspection shows clogged fan filters but
the inside of the cabinet looked relatively clean and well laid out.
Re-seat all connectors in the 'X-axis home' signal chain.  replace the
IM103 digital output Opto-22 module (red color).  Tighten  all Opto-22
module retaining screws and blow out dust from cabinet and fan filters.
Reassemble the cabinet and reinstall.  Called you up to perform a homing
verification and finished.

Since the Az home problem is very intermittent, only time will tell
if replacing the Opto-22 module repairs the problem.  To date we've
re-terminated the magnesensor read head connector (both ends), aligned
the read head and magnet, and replace the Opto-22 module.  If it fails
again we'll try replacing the Sony detector (PD-100) or perhaps the
entire kit (read head, magnet, PD-100).  I have a request for quote into
Industeq. Inc. for both parts.  I'll contact you when the pricing and
availability is known.

Please let me know if the problem returns."

\item 10/11/10: Replaced camera after pumping.

Bill: "FYI, vacuum with heat blanket on got down to 5.7e-6
torr. After heat was removed it pumped down to
1.8e-6 torr."

\item late 2007, Mark Klaene and Dave Woods redid the lower dome slit
system to use more robust switches (and relays?).

\item 11/06: lube of azimuth bearing

\item 05/04: lube of azimuth bearing:
"about 1/3 tube of grease was applied (Mobil fairly lightweight, type ???; this
was done through a small grease port that did not have a grease fitting; we
used such a fitting but the thread of the hole apparently didn't match the
fitting well, so we could not get the fitting to stay in. After lubrication,
a new felt seal was installed using masking tape."

\end{itemize}

\section{Puntino analysis}

\label{sect:puntino}

SH computer:  10.75.1.48 (port 5900) tom/nmsuhsp  vnc: nmsu-hsp

Mount Puntino with spacers!

Move filter wheel to OPEN position!

\subsection{Offsets and tilts}

\subsubsection{post-April  2012}
\begin{verbatim}
Moved focal plane back 1.7 inches by installing spacer

mount SH with TOP towards CCD controller
should use metal spacers if possible to allow guider to focus!

star at (680,320) in guider at guideloc 1500, INST 2 puts star
  in SH at INST 1 after GOFFSET 50 0
  (so 630,320!)

set rotator at approx 10-alt to get spiders diagonal:
   positive x offset moves dots down
   positive y offset moves dots right
   S tilt --> +x    N tilt --> -x
   W tilt --> +y    E tilt --> -y


\end{verbatim}
\subsubsection{pre-April 2012}
\begin{verbatim}

focus around -2000
guidefoc 300
star at (680,320) in guider at guideloc 1500, INST 2 puts star
  in SH at INST 1

set rotator at approximately 20-alt
then:
positive x offset moves dots to the right
positive y offset moves dots up
N tilt --> +y   S tilt --> -y
W tilt --> +x   E tilt --> -x


\end{verbatim}

\subsection{Analysis}

\subsubsection{120501}
\begin{verbatim}
Start image 321
\end{verbatim}

\subsubsection{120426}
\begin{verbatim}

reference 28 taken as test
reference 29 taken just before sunset

saw relatively poor SH results with coma and spherical, and astigmatism
jiggled primary --> coma changed significantly
moved secondary to remove coma, and spherical seemed to improve
moved tertiary to remove astigmatism, and spherical seemed to improve

looked at range of altitudes in the east, and coma didn't change by too much

ended up at xtilt 0.02, ytilt -0.54, tertiary 31900




\end{verbatim}
\subsubsection{120219}
\begin{verbatim}

reference 26 taken as test
reference 27 taken just before sunset

120226
595,575 at inst=2
590,585


185-...  moved star, moved focus to more position (hit guider range)

193 new star at higher alt

put in 1.375 (2 wood + 1 metal), move guider to opposite end of range (0),
focus much more negative (~-6000) no chance of guiding

196 still not in focus, actually maybe this is good!
 worked on centering, see 212
213 moved focus -50
214 df 25 to focus=-5325, very nice
changed PA by 20 degrees

--> Looks like moving back 1-1.25 inches will be good!

\end{verbatim}
\subsubsection{111111}
\begin{verbatim}

  moved primary, SH gave tilts of 0.16,-0.36 around alt=50-60

\end{verbatim}
\subsubsection{111103}
\begin{verbatim}

alt 55
focus -2725
(0.06, -0.04)
  102   coma 1.8  -88  0.11 -1.80

(0.06, 0.01)
  103   coma 2.67 -88  0.08 -2.67       bigger ytilt -> more negative y coma

(0.06, -0.09)
  104   coma 0.59 -101 -0.11 -0.58

(0.06, -0.11)                        (is this correct?)
  105   coma 0.51 -8.7  0.50 -0.077  (weird y?)

(0.04, -0.11)
  106   coma  0.42 -77  0.09 -0.41      

(0.02, -0.13)
  107   coma 0.6 154    -0.54 0.26      smaller xtilt -> more negative x coma

(0.02, -0.11)
  108   coma 0.58 135   -0.41 0.41

(0.03, -0.10)
  109   coma 0.58 136   -0.41 0.40
      focus 8.39
df -50
  110     focus 10.41
df 150 to -2625
  111     focus 4.36
focus -2575
  112     focus 2.27
focus -2525
  113      focus 0.47
        coma 1.3 -120   -0.65 -1.13
repeat
  114     coma 
repeat
  115    coma 1.1 -121   -0.56 -0.94
(0.03,-0.12)
  116     coma 0.67 -157 -0.62 -0.26
(0.01, -0.12)
117     coma 2.02 -153   -1.78 -0.92
(0.05, -0.12)
118     crash
119     coma 0.87 -5.6   0.87 -0.08
(0.04, -0.12)
120     coma 0.33 -20.6  0.31 -0.11
altitude at 45

move to higher altitude 67
122  coma   1.32 -147
123 bad
124 bad
125  coma  1.64 -146   
offset
126  bad offset
127 bad offset
128 still off
129 still off
130  coma 1.77 -130



230, 57, -78
Procedure:

  set filter to open position
  GOTO 170,alt with INST=2, guideloc 1500
    move star to 660,380
    inst=0, find guide star and star guiding
  Determine PA to align spiders to diagonal  
    pa=20, rotator=-48 at alt=68
           rotate=-15   at alt=35
       rotator=20-alt
  (figure out which direction in guider corresponds to which direction in SH)A
    goffset +x  moves SH image right
    goffset +y  moves SH image up
  Center image and guide 
  Focus
  Take several mirror images
  Adjust xtilt and ytilt seperate to determine effects on coma

        focus  alt   tilt      focus    coma
0145,           70 0.04,-0.12
0146   -2990    67
0147   -2740                     9.44    1.59, 0
0148   -2590                     2.65    1.48, -3
0149               0.04, -0.08           0.55, 1.5
0150            66       -0.06           0.55, 23.8
   center a bit
0151
   newstar
0152            31   forgot INST 1
0153                bad PA, and bad pointing
0154                 forgot INST 1
0155                bad pointing
     guideinst 2 0 -40
0156                still off in x, missing spots on R
     guideinst 2 40 0
0157                                    2.29, -178 pointing OK, missing spots, dome?
0158                                               still vignetted
0159                                    2.85  -178

0160                0.04, -0.14         1.6   175
     moved dome manually +5 degrees
0161         worse vigetting?
     moved dome manually -5 degrees
0162          even worse!
     moved manuall +15
0163           disaster!
     back to dome slaving!
0164  

move to new star
0165            40
    guideinst 0 -50
0166               0.04                  0.66  123

move to new star
0167            62

\end{verbatim}
\end{document}
