\documentclass[10pt]{report}

\usepackage{makeidx}
\usepackage{epsf}

\includeonly{%
chap1,%
chap2,%
chap3,%
chap4,%
chap5,%
chap6,%
chap7,%
chap8,%
chap9,%
refs,%
index,%
}

\makeindex

\renewcommand{\[}{\begin{eqnarray}}
\renewcommand{\]}{\end{eqnarray}}

\renewcommand{\tilde}{\~\space}

\renewcommand{\floatpagefraction}{0}
\setcounter{topnumber}{1}
\setcounter{totalnumber}{1}

\newenvironment{hanging}{
	\begin{list}{}{
		\labelsep=0pt
		\labelwidth=0pt
		\listparindent=0pt
		\itemindent=-\leftmargini
		\leftmargin=\leftmargini
	}
}{
	\end{list}
}
\newenvironment{references}{
	\begin{hanging}\raggedright
}{
	\end{hanging}
}
\newenvironment{url}{
	\begin{list}{}{
		\labelsep=0pt
		\labelwidth=0pt
		\listparindent=0pt
		\leftmargin=\leftmargini
	}\item\tt
}{
	\end{list}
}
\newenvironment{code}{
	\begin{list}{}{
		\labelsep=0pt
		\labelwidth=0pt
		\listparindent=0pt
		\leftmargin=\leftmargini
	}\item\tt
}{
	\end{list}
}
\newenvironment{example}{
        \begin{list}{}{
                \labelsep=0.25in
                \labelwidth=2.0in
                \leftmargin=2.75in
                \itemindent=0in
                \itemsep=0in
                \parsep=0in
        }
}{
        \end{list}
}



\newcommand{\note}[1]{\marginpar{\raggedright\scriptsize{#1}}}

\makeindex

\title{\Huge\bf NMSU 1m User's Manual}
\author{\Large
Jon Holtzman
and
David Summers
}
\date{\Large November 1998}

\begin{document}

\setcounter{page}{1}
\pagenumbering{roman}

\maketitle

\tableofcontents
%\listoffigures
%\listoftables

\clearpage

\setcounter{page}{1}
\pagenumbering{arabic}


\chapter{Introduction}

The NMSU 1m is an alt-az Ritchey-Chretien telescope located at the Apache
Point Observatory. It was constructed by the AutoScope corporation, although
the company went out of business before they really completed the project.

The 1m dome is the small dome located closest to the 3.5m dome.
The 1m control room is in the main APO building. If you are observing at
APO, you will get a key from the observatory staff. The same key should fit the
main building, your dorm room, and the 1m dome.

\section{Computers }
In the control room, there is currently one computer: 
loki, a 400 Mhz Pentium PC, which runs the Linux operating system.

\begin{itemize}

\item loki is the main computer used for operating the 1m. 

A wide variety of software, including IRAF, xvista, TeX/LaTeX, etc. is 
available.
\end{itemize}

On loki, users should use the observe account, which has the password:
\begin{quote}
clear\$skies
\end{quote}

On loki, start up the X window systems after login by typing:  
\begin{quote}
startx
\end{quote}

Remote logins to loki (and uhuru) can only be done using slogin and the RSA
authentication method. See Jon for details.

In the dome, there are two racks which hold all of the telescope and CCD
control hardware. The left rack holds the main power switches for the
telescope/dome, and also the CCD controller and the PC which operates the
science CCD. The right rack holds two PCs which control the telescope and
the guide camera. The three PCs are:
\begin{itemize}
\item PC which handles telescope and dome control, which is called the
TOCC computer. IP name (address) is tocc.apo.nmsu.edu (192.41.211.25).
This is a Pentium 200 MHz PC.

\item PC for the science CCD, which is called nmsucam1 or the Princeton
Instruments(PI) computer. This is a Dell 486. It runs a program called
winview, which was written by Princeton Instruments and which
runs under Windows 3.1. IP name (address) is nmsucam1.apo.nmsu.edu 
(192.41.211.19).

\item PC for the SpectraSource (guider) CCD, which is called the SPEC computer
(or alternatively, mc1m, or the MC). IP name (address) is pic1m.apo.nmsu.edu 
(192.41.211.22). This is a Pentium 200 MHz PC.
\end{itemize}

All three computers have a common monitor and keyboard. You can switch 
between which computer is being talked to with the 3 position toggle
switch above the keyboard in the right rack. In normal operation, you
don't need to type anything from the dome keyboard at all, although it
may be advisable to do the telescope initialization which outside in the
dome. To do this requires some minimal input to the TOCC computer, so
it is recommended to leave the toggle switch in the TOCC position. 
Toggling this switch to the PI position occasionally causes that computer
to hang up, so don't switch the switch without any reason.

In addition, there is another PC in the dome, uhuru, which runs Linux. 
This computer controls a video camera which can be used during remote
operation to confirm the health of the telescope (in particular, the
rotator).

\section{Emergency telescope control}

At the current time, the only emergency stop button in the dome is the
button which controls the power switch at the top of the left power rack.
\textbf{Please identify the location of this button before ever using
the telescope.}
Hitting this button will kill power to the telescope and dome motors and
should immediately stop the telescope and dome.  This should
be done if there is ever any question about whether the telescope is
moving safely, or is about to run into something or someone.
In the near future, we hope to reinstate the red panic buttons.
At the current time, if the situation requires that you hit the emergency
stop button, \textit{you must call Dave or Jon before restoring power and
using the telescope}; after hitting the stop button, the telescope will
be lost and not be aware of the fact that it is lost, which can be a
very dangerous situation. Hopefully, you will never need to hit this
button.

Emergency stop is also available through computer control using the
power program which is automatically started when the telescope control
suite (tcomm) is started.

%Note the location of two red {\bf panic} buttons in the dome. 
%One is
%located on the telescope structure directly in front of you as you enter
%the dome, and the other is located on the left power rack. Hitting either
%of these buttons will immediately cut power to the telescope. This should
%be done if there is ever any question about whether the telescope is
%moving safely, or is about to run into something or someone.

\section{WARNING about manual telescope motion}

Although it is possible to move the telescope manually, this should {\bf never}
be done. Always move the telescope using computer control, unless instructed
otherwise by Dave or Jon. The reason for this is that it is possible with
an alt-az telescope to have cable wrap-up problems, which can be avoided
only if the software accurately can keep track of where the telescope
is. If the telescope is moved manually, the software won't know about it,
and it may then command moves which will cause trouble. This is especially
true for the rotator, which has no limit switches at all, and consequently
its state of wrap is monitored by software only.

\chapter{Starting up}

You should get the telescope started and initialized well before it gets
dark, to make sure everything is working well. Everything can be initialized
without opening the dome and mirror covers. You should probably wait to
open the dome and mirror covers until roughly an hour before twilight,
just to make sure that bad weather doesn't surprise you.

\section{Power to the telescope}

All power to the telescope components is done using software control
through a device called the MasterSwitch. This device allows network control
of eight outlets in the dome, into which various components are plugged
in. Power switches in the dome should not be used except in the case
of an emergency. 

\section{Starting up the software in the control room}
Everything for the 1m can be done from inside the control room (or remotely from
NMSU!). The telescope is run from the Linux PC, loki. On loki, make
a directory which you will use for the night's observing. The convention
is to make a directory immediately under the home directory (/home2/observe)
with the name of the subdirectory being the UT date for the current
night in the format yymmdd, e.g., if you are observing on October
16/17, mkdir 971017, and then change to that directory:  
\begin{quote}
cd ~observe\\ 
mkdir 971017\\ 
cd 971017\\ 
\end{quote}

When you start the telescope control programs, a variety of windows will
be opened. To keep an organized desktop, there is a menu item (left or
right button on a blank area of the desktop) called tcomm xterm which will
start an xterm of an appropriate size to avoid overlap with other windows;
you may wish to start this xterm, although the telescope control program
can be started from any xterm.

Start the suite of telescope control programs by entering
\begin{quote} tcomm \end{quote} in the xterm window. 
As noted above, it may be a good idea to
have nothing else running when you are controlling the telescope.

The tcomm command will
actually start six different progams running - one is the power control
program, four others will run
in new xterms which will automatically be opened, and one (the main command
program will run in the window where you typed tcomm). The programs are:

\begin{itemize}
\item command : main command window from which ALL commands should
              be entered.
\item telescope control (port) : lower right, used to communicate with TOCC
              computer. This progam emulates the TOCC computer display and
              you can directly type in this program with results which are
              the same as if you were typing in the dome
\item CCD control (ccd) : talks with the PI computer to control the science CCD
\item SPEC control (gccd) : talks with the SPEC computer to control the guider CCD
\item status : upper right, reports telescope and CCD status information to user
\item power : upper panel of screen, shows power status and can be used to
emergency stop the telescope, and toggle individual powers (which should not
be necessary, as these can also be controlled through the command window).
\end{itemize}

In the future, the port, ccd, and spec programs may come up in an iconified
state, as they are really only necessary for engineering work - you should
not need to ever type anything in these windows.

When you start tcomm, the command program will automatically check to 
see whether various devices (telescope control program, telescope and dome
motors, science CCD, and guider CCD) are powered up and responding. If
they are, the program will proceed without requesting any input. If not,
the program will ask if you wish to power various devices up. 

Once the powers are checked, you will be prompted to initialize the telescope
if it has not already been initialized - see below for more details on
this. After this, you will get a Command: prompt and the program is
ready to accept normal commands.

To quit from the program, you enter the QU command. When this is entered,
the program will ask you if you wish to quit the telescope control program
running in the dome. If you are done for the night, you should answer
yes to this question; however, the telescope will need to be reinitialized
after quitting the program in the dome. If you do not quit the dome program,
then restarting tcomm will allow immediate operation of the telescope.
After this question, the program will ask if you wish to kill power to
\textbf{everything} in the dome; if you have quit the telescope control
program in the dome, you can answer yes to this if you are really done.

It is very important, however, that you do not kill the power to the
telescope control program in the dome without exiting it gracefully.
Note that the telescope control program which runs in the dome 
needs to be exited gracefully
(i.e., commanded to exit) because the program writes out the current
rotator position which is essential to record to prevent cable wrapups.
This is so important that the software has a built-in feature which 
makes a record of when the program is exited. If the program is not
exited gracefully, the next startup will REQUIRE THE ENTRY of a password
FROM THE DOME before proceeding further. If you get in this mode, you
MUST call Dave or Jon - note this is true even if you happen to know the
password! Because of this, make sure NOT to turn off the power via the
network power switch without gracefully exiting the telescope control
software first - if you do, you won't be able to get it started again!

\section{Initializing the telescope}

When the computers are powered up, the telescope and CCD programs go
through some initialization commands. For the CCDs, no user input is 
required - you will see a series of commands come up automatically in
the CCD control windows. For the telescope, some interaction is required
which needs user input either from the command window
or from the TOCC computer in the dome. During initialization it is important
to watch the rotator to make sure that cables do not wrap - as a result,
initialization should either be done in the dome, or, if operating remotely,
using a video camera view of the rotator.

After the TOCC computer is booted up, you will see on the screen (both
in the dome and in the telescope control window) the following question:

\begin{quote}
Do you wish to initialize the telescope with a :\\
\  F: full (normal)init - finds home positions without any previous knowledge\\
\ I: medium init (uses stored coords to find home positions - will be very\\
\ Q: very quick init (stored coords - assumes telescope hasn't been moved!)\\
\ \    slow if stored coords are wrong)\\
\  S: skip initialization\\
Enter your choice: \\
\end{quote}

If you will be initializing from the dome, enter S. In the dome, enter
HO (for HOME telescope) at the command prompt to start initialization.
This will take approximately 5 minutes.
The telescope will move first in azimuth, then in altitude, and finally in
the rotator, until it finds its home switches. This is a good time to
look at how the telescope moves and to confirm that no cables appear to
be under high tension. In particular, you should watch the rotator
closely. If it appears that the cables leading to the rotator are wrapping, 
hit the emergency stop button (on upper part of left panel in dome), or,
if running remotely, by clicking the mouse in the red Emergency Stop section 
of the power program at the top of the screen. If this happens, you should
check with Dave or Jon before restarting anything.

You will not be able to move the telescope until it is initialized.

Next the program will ask:

\begin{quote}
Do you wish to initialize the dome (Y or N)?
\end{quote}

Enter Y. The dome will spin around until it finds its home sensor. Very
rarely, this fails; if it fails, the countdown on the screen will expire
without the home position being found. If it fails, this is not a disaster 
as you can command a dome initialization from the command prompt later on
with the DI command - you will need to remember to do this!

Next the program will ask:

\begin{quote}
Do you wish to slave the dome (Y or N)?
\end{quote}

Enter Y. When the dome is slaved, it should automatically rotate so that
the slit is oriented in front of wherever the telescope is pointed.

Finally the promgram will ask:

\begin{quote}
Do you wish to open the louvers (Y or N)?
\end{quote}

The louvers allow significant airflow through the dome. They should be 
opened to assure optimal image quality \textit{if it is not too windy}.
If it is windy, then image quality may be better with the louvers closed.
Also, opening the louvers exposes the telescope to the elements, so they
should only be opened after ensuring that weather conditions do not
threaten the health and safety of the telescope! If in any doubt, do not
open the louvers at this point; they can be opened/closed later from
the command prompt using the OL/CL (open/close louver) commands.

Note that these startup questions \textbf{do not} open the dome - you
must remember to do that yourself, after confirming that weather permits
it!

A list of available commands can be viewed at any
time using the HP (Help) command.

\section{Weather and telescope safety}

You are fully responsible for the safety of the telescope in terms of
its exposure to weather. You should \textbf{not} open the dome or the
louvers if there is any chance that weather will damage any of the contents
in the dome. 

The easiest way to insure this is to follow to the letter the decisions
of the 3.5m telescope operator. You should check in with the 3.5m operator
at the beginning of the night and get informed about the weather status.
It is important that you must consider not only clouds and the possibility
of rain, but also the humidity and the dust conditions.

You should monitor the APO 10-micron all sky image 
\begin{center}
http://www.astro.washington.edu/deutsch/apoinfo/cloudscan/latestimage.html
\end{center}
and also the APO weather page 
\begin{center}
http://www.astro.washington.edu/deutsch/apo/weather/latestweather.html
\end{center}
closely.


\section{Preparing to observe}

If the weather is good, you should open the dome roughly an hour before
twilight. If in any doubt about whether the conditions are good enough
to open, consult the 3.5m telescope operator - if the 3.5m is closed,
you should be too.  This goes for staying open during the night as well.
Since you will be depending on the measurements made at the 3.5m to 
determine whether you should have the dome open, be sure to check in with
the 3.5m operator at the beginning of the night and tell them that you
will be observing on the 1m and that you would appreciate it if they
would tell you if they are closing the 3.5m for a weather condition. 
If they tell you if they are closing, you should close too!

You can open the dome with the OD command (CD to close the dome). When the
dome is opened/closed, the mirror covers should be closed to prevent anything
from inadvertently falling on the primary. If these are open, the program
will ask whether you want to close them before doing the dome - answer Y.

Next you should open the mirror covers, using OM (CM to close them).

\section{Filling the dewar}

The dewar for the science CCD needs to be filled with LN2 to keep the chip
cold. We have a remotely operable filling system which can be commanded
to fill the dewar. This can be done by simply typing FILL. The FILL 
command will rotate the rotator until the fill opening is pointing up,
open the valve from the filling dewar, and then suspend for between 3
and 10 minutes (exact time depends on time since last fill). The valve
will close automatically when the dewar is filled.

The current hold time for the dewar is approximately 10 hours. It takes
about 1.5 hours to cool the chip down completely if it starts out at
room temperature. So you must fill the dewar several hours before twilight.
To avoid having to fill the dewar during the night, we recommend a fill
something during mid-afternoon, then an additional topping off of the 
dewar just before twilight.

After the FILL is complete, the telescope will move back to the position
it was at before the FILL command was issued, and tracking will resume at
this position. Note, however, that this is basically a full slew, and the
the exact previous pointing will \textbf{not} be recovered.

\section{Summary of startup procedure}

\begin{enumerate}
\item Start up the software in the control room from an appropraite directory
using the $tcomm$ command

\item In the dome (or inside), answer the question about initializing the telescope
by requesting a full initialization.

\item Initialize the dome

\item Fill the dewar

\item Open the dome and mirror covers (in that order) about an hour before
twilight, weather permitting.

\item Top the dewar off just before twilight.

\end{enumerate}

\chapter{Operating the telescope}

The telescope can either be operated from the dome or from the control room.
Commands are generally identical from either location, but there are a few
commands which are only available from the dome, and a few which are only
available from the control room.

\section{The status display}

The upper section of the status window gives information about the 
telescope/dome systems. The left column gives the current RA, DEC, 
hour angle of the telescope, coordinate epoch, and position angle.
The next column gives the azimuth, altitude, rotator, and focus positions
for each of the three focus motors. The rightmost column gives the
current UT, local sidereal time, airmass, and the focus position in
a coordinate system which gives mean focus, xtilt, and ytilt.

Other entries give the dome position and status (initialized/unitialized,
slaved/not slaved, open/closed), and mirro cover status (open/closed).

If {\it any} of the items appear in reverse video, there is likely to be
something configured such that you will have significant problems with
your observations (e.g., the dome or mirror covers are closed!).

Several of the items are not encoded, so the status only reports where
the computer {\it thinks} the  item is, which is not necessarily where it
{\it really} is, although if things are working well, the computer should
correctly know where things are.
However, if something seems peculiar, it is worth going out and looking
at the telescope/dome!

\section{The command window}

All commands should be entered in the command window. For all commands, 
you will not get a command prompt until the previous command completes.

The software allows you to write simple scripts in an external file which
can be read in by the program to issue a series of commands automatically.
Input script files should have commands which are identical to what you
would need to type in the command window. To execute a script, you use
the  command:

\begin{quote}
INPUT filename
\end{quote}

where filename is the name of the file with the commands to execute. Our 
convention is to use .inp as the extension for script files.
A script can call another script, up to 5 layers deep. 

\subsection{Restarting tcomm}
If for some reason
things appear to hang up (i.e., nothing returns for several minutes after
you'd really expect it to, you can restart the tcomm programs after issuing
a CTRL-$\setminus$ in the command window. This should kill all of the windows, and
then retyping tcomm should start everything up again. However, if this is
necessary, please inform Jon of the circumstances - we would really like to
get things working without ANY hangups.

\section{Moving the telescope} 

The following commands can be used to move the telescope:

\begin{hanging}
\item {CO [{\it hh mm ss dd mm ss}]}

The CO command will move the telescope to some user-specified coordinates. 
If you do not specify a position on the command line, the program
will prompt for an RA and DEC, and then ask you to confirm the
move. The coordinates will be interpreted using the current epoch,
which defaults to 1950 (the default epoch can be changed using the
NE command which will prompt you for a new epoch).

\item{SA}

Move the telescope to an SAO star. The program will ask whether you
want an SAO near near the current telescope position (T), near
some other specified RA/DEC (O), or some specified ALT/AZ (A). If
you choose one of the latter two options, you will be prompted for
coordinates. You will then be prompted for an acceptable magnitude 
range for the SAO star ($<$CR$>$ to allow any brightness). Note that
the on-line SAO catalog only contains stars down to 7th magnitude
and does not have fainter catalog members at this time.
The program will find the nearest SAO star to your coordinates, and
ask if you want to move there; enter Y to move.

\item{RF [{\it id}]}

Move the telescope to the position of star {\it id} in an open user
coordinate file (see section ??). If not specified on the command line,
{\it id} will be prompted for.

\item{QM [{\it $\Delta\alpha \Delta\delta$}]}

Offset the coordinate by $\Delta\alpha$ arcseconds in right ascension and
$\Delta\delta$ arcseconds in declination. If not specified on the command
line, the offsets will be prompted for.

\item{OFFSET [{\it $\Delta x \Delta y$}]}

Offset the coordinate by $\Delta x$ \textit{pixels} along rows and
$\Delta y$ pixels along columns of the CCD. If not specified on the command
line, the offsets will be prompted for. These allow ``instrument plane''
offsets, i.e., in the coordinate system of the CCD even if the instrument
is rotated with N no longer oriented along columns. Unlike the 3.5m,
the instrument plane offsets are specified in pixels, not arcseconds.

\item{CENTER}

Move the telescope such that a marked image on the CCD frame will be
moved to the center of the CCD frame. This requires that an image has
already been obtained. If you use this command, you will be prompted to mark the
location of an object in the CCD frame; use C to centroid on the position
of the mouse, or I to just take the integer pixel location of the mouse.

\item{LOCATE}

Move the telescope such that a marked image on the CCD frame will be
moved to any another desired location. Again, this requires that an image
has already been obtained. If you use this command, you will be prompted
to mark the location of an object in the CCD frame, then prompted to
mark a second position to which the object will be moved. Mark these
positions using C to centroid at the location of the mouse, 
or I to use the integer pixel location of the mouse.

\item{PA}

Change the position angle to the specified PA. The position angle is
measured from north to vertical on the chip when the image is displayed
in sky orientation (for the PI CCD, this is (1,1) in the upper left).

\end{hanging}

\section{Coordinate epoch command}

The default coordinate epoch on startup in 1950. The default epoch for
coordinate entry can be changed using:

\begin{hanging}
\item{NE}

Change the input epoch to a new value. The program will give the current
epoch, and prompt for a new value. Note that the epoch on the display will
not change until you move to an object using the new epoch.
\end{hanging}

\section{User catalogs}

Users can create catalogs of object positions which can be referenced by
an ID number to maximize efficiency and minimize input error during the
night. The format of the user catalogs is identical to that used by REMARK
on the 3.5m. Each object should be put on a separate line with the following
format:
\begin{quote}
object name $<$TAB$>$ RA (hh:mm:ss) $<$TAB$>$ DEC (dd:mm:ss) $<$TAB$>$ epoch $<$TAB$>$ (proper motion in RA) $<$TAB$>$(proper motion in DEC)
\end{quote}

The following commands can then be used with these files:

\begin{hanging}
\item{OF}

Open a new input file. The file name will be prompted for.

\item{RF [{\it id}]}

Move to object {\it id}. If no {\it id} is given, it will be prompted for

\end{hanging}

\section{Focus commands}

The telescope can be focussed by moving the position of the secondary. 
The secondary position is controlled by three separate motors, so the
mirror can be tilted as well as pistoned. Tilting the mirrors changes
the collimation and thus is not a motion which is allowed to normal users.

The focus commands are:

\begin{hanging}
\item{FOCRUN}

Take a series of exposures at different focus values. You will be prompted
for a starting focus value, a delta focus (amount to change focus between
each exposure), a number of exposures, and an exposure time. After entering
these, the program will automatically take the requested number of exposures.
The program will always approach a given focus value from the same direction,
\textit{assuming you specify a positive delta focus value}.

\item{FO {$f$}}

Move the focus to position $f$. This command will automatically approach
the desired position from the same direction.

\item{DF {$\Delta f$}}

Moves the focus by a relative amount, $\Delta f$. If not specified on the
command line, the amount to move will be prompted for

\item{XTILT $tilt$ (ENGINEERING)}
\item{YTILT $tilt$ (ENGINEERING)}

Tilt the secondary mirror to the specified angle.
\end{hanging}

\chapter{Using the science CCD}

\section{The instrument}

The main science instrument is the Princeton Instruments 1024$\times$ 1024
CCD. This device has 24$\mu$ pixels, given a scale of approximately
$0.81 ^{\prime\prime}$/pixel.

\section{The status display}

\section{The image display window}
The first time an image is taken, the program opens an Image Display
window on the console, and loads a color map as well as displaying the
image.  Once an image has been displayed, the Display window will accept
interaction asynchronously of commands in the command window,
provided that a wait for input, or any other I/O is not pending.  To
interact with the image, simply move the mouse onto the display window.
The current pixel location of the cursor will be displayed in a frame at
the base of the image display along with the pixel intensity.  The arrow
keys are used for find control (one pixel at a time) of the cursor
position.

The following mouse buttons and keyboard keys are active while the mouse
is located on the image display:

\begin{center}
{\bf Mouse Buttons}\\
\begin{tabular}{ll}
\hline
Button & Function\\
\hline
LEFT  &ZOOM IN, centered on the cursor\\
MIDDLE&ZOOM OUT, centered on the cursor\\
RIGHT &PAN, move the pixel under the cursor to the center\\
\hline
\end{tabular}
\end{center}

\begin{center}
{\bf Keyboard Commands}\\
\begin{tabular}{cl}
\hline
Key & Function\\
\hline
 R &RESTORE image to the original zoom/pan\\
 + &BLINK Forwards through the last 4 images.\\
 - &BLINK Backwards through the last 4 images.\\
 P &Find the PEAK pixel near the cursor \& jump the cursor there\\
 V &Find the LOWEST pixel ("Valley") near the cursor \& jump the cursor there\\
 \# &"Power Zoom" zoom at the cursor to the maximum zoom factor\\
 H &Toggle between small and full-screen cross-hairs\\
 ] &Clear boxes and stuff off the image display\\
\hline
\end{tabular}
\end{center}

\noindent{\bf Color Bar Adjustment:}

If you place the mouse on the color bar, these commands are available
to adjust the contrast of the image:
\begin{example}
  \item[LOW CONTRAST]{Hold down the LEFT Mouse button, drag the left
       end of the color bar.}

  \item[HIGH CONTRAST]{Hold down the RIGHT Mouse button, drag the right
       end of the color bar.}

  \item[ROLL COLOR MAP]{Hold the MIDDLE Mouse button, "roll" the
       color bar left or right.}
\end{example}
The position of the mouse cursor displays the range of intensities
represented by that color.

Pressing the R key while the mouse is on the color bar restores the
original color map (undoing any change of the contrast or "roll" changes
made with the mouse buttons).


\section{Commands}

The following commands are available:
\begin{hanging}
\item{EXP [$time$]}

Take an exposure of length $time$ seconds.

\item{FILTER [$n$]}

Change the filter wheel to position $n$. In the current state, the filters
are in the following locations: U(1), B(2), V(3), R(4), I(5)

\item{FILENAME [$name$]}

Change the default root file name to $name$.

\item{NEWEXT [$ext$]}

Change the extension for the next file to be the number $ext$.

\item{FITS}

Set filetype for future image stores to be FITS.

\item{-DISK}

Turn off autosaving of images. Turn this back on using FITS.

\item{+DISPLAY/-DISPLAY}

Turn on/off autodisplay of images after they are taken.

\item{SCALE $low\ high$}

Redisplay the current image with greyscale scaling between $low$ and $high$.

\item{SKYSCALE}

Sets mode for autoscaling to have black level somewhat below mean level in
image, and white level above it. This is the default.

\item{FULLSCALE}

Sets mode for autoscaling of images to be from minimum pixel (black) to
maximum pixel (white).

\item{SAMESCALE}

Sets scaling for future pixels to be identical to the current scale.

\item{CENTER}

Move the telescope so that an object/location marked on the display window
will move to the center of the CCD field. You will be prompted to mark the
location of an object in the CCD frame; use C to centroid on the position
of the mouse, or I to just take the integer pixel location of the mouse.

\item{FLAT \textit{ndesired nmax startexp fudge}}

Takes a series of flat fields, adjusting exposures times to maintain good
S/N, for use in taking twilight flats. A ``good'' flat is defined as one
with a mean level (over bias) of somewhere between 5000 and 20000 DN, with
a minimum exposure time of 1 second. The
routine will start out by taking an exposure of length \textit{startexp}.
From this, it will compute a new exposure time which is required to obtain
a mean of 10000, after multiplying by the fudge factor \textit{fudge},
which accounts for the changing brightness of the twlight sky (\textit{fudge}
should be $\sim 1.5$ for evening twilight and arounge $\sim 0.7$ for
morning twilight. In no case will the program let the exposure time be longer
than 30 seconds, and any 30 second flat will be counted as a ``good'' flat
to prevent a large number of unsuccessful flats being taken when it is too
dark.  The program will continue taking flats until it gets
\textit{ndesired} ``good'' flats or until it has taken ``nmax'' flats.


\end{hanging}

\section{Standard star observations}

Several commands are available to increase your efficiency in observing
standard stars. There is a full use catalog of all of the Landolt standards,
plus several others, in the top level observe directory, with filename
standards.apo. To allow use of this catalog in conjuction with your own
catalog of objects, there are separate commands to open and read from the
standards file, so you can have both the standards file and a user file open
simultaneously. The standards file also includes the UBVRI magnitudes of
the stars, and the observing program knows about the throughput of the
1m telescope, so the program is also capable of choosing good exposure
times for you, freeing you from having to take the time of choosing
exposure times (and making bad choices), and assuring uniform exposure
levels of your different standards

The relevant commands are:

\begin{hanging}

\item{OS}

Open the standards file (analagous to OF for user files). 
If you are running in a subdirectory under the
top level observe login directory, you will enter: ../standards.apo

\item{RS [{\it id}]}

Move to standard {\it id}. If no {\it id} is given, it will be prompted for.

\item{STAN [\textit{filtnum}]}

Take an exposure of the current standard through filter \textit{filtnum}. You
must have already moved to the star using the RS command. The program will move
the filter automatically and will take an exposure based on the known
magnitude of the star and the throughput of the system.

\item{FUDGE [\textit{fudge}]}

Sets a fudge factor for the automatic expossure time calculation of the STAN
commands. All standard star exposure times will be multiplied by the fudge
factor specified. 

\item{MAG m1 m2 m3 m4 m5 m6}

\textit{Do not use this command unless instructed to - defaults should be
correct!)}
Tell the program which magnitude to use for the exposure time calculation
for each filter. There are 6 filter slots, so 6 integers must be specified.
The possible choices for the magnitude codes are 1, 2, 3, 4, or 5, corresponding
to U,B,V,R,or I. So, for example, if UBVRI are loaded in slots 12345, you
would use the command MAG 1 2 3 4 5 0. The correct defaults are loaded at
startup.

\item{ZERO z1 z2 z3 z4 z5 z6}

\textit{Do not use this command unless instructed to - defaults should be
correct!)}
Tell the program which zeropoints to use for the exposure time calculation
for each filter. There are 6 filter slots, so 6 zeropoints must be specified.
The zeropoint gives the magnitude which gives 1 DN/second through the
desired filters.

\end{hanging}

\chapter{Using the guider CCD}

In general, you can issue the same commands to the guider CCD as you
can to the science CCD. To do so, use the same command names but preceeded
by the letter ``G'', e.g., GEXP \textit{t} will take a guider exposure
of length \textit{t} seconds.

To minimize errors from inaccuracies in the pointing model, you might
go to an SAO star near you object, take an exposure, do a CENTER command,
and then update the telescope coordinates with a UC command before moving
to your object and starting to guide. 

To start guiding, first take a guider exposure to see if there is an
acceptable guide star and to determine a good exposure time. Ideally, you
will have a star which has good S/N with an exposure time of less than
or approximately equal to 1 second. Once you have a picture taken with
a good exposure time, you can start guiding by issuing the GUIDE command.
This will take a picture with your exposure time, ask you to mark the
guide star in the guide CCD image, and will then start autoguiding on
this star. 

Guiding will continue until you do a slew with the telescope or unless
you issue a GUIDEOFF command.

Other commands relevant to guiding:

\begin{hanging}

\item GUPDATE \textit{n}

Sets the number of guide exposures to average centroids from before sending
a position update command to the telescope. Default is 5.

\item GSIZE \textit{n}

Sets the size of the box used for computing centroids. Default is 11 pixels.

\item NOSHUTTER

Puts guider in NOSHUTTER mode where exposures are taken without using the
shutter. You must remember to issue the OPEN command to open the shutter
before starting to guide! 
Currently, this command must be issued from the
guider CCD control window (not the command window).

\item SHUTTER

Puts guider into normal SHUTTER mode.
Currently, this command must be issued from the
guider CCD control window (not the command window).

\item OPEN

Opens the guider shutter.
Currently, this command must be issued from the
guider CCD control window (not the command window).

\item CLOSE

Closes the guider shutter.
Currently, this command must be issued from the
guider CCD control window (not the command window).

\item WRITE

Toggles mode where each individual guide exposure is saved to disk. Default
is \textit{not} to save each image.
Currently, this command must be issued from the
guider CCD control window (not the command window).

\end{hanging}

\chapter{Shutting down}

To close the telescope down, you need to move the telescope to its stow
position, close the mirror covers and dome, and exit the telescope control
program gracefully. {\it Do not power down the telescope control computer
without first exiting the program gracefully - if you do so, you will not
be able to start it up again!}. The following commands are used:

\begin{hanging}
\item{CM}

Closes the mirror covers. The telescope will automatically move to be 
pointing nearly vertically, since this reduces stress on the mirror cover
motors. 

\item{ST}

Stows the telescope in the default stow position.

\item{CD}

Closes the dome.

\item{QU}

Quits the program. When you enter QU from the command window in the control 
room,  the program will ask you if you wish to quit the telescope control
program in the dome as well as the remote programs. Generally, you should
answer Y to this question; doing so, will require that you restart the
dome telescope program (by rebooting the telescope control PC) and reinitialize
the telescope. If you will be restarting tcomm soon and do not want to be
forced to reinitialize, you can enter N. However,
\textit{Do not kill the power to the telescope control computer in the dome
until you have gracefully exited the program running there, or else you
will not be able to get started again!.}
If you enter QU on the TOCC PC in the dome, you can quit the program running
in the dome.

Finally, the program will ask if you wish to kill power to everything in
the dome. If you have quit the telescope control program in the dome, you
can answer Y here; if not, answer no.

\end{hanging}

\appendix

\chapter{Command summary}

\input commands

\chapter{Standard stars }

For user convenience, a master coordinate/standards file with a large number 
of standard stars is available which can be used with the OS/RS and STAN
commands. OS/RS allow ease of pointing to a desired standard, and the STAN
command can be used to take an exposure, letting the software calculate a
good exposure time based on the standard magnitude and the system
throughput through a desired filter.

The master standards file is called standards.apo, and resides in the top
level observe directory (/home/loki2/observe/standards.apo). The current
version contains 918 entries. Entries 1-223 come from Landolt 1982,
224-250 are dummy entries, 251-280 are standards in M67 from Montgomery,
281-300 are dummy entries, 301-826 are from Landolt 1992 (some of which
duplicate entries from Landolt 1982), 827-900 are
dummy entries, 901-918 are southern spectrophotometric standards
from Landolt 1992a.

An abbreviated list of recommended single star standards is presented below.
These stars were chosen to be at a good brightness for the 1m, sample
stars around the sky, with care given to include a red and blue star at
all right ascensions. Stars marked with an asterisk are especially preffered.

The ID numbers are the numbers by which the star is referred to in the
standards.apo file, i.e., you should do a RS id to move to a given
star.

\pagebreak
\begin{tabular}{lllllll}
 &  ID&    RA    &      DEC  &  Epoch&     V  &     B-V    \\
$\ast$&  3&00:37:36.00&-15:04:51.0&1985.0 & 10.881 & -0.190  \\
$\ast$&324&00:55:14.00&+00:56: 7.0&2000.0 & 10.627 &  1.138  \\
 &  4&00:46:19.00&-11:57:32.0&1985.0 & 11.775 & -0.294  \\
$\ast$&311&00:53:16.00&+00:36:18.0&2000.0 & 10.595 &  1.638  \\
 &322&00:55:10.00&+00:43:14.0&2000.0 & 11.613 &  0.436  \\
 &333&00:55:40.00&+00:36:18.0&2000.0 & 11.782 &  1.048  \\
\multicolumn{7}{l}{ }\\
$\ast$& 21&01:57:10.00&-02:10:37.0&1985.0 &  8.874 & -0.082  \\
$\ast$&356&01:55:26.00&+00:56:43.0&2000.0 & 11.620 &  1.083  \\
 &352&01:54: 8.00&-06:42:54.0&2000.0 & 12.406 & -0.012  \\
$\ast$& 15&01:54: 4.00&+00:42:40.0&1985.0 &  9.569 &  0.454  \\
\multicolumn{7}{l}{ }\\
$\ast$& 29&02:56:54.00&-02:03:26.0&1985.0 & 10.307 & -0.104 \\
$\ast$&374&02:57:46.00&+00:16: 2.0&2000.0 & 11.204 &  1.219 \\
 &361&02:30:17.00&+05:15:51.0&2000.0 & 12.799 & -0.054 \\
 &375&02:58:13.00&+01:10:53.0&2000.0 & 11.594 &  1.418 \\
 &373&02:57:21.00&+00:18:38.0&2000.0 & 11.728 &  0.301 \\
\multicolumn{7}{l}{ }\\
$\ast$&381&03:52:54.00&+00:00:19.0&2000.0 & 10.010 &  0.147\\
 &420&03:56:13.00&+00:08:43.0&2000.0 & 11.491 &  0.736\\
$\ast$&415&03:55:31.00&-00:09:13.0&2000.0 & 11.531 &  1.126\\
$\ast$& 36&03:53:28.00&-00:10:58.0&1985.0 &  9.574 &  0.529\\
\multicolumn{7}{l}{ }\\
$\ast$&422&04:51:43.00&-00:10:12.0&2000.0 & 10.591 &  0.247 \\
$\ast$&426&04:53:19.00&-00:05: 4.0&2000.0 & 11.140 &  1.074 \\
 &425&04:52:59.00&-00:14:44.0&2000.0 & 11.719 &  0.179 \\
 &423&04:52:35.00&+00:22:29.0&2000.0 & 11.716 &  1.334 \\
$\ast$& 43&04:51:44.00&+00:00:39.0&1985.0 &  9.652 &  0.598 \\
\multicolumn{7}{l}{ }\\
$\ast$&433&05:57:37.00&+00:13:42.0&2000.0 &  9.781 &  0.202 \\
 & 51&05:56:41.00&+00:13:17.0&1985.0 &  9.260 &  0.594 \\
 &429&05:52:28.00&+15:53:15.0&2000.0 & 13.032 & -0.249 \\
$\ast$&435&05:58:25.00&+00:05:12.0&2000.0 & 10.788 &  1.363 \\
 &434&05:57:55.00&-00:09:29.0&2000.0 & 11.483 &  1.872 \\
\multicolumn{7}{l}{ }\\
$\ast$&462&06:52: 5.00&-00:18:19.0&2000.0 &  9.539 & -0.004 \\
$\ast$&461&06:52: 4.00&-00:27:18.0&2000.0 & 10.030 &  1.180 \\
 &443&06:51:34.00&-00:11:28.0&2000.0 & 10.572 &  0.609 \\
 &459&06:52: 2.00&-00:27:21.0&2000.0 & 10.536 &  0.202 \\
 &467&06:52:12.00&-00:19:17.0&2000.0 & 11.930 &  1.356 \\
\multicolumn{7}{l}{ }\\
$\ast$&498&07:55:54.00&-00:16:51.0&2000.0 &  9.398 & -0.155 \\
 & 69&07:53:26.00&-00:23:13.0&1985.0 & 11.149 &  1.005 \\
 & 73&07:54:41.00&-00:15: 7.0&1985.0 &  9.474 & -0.041 \\
$\ast$& 65&07:26:36.00&+05:16:16.0&1985.0 &  9.843 &  1.557 \\
 & 72&07:54:27.00&-00:23: 8.0&1985.0 &  9.807 &  0.407 \\
\multicolumn{7}{l}{ }
\end{tabular}
\pagebreak

\begin{tabular}{lllllll}
 &  ID&    RA    &      DEC  &  Epoch&     V  &     B-V    \\
$\ast$&500&08:52:35.00&-00:39:48.0&2000.0 & 10.139 &  0.157 \\
$\ast$&505&08:53:55.00&-00:32:21.0&2000.0 & 11.384 &  1.317 \\
 & 79&08:52:12.00&-00:06: 4.0&1985.0 &  8.641 &  0.052 \\
 &504&08:53:36.00&-00:36:41.0&2000.0 & 11.799 &  0.494 \\
 &501&08:53:15.00&-00:43:29.0&2000.0 &  9.150 &  1.276 \\
\multicolumn{7}{l}{ }\\
$\ast$&551&09:58:19.00&-00:25:35.0&2000.0 &  9.874 &  0.261 \\
$\ast$&536&09:57: 5.00&-00:31:38.0&2000.0 & 11.575 &  0.812 \\
$\ast$&511&09:31:18.00&-13:29:20.0&2000.0 & 10.067 &  1.501 \\
 & 85&09:47:59.00&-02:38:29.0&1985.0 &  8.636 & -0.159 \\
\multicolumn{7}{l}{ }\\
$\ast$&100&10:54:32.00&-01:20:38.0&1985.0 &  9.380 &  0.060 \\
$\ast$&562&10:55: 6.00&-00:48:19.0&2000.0 & 10.069 &  1.083 \\
 &561&10:50:54.00&+06:48:57.0&2000.0 & 11.675 &  1.644 \\
 &564&10:57: 4.00&-00:13:10.0&2000.0 &  9.903 &  0.664 \\
\multicolumn{7}{l}{ }\\
 &566&11:08: 0.00&-05:09:26.0&2000.0 & 13.059 &  0.035 \\
$\ast$&107&11:32: 3.00&+05:21:40.0&1985.0 & 10.117 & -0.242 \\
$\ast$&112&11:56: 8.00&-00:25:13.0&1985.0 & 10.903 &  1.089 \\
$\ast$&106&11:14:46.00&+05:02:20.0&1985.0 &  9.348 & -0.116 \\
 &568&11:47:44.00&+00:48:55.0&2000.0 & 11.153 &  1.752 \\
 &569&11:56: 6.00&-00:47:54.0&2000.0 &  9.861 &  0.368 \\
\multicolumn{7}{l}{ }\\
$\ast$&595&12:43: 7.00&-00:32:21.0&2000.0 &  9.705 &  0.476 \\
$\ast$&605&12:45:17.00&-00:16:37.0&2000.0 & 11.479 &  1.106 \\
 &114&12:41:42.00&-00:29:21.0&1985.0 & 11.207 &  0.768 \\
\multicolumn{7}{l}{ }\\
 &615&13:42:21.00&+01:30:17.0&2000.0 & 10.367 & -0.166 \\
$\ast$&124&13:38:49.00&-01:08:53.0&1985.0 &  9.426 &  0.977 \\
 &616&13:58:52.00&-02:55:12.0&2000.0 & 10.254 & -0.186 \\
$\ast$&612&13:35:25.00&-00:23:38.0&2000.0 & 10.270 &  1.422 \\
 &614&13:40: 4.00&-00:02:10.0&2000.0 & 11.453 &  0.385 \\
 &606&13:28:22.00&-02:21:28.0&2000.0 & 11.336 &  1.491 \\
\multicolumn{7}{l}{ }\\
$\ast$&133&14:43:28.00&-00:33:20.0&1985.0 &  9.484 &  0.380 \\
 &128&14:38:43.00&-00:11: 0.0&1985.0 &  9.088 &  0.701  \\
$\ast$&619&14:40:52.00&-00:23:36.0&2000.0 &  9.785 &  1.362  \\
\multicolumn{7}{l}{ }\\
$\ast$&135&15:36: 3.00&-00:12:14.0&1985.0 &  9.037 &  0.401  \\
$\ast$&657&15:40:17.00&-00:21:13.0&2000.0 & 11.311 &  1.237  \\
 &139&15:37:48.00&+00:17:16.0&1985.0 & 11.715 &  0.764  \\
\multicolumn{7}{l}{ }
\end{tabular}
\pagebreak

\begin{tabular}{lllllll}
 &  ID&    RA    &      DEC  &  Epoch&     V  &     B-V    \\
 &145&16:33:36.00&-03:58:59.0&1985.0 &  8.944 & -0.281  \\
 &146&16:34:35.00&-00:02:18.0&1985.0 &  9.199 &  0.384  \\
 &149&16:36:28.00&-00:00:54.0&1985.0 &  9.059 &  0.965  \\
$\ast$&673&16:37:47.00&-00:33: 6.0&2000.0 & 10.703 &  0.179  \\
$\ast$&671&16:37: 0.00&-00:34:40.0&2000.0 & 11.309 &  1.380  \\
 &674&16:37:50.00&-00:00:37.0&2000.0 & 11.384 &  1.432  \\
\multicolumn{7}{l}{ }\\
$\ast$&154&17:37:56.00&+04:21:19.0&1985.0 &  9.095 & -0.054  \\
$\ast$&690&17:45:42.00&-00:21:34.0&2000.0 & 10.353 &  0.609  \\
$\ast$&682&17:05:15.00&-05:05: 5.0&2000.0 & 10.071 &  1.415  \\
 &688&17:45: 2.00&-00:29:28.0&2000.0 & 10.990 &  1.739  \\
 &683&17:44: 6.00&-00:24:59.0&2000.0 & 11.493 &  0.323  \\
\multicolumn{7}{l}{ }\\
$\ast$&170&18:45:50.00&-07:56:56.0&1985.0 &  9.375 &  0.236  \\
$\ast$&727&18:43:52.00&+00:22:58.0&2000.0 & 11.585 &  0.944  \\
$\ast$&698&18:41:29.00&+00:15:22.0&2000.0 & 10.025 &  0.303  \\
 &730&18:43:55.00&+00:02: 1.0&2000.0 & 11.861 &  1.309  \\
 &724&18:43:34.00&+00:19:40.0&2000.0 & 11.121 &  0.555 \\
 &703&18:42:19.00&+00:08:24.0&2000.0 & 11.944 &  1.023 \\
\multicolumn{7}{l}{ }\\
$\ast$&174&19:36:19.00&+00:35:31.0&1985.0 &  9.706 &  0.164 \\
 &178&19:37: 9.00&+00:24:24.0&1985.0 & 10.608 &  0.889 \\
$\ast$&731&19:37:16.00&+00:12: 5.0&2000.0 & 10.744 &  1.738 \\
 &177&19:36:58.00&+00:23:44.0&1985.0 & 10.384 &  1.961 \\
 &733&19:37:29.00&+00:25: 1.0&2000.0 & 12.388 &  0.395 \\
\multicolumn{7}{l}{ }\\
$\ast$&744&20:42:46.00&+00:16: 8.0&2000.0 & 12.086 &  0.152 \\
$\ast$&743&20:42:36.00&+00:07:21.0&2000.0 &  9.905 &  1.210 \\
 &739&20:41:18.00&+00:16:28.0&2000.0 & 11.352 &  1.601 \\
 &741&20:42:14.00&+00:09: 1.0&2000.0 & 11.424 &  0.454 \\
 &745&20:42:55.00&+00:15: 4.0&2000.0 & 11.549 &  1.031 \\
\multicolumn{7}{l}{ }\\
$\ast$&764&21:41:28.00&+00:40:14.0&2000.0 & 10.004 &  0.454 \\
$\ast$&767&21:41:51.00&+00:39:19.0&2000.0 & 10.306 &  1.058 \\
$\ast$&194&21:35:11.00&+05:25:11.0&1985.0 &  8.301 & -0.054 \\
 &765&21:41:44.00&+00:17:37.0&2000.0 & 11.742 &  1.194 \\
\multicolumn{7}{l}{ }\\
 &215&22:49:41.00&-13:23:12.0&1985.0 & 10.112 & -0.115 \\
$\ast$&806&22:41:45.00&+01:12:38.0&2000.0 & 11.916 & -0.041 \\
$\ast$&807&22:42:10.00&+01:10:17.0&2000.0 & 11.101 &  1.206 \\
 &805&22:41:37.00&+00:59: 7.0&2000.0 & 11.601 &  1.362 \\
 &209&22:41:22.00&+01:12: 6.0&1985.0 & 10.908 &  0.571 \\
\multicolumn{7}{l}{ }\\
$\ast$&823&23:42:36.00&+01:05:58.0&2000.0 & 11.161 &  0.468 \\
 &824&23:42:41.00&+00:45:10.0&2000.0 &  9.695 &  0.615 \\
$\ast$&825&23:44:15.00&+01:14:13.0&2000.0 & 10.434 &  1.028 \\
\multicolumn{7}{l}{ }
\end{tabular}

\chapter{Transferring files to NMSU}

\chapter{Routine (weekly) maintenence}

\chapter{Long term maintenance}

\include{refs}
\printindex

\end{document}
