\documentclass{article}[12pt]

\usepackage{epsfig}
\usepackage{html}

\textwidth=8in
\textheight=10in

\hoffset=-1.7in
\voffset=-1in

\title{NMSU 1 meter telescope: design and performance}
\author{Jon Holtzman, NMSU}

\epsfxsize 7in 

\begin{document}

\maketitle

\vskip 0.5in

Complete document in a 
\htmladdnormallink{single PostScript file}{../1m.ps.gz}.

\section{Telescope: optical/mechanical design}

\begin{itemize}
\item Telescope is an alt-az Ritchey-Chretien design and produces an F/6.06 
beam.

\begin{latexonly}
\epsfbox{1mlayout.eps}
\end{latexonly}
\begin{htmlonly}
\begin{rawhtml}
<IMG SRC=1mlayout.gif WIDTH=100%>
\end{rawhtml}
\end{htmlonly}

\item Basic design optical parameters (units are inches)

The telescope mirrors were refigured in 2002 because we strongly suspected that
the surfaces were not particularly good (which was confirmed by optical
testing before repolishing). During repolishing the followin optical 
prescription was adopted:

\begin{center}
\begin{tabular}{lcccc}
NAME & RADIUS & THICKNESS & DIAMETER & CONIC \\
Primary & -200 & -62.1 & 40 & -1.202461 \\
Secondary & -128.8185 & 47.50 & & -8.38632 \\
Tertiary & Inf & 44.784 & & \\
Focal plane & & & 
\end{tabular}
\end{center}

The following discussion describes how this prescription was arrived at.

\item For the original optical design, our best idea of the final prescription
is taken from a document provided by Kurt Anderson (values labelled
``Preferred'' for 20 Dec 1993):

\begin{center}
\begin{tabular}{lcccc}
NAME & RADIUS & THICKNESS & DIAMETER & CONIC \\
Primary & -200 & -62.648 & 40 & -1.202461 \\
Secondary & -127.059 & 49.00 & & -8.434993 \\
Tertiary & Inf & 41.648 & & \\
Focal plane & & & 
\end{tabular}
\end{center}

We have essentially no information as to how the as-built
optical parameters match these design parameters.

When we built our new guide box, our best measured distance from tertiary 
to rotator mount was 30 inches (however, see below!).
Our old guide box/filter wheel had a total thickness of 9.875 inches, for
a total tertiary to focus distance of 39.875 inches.

Our new guide box/filter wheel was designed to increase the tertiary to
focus distance by a couple of inches to better match the design 
prescription (under the assumption that the primary-secondary separation
matched the design value, see below). The guide box has a thickness of
9.25 inches, the filter wheel, 1.313 inches, and our best measurement
for the optical distance of the Princeton CCD from the front mounting surface is
1.125 inches, for a total tertiary to focus distance of 41.688 inches.

The prescription I was then using for ray tracing in Zemax is:

\begin{center}
\begin{tabular}{lcccc}
NAME & RADIUS & THICKNESS & DIAMETER & CONIC \\
Primary & -200 & -62.66 & 40 & -1.202461 \\
Secondary & -127.059 & 48.6585 & & -8.434993 \\
Tertiary & Inf & 30 & & \\
Rotator mount & & 11.93 & \\
Focal plane & & & 
\end{tabular}
\end{center}

where the focal plane location and primary-secondary separation were 
allowed to float to get best image quality.

We later switched to using an Apogee AP7P camera, which has the CCD 
1.10 inches behind the front mounting surface (as measured by microscope);
we added an adapter plate of width .375 inches (for a reason that
is not totally recalled!).

\item Measured separations. Several (re)measurements of separations were
made in 2002 in preparation for mirror repolishing. 

Several of these measurements were made from the edge of the primary
to the center of another element (tertiary or secondary). To correct for
the distance to the primary vertex requires a correction from the hypotenuse
of a triangle to the on-axis distance, plus a correction for the vertical
distance between the primary vertex and the edge of the mirror. For
the hypotenuse correction, a diameter of the front surface of the mirror
was measured (approximately) to be 41 inches; note that using a smaller
diameter leads to some inconsistencies in measurements (see JCB 6/02 and
JH 6/02). The vertex ``sag'' was computed using a radius of curvature
of 200 inches, resulting in a 1 inch sag.

JCB 4/02: measured distance
from primary edge to secondary center as 166.2+/-0.3 cm (65.43 inches); 
correcting for hypotenuse gives (62.14+/- 0.12 inches). Distance from
vertex to "edge center" adds another inch based on radius of curvature,
to give 63.14. Using geometric center of tertiary, measured secondary-tertiary
to be 1227 cm (48.31 inch). Tertiary to rotator mount measured outside
of bearing to be 79.4cm (31.25"), rotator mount to filter mount 27.3 cm
(10.75")

JH/DS 5/20: measure primary-secondary with tertiary removed. Method 1:
mirror to mirror cover lip 2", lip to tertiary mount 5", tertiary mount
to secondary center 55.5", total 62.5 + distance from primary vertex to
spot on primary out at edge of central hole. Method 2: secondary center
to primary edge (as JCB) gives 65", correcting for hypotenuse gives 61.68;
add another inch as above to get 62.68. Using plumb line to determine tertiary
center, get 47.56 " secondary-tertiary. Tertiary to CCD mount measured
through the hole to give 42.185"

JCB 6/02: distance from primary edge to tertiary center 24.75 inches,
corrected for hypotenuse gives 13.87, plus 1 for sag gives 14.87 inches.
Distance from tertiary to secondary is 47.71. Distance from primary edge
to secondary center is 64.81, corrected for hypotenuse is 61.48, plus 1
for sag is 62.48. (Note for consistency check: $14.87+47.71=62.58$)
Tertiary to CCD mount is 42.19 inches. CCD mounting plate measured
to be .375 inches. Apogee web site gives 1.161 inches for distance from
front of camera to CCD.

JH 6/02: distance from primary edge to tertiary center 24.625 inches,
corrected for hypotenuse gives 13.64, plus 1 for sag gives 14.64 inches.
Distance from tertiary to secondary is 47.875. Distance from primary edge
to secondary center is 64.875, corrected for hypotenuse is 61.55, plus 1
for sag is 62.55. (Note for consistency check: $14.64+47.875=62.515$)

JH/JCB 6/02: microscope gives distance from front window of Apogee to
CCD surface is 0.802 inches. Micrometer give 0.682 inches from front
mounting surface to front window. Mounting plate thickness 0.375 inches.

Summary (numbers in parentheses are derived from other measurements) :

\begin{center}
\begin{tabular}{lllllll}
distance&design&Zemax&JCB 4/02&JH/DS 5/02&JCB 6/02&JH 6/02\\
primary-secondary&62.648&62.66&63.14&62.5+,62.68&62.48&62.55\\
secondary-tertiary&49&48.66&48.31&47.56&47.71&47.875\\
primary-tertiary&(13.648)&(14)&(14.83)&(15.12)&14.87&14.64\\
tertiary-rotator mount&&&31.25&(31.622)&(31.627)&(31.622)\\
rotator mount-filter mount&&&10.75&10.563(design)&10.563&10.563\\
tertiary-filter mount&&&(42)&42.185&42.19&42.185\\
CCD mounting plate thickness&&&&0.375&.375&.375\\
front of camera to CCD&&&&1.161&1.109&1.109\\
tertiary-focal plane&41.648&41.93&(43.54)&(43.72)&(43.67)&(43.67)\\
(secondary-focal plane)&90.648&90.59&(91.84)&(91.28)&(91.38)&(91.545)
\end{tabular}
\end{center}

Adopt as-built:

\begin{center}
\begin{tabular}{ll}
primary-tertiary&14.8\\
tertiary-secondary&47.875\\
tertiary-filter mount&42.185\\
mounting plate&.375\\
camera surface-CCD&1.11
\end{tabular}
\end{center}

In measured configuration, secondary has range to move approximately 0.125
inch away from primary, and 1 inch towards primary. Primary support could 
be modified to move primary towards secondary (e.g. increase pad thickness),
but not necessarily further away.

For new design, modify distances conservatively given freedom to adjust
primary towards secondary, secondary towards tertiary, and CCD away from
tertiary:

\begin{center}
\begin{tabular}{ll}
primary-tertiary&14.6\\
tertiary-secondary&47.5\\
tertiary-filter mount&42.2\\
mounting plate&.375\\
camera surface-CCD&1.11
\end{tabular}
\end{center}

\item sizes of optics. JCB 4/02 measured secondary diameter 43.8+/- 0.3 cm
(17.24 +/- 0.12 inches); secondary obscuration 45.7+/- 0.3 cm 
(18+/-0.12 inches). Primary mirror is 41 inches diameter, center hole 
approximately 13.5 inches diameter, thickness is 6 inches

\item Primary mirror support: there is a aluminum "stovepipe" that is
bolted to the back frame. It has anout diameter of 11.982-11.984 inches
as measured by Rick from Sunspot. There is a blue plastic ring around
this that is approximately 1 cm thick and 6 inches long. The mirror
fits over this ring. The mirror rests on the ring on just the front and
back faceplates, as the central part of the mirror is recessed. On the
front plate, the hole is somewhat oversized when compared to the outer
diameter of the plastic; measurements with shims gave the space between
the mirror and the plastic as less than 0.050" everywhere, $>$0.001" 
from 1-9:00, $>$0.015" from 2-8:45, $>$0.020" from
2-8:00, $>$0.025" from 2-7:00, $>$0.030" from 2-6:30, and $>$ 0.040" from
3-6:00, and had a maximum spacing of about 0.045" at 5:00, when 
measured in April 2004 (i.e. the mirror was shifted a bit to the right).

In early May 2004, we pulled the primary and removed the plastic sleeve.
The sleeve was smoothed at the Sunspot machine lab, so the tolerances
are now even greater than before. In addition, the sleeve was unable to
go all the way to the bottom of the stovepipe because of the weld there, so
the inner bottom part of the sleeve was ground away to allow the sleeve
to go to the bottom, and hopefully, not get pushed around by thermal expansion
at the weld.
\item Rotator mounting details: 12(?) hole bolt pattern with 1/4-20 bolts 
spaced evenly on a circle 14 inches diameter. From the center of the
rotator, you can mount anything out to a radius of 12 inches. Once you
are back 6 inches from the rotator, there is some additional clearance;
by the time you are 8 inches from the rotator, there are no constraints
on a mounted instrument.

\item Filter wheel mounting details.

\item Alt-az-rotator system has 3 axes of motion; at current time, rotator
exists on only one of the Nasmyth ports. Altitiude and azimuth axes have
limit switches and magnetic home sensors. Rotator axis does not have
limit switches, only a magnetic home sensor; to avoid rotator cable wrap
problems, the rotator position must be carefully stored using software
and remembered from session to session.

\item All drives are friction drives. Altitude and azimuth have friction
driven encoders as well; rotator only has the motor drive. On the
azimuth axis, there is clear evidence that the ``scale'' of the axis,
degrees/mm, varies slowly with temperature despite the fact that the
material of the drive shaft and drive wheel are the same. We have plans
to implement a temperature-dependent scale to correct for this to
first order.

\end{itemize}

\section{Instruments}

\subsection{Current guide box/filter wheel}

We had a new guider and filter wheel built by
\htmladdnormallink{Astronomical Consultants and Equipment}
{http://www.astronomical.com} (Peter Mack) out of Tucson. This was installed
in mid-1999.
The idea was to make a much thinner filter
wheel which will allow the guider to be located much closer to the
focal plane. This allows the guider to come closer on-axis without
vignetting the primary beam to the science instrument. This is motivated
so we can maximize the size of baffles in the system to minimize scattered
light, which currently poses a problem. 

The new guider consists of a guide box which is 9.25 inches thick, 
followed by a thin (1.313 inch) filter wheel, followed by the CCD.
The guider module plugs into the side of the guider box, and uses our
SpectraSource CCD. The guide module has a pickoff mirror to direct light
to the guide camera. The entire pickoff/camera assembly is on a radial
stage; in addition, the pickoff mirror is on a separate stage to allow
focus control.

\subsection{Old guide box}

\begin{latexonly}
\epsfbox{1m.eps}
\end{latexonly}
\begin{htmlonly}
\begin{rawhtml}
<IMG SRC=1m.gif WIDTH=100%>
\end{rawhtml}
\end{htmlonly}

Old design had a spacer box mounted on rotator mount, which has a
back mounting plate 5.25" behing the rotator mount. An off-axis guider
camera is located inside this box, mounted to the back plate.  This
mount holds a 1.25" diagonal mirror along with a SpectraSource 512x512
CCD camera with 20 micron pixels.  The center of the guider is currently
located $\sim$ 2300 arcseconds off axis.  However, at the current time,
the diagonal mirror housing partially vignettes the beam going to the
science instrument.  There is currently no corrector in front of the
guider.

Behind the spacer, a filter wheel is mounted. The current filter wheel
is 3.5" thick. It accomodates 6 2" square filters. 

Behind the filter wheel, the science CCD camera is located. It has a
focal plane which is located 1.25 " (optical distance) behind the
mounting surface. The science CCD is 1024x1024 array put together 
by Princeton Instruments. The CCD pixels are 24 microns square.

The old design placed the CCD array about 2 inches closer to the
secondary than is suggested by ray-tracing using the nominal optical
parameters.

\subsection{Apogee CCD camera}

We are currently operating using an APOGEE AP7p camera as the science
imager. This has a 512x512 thinned backside illuminated SITe chip  with
24$\mu$ pixels
in it. It is a thermoelectrically cooled camera which can operate about
50 degrees below ambient, leading to operating temperatures between
-50 and -30 C, depending on the season. Dark current is significant
at the warmer temperatures.

\subsection{Princeton CCD camera}

We have a Princeton Instruments CCD camera which has a 1024x1024
SITe array with 24$\mu$ pixels. This is a liquid nitrogen cooled
system. We had a commandable fill system built for this dewar by 
VBS industries (see systems documentation).

The PI camera was purchased in the early 1990s. We have had a fair 
amount of trouble with low level noise in the camera, and it has
been sent back to the manufacturer several times. Princeton Instruments
has been taken over by Roper Scientific.

\section{Computer and software control}

The telescope, science CCD and filter wheel, and guider CCD are each
controlled by a separate PC. The telescope is controlled by the TOCC
computer, running DOS, the science CCD by the PI computer running 
Windows 3.11, and the guider CCD by the SPEC computer running DOS.

An original version of the alt-az control system running on the TOCC
computer was written by the AutoScope corporation, but this had many
problems and large portions of it have been rewritten (by Jon Holtzman) 
to form the current version. This software also controls the dome
and dome shutter, as well as commandable fill system for the CCD dewar.

The science CCD is controlled by software provided by Princeton Instruments.

The guide CCD is controlled by software written by Jon Holtzman using
library routines provided by SpectraSource.

All three programs look, in addition to input from the local keyboard(s),
for input in specified files which are visible over the network. This
allows for complete remote operation.

For remote operation, a suite of programs (tcomm) is run from a UNIX
workstation. Basically, a program is run which talks with each of the
three computers in the dome. In addition, there is a master command
program which is the user interface and sends commands to the UNIX client
programs. There is also a status program which monitors status files sent
back from the telescope and the CCD programs and displays a status
window with the current pointing of the telescope, exposure/filter 
information, etc.  Finally there is a program which controls the power
to various devices in the dome via a network power switching device.

\subsection{Notes on pointing software}

\begin{htmlonly}
See postscript version; HTML version is garbled!
\end{htmlonly}
\begin{latexonly}
We define the rotator coordinate system to be a right-handed coordinate
system centered on the rotator center, with axes which will be defined
to run parallel to NS and EW when the rotator position angle is set to 0.
Call rotator coordinates $(x^\prime,y^\prime)$. Right ascension and 
declination (sky coordinates) are related by the fact that the sky
coordinates are left-handed and by the rotator position angle ($\omega$):

$$-\alpha = x^\prime \cos\omega - y^\prime \sin\omega$$
$$\delta = x^\prime \sin\omega + y^\prime \cos\omega$$

where the sign convention for the position angle is that a positive
position angle rotates N counter clockwise on the detector when viewed
in sky orientation.

Any instrument will then be defined by its pixel scale ($S_x, S_y$), rotation
of the pixels relative to the rotator coordinate system ($\theta$),
and the location of the detector center in the rotator coordinate system
$(c_x,c_y)$. If instrument coordinates are represented by $(x,y)$, then

$$x^\prime = (x-x_c) S_x \cos\theta + (y-y_c) S_y \sin\theta + c_x$$
$$y^\prime = (x-x_c) S_x \sin\theta - (y-y_c) S_y \cos\theta + c_y$$

$S$ and $\theta$ are determined by moving a star around in the detector
(command INIT, GINIT). In general, $S_x$ and $S_y$ will be constrained
to have the same absolute value (or to have a ratio of absolute values
given by the ratio of pixel dimensions); if the detector has a left-handed
orientation in sky coordinates (i.e. origin at upper left for sky
orientation), the scales will have the same sign, but if right-handed,
the convention is to make $S_x$ negative.

$c_x$ and $c_y$ are related to the offset which
is reqired to move a star from the center of the rotator to the center
of the instrument; if a QM A B moves the star from the center of the
rotator to the center of the instrument, then $c_x = A$ and $c_y = -B$,
where there is a sign flip for right ascension since sky coordinates
are left-handed while rotator coordinates are right-handed.

Pointing models are determined to put stars at the rotator center.

To place an object in the center of a desired instrument, one thus needs to
add an instrument correction to the ($\alpha,\delta$) of the target. Since
we need to move in the opposite direction of the offset, we add:

$$\Delta\alpha = c_x \cos\omega - c_y \sin\omega$$
$$\Delta\delta = -c_x \sin\omega - c_y \cos\omega$$

To offset the telescope in instrument coordinates, one has

$$\Delta x^\prime = \Delta x S_x \cos\theta + \Delta y S_y \sin\theta$$
$$\Delta y^\prime = \Delta x S_x \sin\theta - \Delta y S_y \cos\theta$$

or

$$\Delta \alpha = - \Delta x S_x \cos(\theta+\omega) 
                  - \Delta y S_y \sin(\theta+\omega)$$
$$\Delta \delta = + \Delta x S_x \sin(\theta+\omega) 
                  - \Delta y S_y \cos(\theta+\omega)$$

Remember, if one wishes to move an object a desired number of pixels, one
must move the telescope in the opposite direction.
\end{latexonly}

\subsection{Notes on guider software}
\begin{htmlonly}
See postscript version; HTML version is garbled!
\end{htmlonly}
\begin{latexonly}

The guider has two parallel stages, a positioning stage and a focus stage.
The positioning stage moves both the camera and the pickoff mirror, while
the focus stage moves the pickoff mirror only. Hence, changing the focus
also changes the position in the sky.

Let $P$ be the radial position in the sky in arcsec, $x$ be the radial
position in microns, $R$ the R-stage step position, and
$F$ the F-stage step position. Let $dR/dF$ be the ratio of R-steps to F-steps
which keeps $P$ constant. Let $dP/dR$ be the scale of R-stage, i.e.
arcsec/steps; $dP/dR = S dx/dR$, where $S$ is the telescope focal plane
scale. Then

$$P = [(R-R_0)+(F-F_0){dR\over dF}]{dP\over dR} + P_0$$

However, this is complicated by the fact that the guider focus changes with
position in the sky because the focal plane is curved. If we parameterize
this change by:

$$F(P) = F_0 + k_1 P + k_2 P^2$$ 

then

$$P = [(R-R_0)+(k_1 P + k_2 P^2){dR\over dF}]{dP\over dR} + P_0$$

Inverting to get $R(P)$,

$$(R-R_0) {dP\over dR} = (P - P_0) - (k_1 P + k_2 P^2) {dR\over dF}{dP\over dR}$$

or
$$R = {(P-P_0) - (k_1 P + k_2 P^2) {dR\over dF}{dP\over dR} \over {dP\over dR}} + R_0$$

\end{latexonly}

\section{Telescope calibration items}

After changing the optical configuration of the telescope, several calibration
steps may need to be taken:
\begin{enumerate}

 \item Determine rotator center. Probably best done by going to star field,
set position angle, disable rotator (ZDISABLE), change position angle by
less than 180 degrees, avoiding cable wrap issues, do a long (300-600s) 
integration and find the center.

 \item Do a pointing model. Put rotator center in pm.pro and do model using
recentering.  This creates pm???.fits files. Procedure posrecen.pro will
turn these into TPOINT input file. TPOINT generates a pointing model:
INDAT, CALL ALTAZ, USE NRX NRY, FIT, OUTMOD.

 \item Do instrument blocks. Need to determine offset from rotator center
to center of each instrument at PA=0 with QM commands. Guider CX coordinate
currently needs to be compiled into ccd/gccd.c.  GINIT should determine
scales and rotations, which together with offsets form instrument blocks.

 \item Determine proper guider focus (at rotator==0?) and set guider
home position accordingly.

 \item  Focus offsets, with altitude and rotator position/stage location 
for guider: foctesta.pro, foctestg.pro, focadj.pro. Focus offsets for filters.

\end{enumerate}

\section{Current telescope performance}

\subsection{Pointing/tracking}

The telescope after nominal mechanical optical alignment has poor
pointing performance. However, the pointing can be fairly well modelled
using standard terms for an alt-az telescope implemented, e.g., by the
TPOINT software, which we use. When making a pointing model using 100-200
stars around the sky, we typically achieve rms pointing performance 
of $\sim$ 20 arcsec.

Pointing accuracy appears to be limited by apparent small changes in
scale (encoder steps/degree for alt and az, motor steps/degree) with
time. The changes, at least in azimuth, do \textit{not} appear to
be correlated with the temperature.

In addition, we appear to very occasionally have major shifts in some
component that require us to do an entirely new pointing model. This
is not well understood, but it appears to happen very infrequently.

Tracking performace directly comes from this pointing performance because
we track using encoder information in the same way that we point.  The
tracking performance varies with location in the sky. Typically, we
estimate that moderately good tracking, i.e. image quality can be maintained,
for roughly 3-5 minutes.

\subsection{Guiding}

We have implemented guiding software to allow for longer integrations.
With the current system, we perform relatively slow guiding; we can
update roughly once every 1-2 seconds fastest. Our usual mode of operation
is to operate significantly slower, usually computing an offset every
1-2 seconds but averaging five of these before sending a correction to
the telescope.

The performance of the guiding still depends on the quality of the
pointing model because we only have position measurements of the guide
star in two coordinates while the telescope is a three axis system.
We currently assume that the rotator position is correct, and guide
in alt-az alone. This appears to be sufficient so long as the pointing
model is good; when it is not good, we get noticable rotation around the
position of the guide camera when we are guiding. Better guiding can
also be obtained if the observer is careful to rezero the coordinate
pointing immediately before guiding (i.e. setting the pointing to be
correct to first order).

Our current guiding performance has yet to be fully quantified.

\subsection{Image quality summary}

We are continually working on image quality. In its original condition,
the telescope never never achieved image quality much better than $\sim$ 1.7 arcsec,
with a strong wavelength dependence such that the best images were achieved
at the longest optical wavelengths, and significantly worse performace was seen
at shorter wavelengths.

Based on analysis of the images, we decided that the optics were likely
the cause, and we sent all three mirrors to be refigured/repolished in June 2002
to Rayleigh Optical Corporation in Baltimore. They found that the primary
mirror indeed did not have a very good figure. The mirrors were returned in
October 2003. 

Since then the image quality is better, although still leaves some room
for improvment. Our best images are now around 1.25 arcsec FWHM, with a
moderate wavelength dependence in the same sense as before. However, we don't
usually seem to be able to obtain this for longer exposures; currently,
we are investigating focus drifts, collimation drifts, and seeing effects.

%The image quality currently leaves quite a bit to be desired. We have

%Several things could potentially contribute to the poor image quality, which
%we summarize here; supporting data for some of these statements is
%presented in the next section.
%
%\begin{itemize}
%  \item misfigured optics: primary, secondary, and/or tertiary. We know
%     from out-of-focus images that optics errors are significant, as there
%     are strong non-uniformities that are very repeatable. The color
%     dependence of the image quality may be consistent with mid-frequency
%     errors.
%  \item collimation errors. We adjust the secondary on the sky to remove
%     obvious traces of coma. The procedure could leave some astigmatism;
%     also, we might have some astigmatism if we are not observing on the
%     optical axis (it is very difficult to adjust the tertiary). However,
%     there's not an obviously large component of astigmatism in the images.
%     We clearly have some spherical aberration; whether this is due to
%     improper spacing of the primary and secondary (and thus improper location
%     of the focal plane) or comes from optics not built to design specs
%     is still TBD. The color dependence of the image quality could not
%     be explained by collimation errors.
%  \item seeing errors. The fact that the image quality is \textit{never}
%     good suggests that seeing is not the dominant component, along with
%     the repeatability of out-of-focus intensity patterns. However, seeing
%     can be important at least at the beginning of the night; it is fairly
%     clear that the best image quality (as poor as it is), can only be 
%     achieved after the telescope has been opened for several \textit{hours}.
%     This suggests that thermal problems may dominate the image quality
%     at the beginning of the night.
%  \item tracking errors. The amplitude of these is currently not well
%     known.
%\end{itemize}
%
%At this point, we need a clear and quantitative understanding of the relative
%amplitudes of each of these problems, so we can determine which problem to
%attack first.
%
\subsubsection{Image quality data}

We have attempted to collimate the telescope several times. We have
twice done mechanical collimation by 
\begin{enumerate}
\item positioning the secondary (decenter and tilt) based on an alignment 
telescope mounted in the primary hole
\item positioning the tertiary (tilt only) based on an alignment telescope
mounted on the rotator mount
\item positioning the primary based on stellar images
\end{enumerate}

From this initial mechanical alignment, we have refined the collimation
using secondary tilts to minimize the apparent coma in the images. 
The secondary can be tilted using motor control, but not decentered.

\noindent{\textit{Comments from before mirror repolishing}}

Out-of-focus images show no strong signs of coma or astigmatism. However,
they do show very prominent, roughly azimuthally-symmetric, zonal
errors. At the current time, we suspect that the poor image quality
arises from one or more misfigured optics. The primary support system 
is also a possible culprit, but it is not clear how it would produce
azimuthally symetric features, as the primary support structure consists
of a set of three ``tripod'' cup mounts.

There is also some clear evidence of spherical aberation from 
\htmladdnormallink{out-of-focus image pairs}{980312pairs.jpg}. 
\latex{ This is shown in Figure \ref{fig:980312pairs}.} The various
pairs shown here were taken on 12 March 1998 
at 4 different back focal distances; the
top pair was taken at the ``nominal'' CCD position (with the old guider),
the next pair in 6 cm, the next pair in 3 cm, and the final pair
out 1.5cm from the ``nominal'' CCD position. Note with the new guider,
we are closest to the bottom position.

\begin{latexonly}
\begin{figure}
\epsfbox{980312pairs.ps}
\label{fig:980312pairs}
\end{figure}
\end{latexonly}

The image quality appears to be rather strongly dependent on wavelength,
despite all reflective optics (apart from the dewar window). Best image
quality as a function of wavelength is given by the following table:

\begin{tabular}{cc}
Filter&FWHM(")\\
Gunn Z& 1.7\\
I & 1.9\\
R & 2.1\\
V & 2.4\\
B & 2.7
\end{tabular}

The primary mirror was removed from the telescope in February 2000 to be
realuminized; the tertiary had to be removed at the same time, but was
not realuminzed, and the secondary was not removed at all. Upon reinsertion
of the primary, we redid the collimation being more careful about the
positioning of the secondary mirror. However, after several attempts at
collimation, it appears clear that the image quality \textbf{degraded}
since before the primary was removed. This is currently a large puzzle.
Focus data from October 2001 give:

\begin{tabular}{cc}
Filter&FWHM(")\\
I & 2.0\\
R & 2.2\\
B & 2.6
\end{tabular}

\begin{htmlonly}
Coarse focus run data obtained on 15 March 2000 is shown 
\htmladdnormallink{here.}{000315coarse.jpg}
\end{htmlonly}

\begin{latexonly}
Coarse focus run data obtained on 15 March 2000 is shown in Figure
\ref{fig:000315coarse}
\begin{figure}
\epsfbox{000315coarse.ps}
\label{fig:000315coarse}
\end{figure}
\end{latexonly}

Note that the vertical streaking seen in stars near focus is from the
CCD which was being operated with the shutter continually open.

Careful measurements have not yet been made on these images as compared
with previous images (not clear if sufficient archival data exists), but
by eye, it seems that the spherical \textit{might} be a bit worse now
than before; of course, this would be rather hard to understand.

\subsubsection{Sensitivity to collimation errors}

Errors in primary-secondary spacing will lead to the introduction of spherical
aberration. From Schroeder, we have
$$ASA = {m (m^2-1)\over 16 F^3} \left[1+{2\over (m-1)(m-\beta)}\right] {ds_2\over f_1}$$
For the 1m, we have $F_1$=2.5, $F=6$, giving $m=2.4$, 
$k=\rho (m-1)/m  = 0.37$, $\beta = k(m+1)-1 = 0.26$. So we have
$$ASA = 0.0055 {ds_2\over f_1}$$
or $ASA = 11.35$  arcsec per inch of secondary motion.

Note that the shift in the focal plane is given by $(m^2+1)ds_2 = 6.76 ds_2$.

%\subsubsection{Measuring aberrations from out of focus images}
%
%Spherical aberration. Idea is to compare ratio of diameter of obscurations
%inside and outside of focus where each image has the same image size for
%marginal rays, so we can imagine the images spaced equally around marginal
%focus. The location of focus for rays at the edge of the secondary is given
%by:
%$${D_{s2}\over D_{s1}} = {r_2\over r_1}$$
%where $r_1$ is the distance from the first image to the focus of the
%secondary edge rays, and $r_2$ is the distance from the second image to the
%focus of the secondary edge rays. We also have:
%$$r=r_1+r_2 \approx 2 F D_m$$
%where $D_m$ is the diameter of the marginal rays in the images 
%(same for both images, by construction). From these, we get the distance
%between marginal focus and focus of the secondary edge rays, which we will
%call $x$:
%$$x = r_1 - {r\over 2} = F D_m \left[{2\over (1+{D_{s2}\over D_{s1}})} - 1\right]$$
%
%We can relate this distance $x$ to the amount of spherical aberration (TSA)
%by:
%$${y^\prime\over TSA} \approx {y_2\over y_1}$$
%where $y$ is the ray height in the pupil, and $y^\prime$ is the ray height
%at paraxial focus. We also have:
%$${y^\prime \over x^\prime} = {TSA\over s}$$
%where $s$ is the distance between marginal and paraxial focus and
%$x^\prime = s-x$. Combining these gives:
%$$x=s-x^\prime = s (1-{y_2\over y_1})$$
%
%Note that
%$${2 TSA\over s} \approx {1\over F}$$.
%
%Combine results to get:
%$$ TSA = {x\over 2 F (1-{y_2\over y_1})} = 
%{D_m\over 2 (1-{y_2\over y_1})} \left[{ 2\over (1+{d_{s2}\over d_{s1}})} -1\right] $$
%
\end{document}
