A separate file can be prepared for each night's observing. The file should
have the name \textit{yymmdd.inp} where yy, mm, and dd refer to the 
\textbf{UT} date of the observation (i.e. the date after midnight).
The observing file is used to customize what sort of observations will be
made, various observational parameters, set of filters to use, and will
provide a list of potential targets and exposure times for each target.

A sample file can be viewed (in a separate browser window) 
%\htmladdnormallink{here}{http://loki.apo.nmsu.edu/tcomm/default.inp" target="_example}.

It is important to follow the correct file format. The first line is
a header line that is ignored, but it should be used to label the values
used in the second line.
The second line has 9 numerical entries:
\begin{enumerate}
  \item DTFOC:  minimum time interval between refocussing, in hours (suggest 2 
     hours?)
  \item DTEMPFOC: temperature difference that will trigger a refocussing 
     attempt, even if DTFOC has not elapsed, in degrees C (suggest 2?)
  \item PHOTOM:  flag to trigger automatic observations of standard stars,
     set to 0 for no standards, 1 to do standards
  \item DTLOW:  if PHOTOM=1, sets time interval between observations of low
     airmass standards with airmass between 1 and 1.3 
     (a pair will be taken each time)
  \item DTHIGH:  if PHOTOM=1, sets time interval between observations of high
     airmass standards with airmass between 1.6 and 2.0 
     (a pair will be taken each time)
  \item LDOME:  flag to determine whether the lower dome slit will be opened
     (1 to open, 0 to leave closed). The lower dome will begin to occult
     the telescope at an airmass of about 2.0. The reason not to open the
     lower dome shutter if you don't have to is because the lower dome
     shutter is less robust to both computer and mechanical failures than
     the upper dome, so there is a larger chance that it will not close if
     something goes wrong.
  \item NPM: number of pointing models to run before starting observations
  \item NFOCTEST: number of focus tests to run before starting observations
     (these are engineering tests, not normal focus runs, which are always
     done).
  \item DAYTEST: flag to use a test mode where dome is not actually opened,
     allowing script to be run during daytime or bad weather for testing
     purposes only; 0 for normal operation, 1 for test operation
\end{enumerate}

The third line is another header line, this time for filter information. The
fourth line stars with a number that gives the total number of filters that 
will be used for the night; each entry for each object will have to specify
exposure times for each of these filters (although one can certainly specify
0 length exposures to skip any desired filter for any desired object). After
the number of filters, there should be nfilt entries giving the filter names,
where the names are those as used by the control software for setting the
filter (i.e., u, b, v, r, i, z, c); case is not important (although I prefer
lowercase for consistency). For each of
the specified filters, the program will attempt to get twilight flats, 
and also observations of standard stars if the PHOTOM flag is set; 
consequently, you don't want to include filters in this list that you
won't need observations in.

The fifth line is a header line again, this time for the remaining object lines.

Starting with the sixth line, information is given for each desired
object.  The software works by observing the first object in the list that
satisfies a set of requirements. The requirements are that the object has
not already been observed the requested number of times, that sufficient
time (as defined by the observer) has elapsed to observe the object again,
and that the object falls into a desired window in airmass, UT, or hour
angle. If all conditions are met, then the object will be observed for
the amount of time in each filter specified on the object line.

\textbf{IMPORTANT: the object lines are parsed by looking for white space
between columns. As a result, you must not include any ``extra'' white space,
e.g. in an object name or in the coordinates!}

For each object, there needs to be $14+nfilt$ entries on the line. The 
entries for the object lines are:
\begin{enumerate}
  \item Object name. \textbf{No spaces are allowed!}
  \item RA, in format hh:mm:ss (No spaces allowed)
  \item DEC, in format +/-dd:mm:ss (No spaces allowed)
  \item Epoch of coordinates
  \item Minimum airmass to observe (at start of sequence)
  \item Maximum airmass to observe (at start of sequence). To ignore 
     airmass criterion, set minimum airmass to 0 and maximum airmass to 99.
  \item Miminum UT to observe (at start of sequence)
  \item Maximum UT to observe (at start of sequence). To ignore UT criterion,
     set minimum UT to 0 and maximum UT to 24
  \item Minimum HA to observe (at start of sequence)
  \item Maximum HA to observe (at start of sequence). To ignore HA criterion,
     set minimum HA to -12 and maximum HA to 12
  \item Maximum number of visits, i.e. after this many sequences have been
     obtained, this object will be skipped. To request ``infinite'' repeats,
     set this to a very large number.
  \item Minimum time between sequences, in hours. To request continuous
     coverage, set this to 0.
  \item Guide flag, 1 to use guiding during exposures, 0 for no guiding
     (guiding recommended for sequences over 60s).
  \item Exposure times in each of nfilt filters, in the order in which
     they are specified in the filter line (4th line of file). 
  \item Number of exposures to take in each filter during the sequence.
\end{enumerate}
